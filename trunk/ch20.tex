\chapter*{Мы строили, строили...}

Вот и подошло к концу наше путешествие в мир рендеринга. Для того, чтобы дать вам необходимые для дальнейшего самостоятельного
продвижения опорные точки, приведу небольшой список литературы. Список будет совсем небольшим, ведь мы не библиография, да и с появлением
Google с Амазоном любой может изготовить такой или подобный список за считанные минуты. Более того, мы, в отличие от остальной книги,
приведём наш список здесь, в конце главы. Раз уж так получилось, что мы так себе потихоньку прошмыгнули внутрь книжки про Maya, то не
будем сильно наглеть и требовать себе места ещё и в общем списке литературы.
  
 Итак:
  
     Первым пунктом у нас будет – эта {\bf книжка}.

      Странное пожелание, не правда ли? И тем не менее – дочитайте её до
      конца, не бросайте на середине. Поверьте мне – я видел {\it неотредактированные} исходники
      некоторых глав – и уже они были великолепны. Вы, скорее всего,
      больше ничего не узнаете про внешние рендереры из оставшихся глав
      книги – и тем не менее.

	\begin{description}
	\item [Advanced RenderMan: Creating CGI for Motion Pictures]   Основа основ
      Renderman. Начиная с азов, достигает нереальных глубин познания.
      Заставляет мозги работать в ранее неизведанных направлениях – лично
      я по прочтении этой книжки заинтересовался и занялся
      нефотореалистичным рендерингом. Интересно, в какую сторону
      повернутся мозги у вас?

	\item [Renderman Companion] Пока не появился Advanced
      RenderMan, эта книжка была основой основ. Капитально устарела,
      к сожалению. Я долго думал, стоит ли включать эту книгу, настолько
      она устарела и настолько она пугающе нацелена в первую очередь на
      программистов – но тем не менее, вот она, в нашем
      списке.

	\item [Texturing and Modeling: A Procedural Approach]    И опять основа основ, на
      этот раз для тех, кто всерьёз заинтересовался процедурными
      текстурами, историей и современностью этого вопроса. Берите
      исключительно третье издание, в нём куча нового материала и все
      картинки – цветные.

	\end{description}

 Ничего, что все эти книжки – на английском? Ну тогда продолжим. Несколько книжек для начинающих, но к сожалению, тоже не на русском:

\begin{description}
	\item [Essential RenderMan fast]   
	\item [Rendering for Beginners: Image synthesis using RenderMan]   Я не читал ни одну из них, поскольку к моменту их
    выхода начинающим уже не был, но хорошо знаком с автором первой и
    слышал много хорошего об авторе второй и посему – мои
    рекомендации.

\end{description}
  
 Далее – небольшой прыжок в сторону, для тех, кто
    хочет узнать, как вообще делаются рендереры:

	\begin{description}
		\item [Production Rendering, Design and Implementation.] Сборник эссе
	      о том, как делаются рендереры вообще и Renderman-совместимые – в
	      частности. В создании сборника участвовали авторы таких рендереров,
	      как Mantra, RenderDotC, Air, Aqsis, LightFlow и ART VPS. Отличная
	      книжка; жаль только, очень дорогая.
	  
		\item [Physically Based Rendering: From Theory to Implementation]  Учебник для
	      тех, кто хочет написать свой собственный рендерер. В процессе
	      изучения этого фолианта читатель имеет возможность проследить
	      процедуру создания своего рендерера, оснащённого под самую завязку
	      самыми современными технологиями, с подробным описанием
	      используемых алгоритмов – и с полным исходным кодом рендерера,
	      напечатанным прямо в книге (с использованием Literate Programming).
	      1000 страниц и почти 2.5 килограмма чистого счастья для тех, кто
	      понимает.  И под конец нашего списка:
	
		\item [Криптономикон.] А это просто книжка, которую я читаю сейчас, в момент написания
	      этого списка литературы. Она не имеет никакого отношения к предмету
	      разговора; просто она мне понравилась, может, кому-то из вас тоже
	      понравится. Не перепутайте случайно с Некрономиконом – это совсем разные
	      вещи.
	
	\end{description}

  
 И это ещё не конец, на самом-то деле. В наш век
    информационных технологий обойтись одними книжками в своих
    исследованиях вам будет – скажем так – трудно. С другой стороны, в
    наш век поглощений, закрытий и беспрестанных обновлений страниц,
    сайтов и компаний, приводить огромные списки URLей тоже вроде как
    неразумно. Так что, обойдёмся – сколько там у нас было книжек,
    девять? – девятью ссылками:

	\begin{description}
  
		\item [     renderman.ru



]
      Отбросив ложную скромность, приведём сайт, который я создал и
      поддерживаю (при неоценимой помощи и поддержке множества людей).
      Всё о Renderman на русском языке, но в первую очередь – сообщество,
      и лишь затем – учебник и помощник.

		\item [     renderman.org



]
      Корень всего. Полуофициальный  сайт, на котором есть всё о
      Renderman. Бонус-трек:
      www.renderman.org/RMR/OtherLinks/blackSIGGRAPH.html - история о
      закрытии Exluna.

		\item [     renderman.pixar.com
]



      Официальный сайт RAT и Prman. Там же – официальный форум
      поддержки.

		\item [     C.g.r.r. 



]
      Старая добрая конференция comp.graphics.rendering.renderman в
      Usenet. В последнее время не самое активное в плане обсуждения
      место, но в подавляющем большинстве случаев – ультраполезный
      источник информации.

		\item [     www.dotcsw.com/links.html ] 




      А вот если вы хотите найти список ВСЕХ Renderman-совместимых
      рендереров, когда-либо существовавших в природе – то вам сюда.
      Воистину, поучительное чтиво.

		\item [     rendermanacademy.com



]
      Академия Renderman. Сайт взял на себя трудную, но почётную
      обязанность учить народ Renderman-у, с чем в меру сил и умения
      справляется.

		\item [     film.nvidia.com



]
      Родина Gelato. Официоз и PR, в основном.

		\item [     jot.cs.princeton.edu
]



       Домашняя
      страница рендерера Jot. Что ещё сказать? Страница как страница,
      бывают и лучше, конечно.

 		\item [    Google.com 



]
      Этот адрес я мог и не писать, конечно. Найдётся всё, понимаешь.

      Бонус-ссылка: 		

\item [scholar.google.com ]
      специализированный поиск по научным работам, диссертациям и
      исследованиям.

\end{description}


 Я хотел бы поблагодарить нескольких человек, чья
    помошь в работе над этой главой не поддаётся трезвой оценке. Это
    Александр Сегаль, Сергей Невшупов, Константин Харитонов – без их
    комментариев, поправок и советов эта глава не состоялась
    бы.

 Отдельная благодарность Юрию Мешалкину, Егору
    Чащину, Александру Халявину и Вячеславу Богданову за
    предоставленные скрипты и плагины и разрешение их
    опубликовать.

 Вы можете найти всех этих людей на сайте
    нашего community – Renderman.ru.

 До встречи – и успехов!
