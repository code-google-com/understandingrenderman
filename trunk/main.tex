%!TEX TS-program = xelatex
%!TEX encoding = UTF-8 Unicode

\documentclass[11pt,a4paper]{book}
%\usepackage[russian,english]{babel}
\usepackage{geometry}                % See geometry.pdf to learn the layout options. There are lots.
\geometry{a4paper}                   % ... or a4paper or a5paper or ... 
%\geometry{landscape}                % Activate for for rotated page geometry
%\usepackage[parfill]{parskip}    % Activate to begin paragraphs with an empty line rather than an indent

\addtolength{\parskip}{\baselineskip}

\usepackage{graphicx}
\DeclareGraphicsExtensions{png}

\usepackage[russian,english]{babel}

 % \usepackage{verse}
   % \usepackage{textcomp} 

   % \usepackage{epigraph}
    %\usepackage{verbatim}
    %\usepackage{ulem}
    %\usepackage{sectsty}
    
% Surround parts of graphics with box
%\usepackage{boxedminipage}

% Package for including code in the document
\usepackage{listings}
\usepackage{color}

% Will Robertson's fontspec.sty can be used to simplify font choices.
% To experiment, open /Applications/Font Book to examine the fonts provided on Mac OS X,
% and change "Hoefler Text" to any of these choices.

\usepackage{fontspec,xltxtra,xunicode}
\defaultfontfeatures{Mapping=tex-text}
\setromanfont[Mapping=tex-text]{Droid Serif}
\setsansfont[Scale=MatchLowercase,Mapping=tex-text]{Droid Sans}
\setmonofont[Scale=MatchLowercase]{Droid Sans Mono}

\linespread{1.3}
\pagestyle{plain}

\title{Короткая глава в большой книжке,\\ которая раньше называлась\\ "Сношения с внешними рендерерами\\ или присовокупление к Renderman",\\ а теперь вышла\\ отдельной небольшой книгой\\ и поэтому должна называться\\ по-другому,\\ но я пока не придумал\\ название.}
\author{}
\date{}                                           % Activate to display a given date or no date

%\includeonly{ch7,ch8,ch9}

\begin{document}
\Rus
\tolerance=5000
%\tableofcontents
%\maketitle

\newcommand{\gr}[1]{\begin{center}\includegraphics[scale=0.3]{chapterRendermanOK_files/#1.png}\end{center}}

\definecolor{light-gray}{gray}{0.95}

\newcommand{\code}[1]{\lstinputlisting[frame=single, framerule=0pt, framesep=10pt, xleftmargin=10pt, xrightmargin=10pt, basicstyle=\ttfamily \small, backgroundcolor=\color{light-gray}]{#1}}

\chapter*{Введение}

Сколько я себя помню – всегда хотел написать книжку. В детстве мне казалось, что человек, который читает много книг, просто обязан уметь их писать. Это занятие я считал
настолько лёгким и непринуждённым, что от собственно написания книги меня всегда отделяла какая-то малость – домашнее задание в школе, например. Так что графоманом я
был неправильным – книжек не писал и не пытался, только думал о том, что когда-нибудь было бы так здорово...

Для тех, кому стало неинтересно читать эту главу уже сейчас, после первого абзаца, сообщаю – продолжение разговора про Maya будет уже скоро, всего через полсотни
страниц.

Для всех же остальных – после небольшой паузы, пока самые нетерпеливые листают в поисках начала новой главы – сообщаю: меня зовут Алексей Пузиков, и моя мечта сбылась.
Частично сбылась, конечно – пишу я не целую книгу, а только эту главу. И говорить в ней мы будем о внешних рендерерах, их связи с Maya и о Renderman.

Итак, что же пропустят наши неугомонные друзья?
\begin{itemize}
\item рассказ о том, зачем Maya нужны внешние рендереры
\item и как они работают в связке
\item а ещё про Renderman вообще
\item и про Photorealistic Renderman в частности
\item и про MTOR
\item и даже немножко про Gelato, Jot, GRUNT, MayaMan, Liquid и прочие вкусности, названия и аббревиатуры
\item и наконец, на десерт, они не прочтут краткое пособие по новому экстремальному виду спорта – Renderman-читингу.
\end{itemize}

Уже интересно? Ну что ж, давайте начнём!

Однако прежде всего – раз уж мы так весело начали и замахнулись на такой объём информации – поступим с читателем честно и укажем также, чего вы в этой главе не
найдёте. А не найдёте вы в ней подробных учебников, упражнений и уроков, справочных руководств и перевода спецификаций. Мир внешних рендереров необъятен, да что там
говорить – про один Renderman можно (и нужно!) написать не одну толстенную книжку – а у нас с вами в распоряжении всего лишь небольшая глава. Так что нам придётся
ограничиться неким набором ключевых моментов, опираясь на которые вы сможете самостоятельно продолжить изучение этого нового мира и выйти на качественно новый уровень в
своей работе.

% \gr{{chapterRendermanOK_files/image003}

\chapter*{Зачем?}
 

Итак, почему вообще возникла необходимость
   использования внешних движков визуализации в Maya? Возьму на
   себя смелость и выскажу основную мысль этой главы на её второй
   странице, не углубляясь в дебри объяснений и дискуссий –
   стандартный Maya software renderer непригоден для использования в сколько-нибудь серьёзном
   проекте, в особенности – кинопроекте. Не могу поверить, что я это
   сказал и – нет, это гораздо хуже, чем я предполагал – я написал это
   в книжке про Maya. Посмотрите
   кто-нибудь в окно – там небо в Дунай ненароком не упало?
 

Сказав А, скажем и Б - впрочем, рендерить чайники с
   помощью майского рендерера – можно, если их не очень
   много.
 

Теперь, когда разгорячённые толпы Maya-поклонников уже занялись моими поисками, попробуем понять,
   что же стало причиной такого провала в функциональности, в общем,
   очень высококачественного и продуманного продукта. Для начала,
   обратимся к первоисточникам, а именно к документу под названием “Design and Implementation of the Maya Renderer”, в котором
   описываются основные идеи, положенные в основу Maya software renderer. В
   качестве опорных концепций, ставших краеугольными камнями дизайна
   этого рендерера, выделены следующие вещи:
 


\begin{enumerate}
\item Стандартный рендерер Maya должен быть лучше, чем его предшественники – рендереры от TDI, Alias и Wavefront. В то же время, на начальных стадиях разработки нового
рендерера вместо него использовалось проверенное старое решение – Alias Renderer – которое затем по мере разработки заменялось готовыми частями нового рендерера.

\item Стандартный рендерер должен быть максимально удобным в использовании и предоставлять пользователю максимум возможностей для настройки и доступа к
“внутренностям”. Поскольку в качестве системы представления внутренних данных в Maya выбран DAG (прямой незамкнутый граф)– стандартный рендерер также должен быть
привязан к этой системе, предоставлять для неё свои собственные ноды для настройки самого процесса рендеринга и использовать ноды от HyperShade для настройки свойств
материалов.

\item Качество изображения должно быть высоким и предсказуемым.

\item Должно быть сделано всё возможное, чтобы ускорить рендеринг.

\item Поскольку рендерер является встроенным в Maya, в большинстве случаев он будет выполняться одновременно со своим материнским приложением, что означает
необходимость для рендерера быть гибким и нетребовательным по отношению к объёму необходимой для работы памяти (просто потому, что память нужна и самой Maya).

\item Новый рендерер должен быть по возможности универсальным и допускать расширение в будущем без необходимости полного переписывания.
 
\end{enumerate}


Что же в результате получилось у авторов Maya software renderer? Опять
   таки, не углубляясь в ненужные детали, укажем основные особенности
   готового продукта:

\begin{itemize}

\item      Стандартный рендерер Maya является
   адаптивным иерархическим сканлайн-рендерером с возможностью
   использования рейтрейсинга. Модные и потихоньку начинающие
   использоваться в продакшне Global Illumination, Final Gathering, Ambient Occlusion Mapping – не
   реализованы. Данный рендерер является триангулирующим. В процессе
   подготовки к рендерингу сцена бьётся на части, каждая из которых
   оценивается по количеству входящих в эту часть треугольников –
   таким образом, рендерер пытается не выйти за обозначенные ему рамки
   доступной памяти.
 

\item     Несмотря на объявленную гибкость –
   рендерер невозможно расширить за счёт собственноручно написанных
   плагинов. Всё, что можно сделать – реализовать собственные
   генераторы геометрии и дописать свои материалы для HyperShade – доступ к другим возможностям рендерера, не говоря уже о
   замене модулей или встраивании в его собственной пайплайн - не
   обеспечивается.
 

Впрочем, это была мелкая неприятность (ведь многие
   современные рендереры не позволяют даже такой гибкости). А вот
   дальше начинаются неприятности покрупнее.
 

\item      Для получения сколько-нибудь гладкой
   поверхности на любых поверхностях – кроме полигонов – этому
   рендереру требуется достичь очень высокого уровня тесселяции
   (разбиения на полигоны), что сказывается на общей
   производительности рендерера. Попросту говоря, любая попытка
   получить сколько-нибудь качественную картинку с
   использованием NURBS или Subdivision Surfaces обречена на долгое ожидание результата.
 

\item      Для получения теней рендерер использует
   по умолчанию технологию shadow mapping (хотя, как
   мы указывали ранее, существует возможность включения рейтрейсинга).
   Так вот, как показывает опыт, эти самые карты теней в Maya имеют ограничение по размеру не более 3200 по каждой стороне.
   Превышение данного ограничения приводит теоретически – к серьёзному
   замедлению рендеринга, а на практике – почти всегда роняет рендерер
   и вместе с ним – саму Maya. Такое
   поведение не позволяет получить сколько-нибудь качественные тени
   для кино, high definition video и вообще любых изображений высокого разрешения – например, для
   рекламных плакатов – при помощи shadow maps .
 

\item      Стандартный рендерер Maya отвратительно
   работает с большим количеством объектов и с большими разрешениями
   (больше 2048x2048). Причём,
   если при попытке отрендерить кадр в большом разрешении мы получим
   просто некоторое нелинейное замедление рендеринга (я мог бы
   сказать, что данное замедление является экспоненциальным, но, к
   собственному стыду, у меня не хватило терпения проверить эту
   гипотезу на практике и дождаться окончания рендеринга тестовых
   сцен), то при большом количестве объектов мы рискуем просто
   обвалить процесс.
 
\end{itemize}

Следует отметить, что разработчики Maya приложили максимум усилий для того, чтобы максимально улучшить
   свой рендерер и хоть как-то адаптировать его для условий реального
   продакшна. Однако возникает такое впечатление, что успех Bingo и других анимационных проектов, сделанных при помощи
   исключительно встроенного движка, настолько повлиял на умы
   программистов в Alias, что вместо
   того, чтобы действительно заняться оптимизацией своего детища, они
   пошли на поводу у собственного менеджмента и начали вводить в него
   новые возможности (впрочем, они поступили ещё хуже – начали писать
   новые рендереры). Таким образом, в Maya появились PaintFX renderer, vector renderer, volume renderer и другие
   приятные – не скрою – расширения стандартного рендерера – но это не
   исправило ситуации с основным движком и привело к тому, что (я
   пригнулся и прикрыл голову руками) им просто перестали пользоваться
   для сколько-либо серьёзных задач. В конце концов, это увидели и
   сами создатели Maya и смирились
   со своим поражением, встроив в программу в качестве стандартной
   опции Mental Ray. У
   них это получилось не так ловко, как у разработчиков из Softimage (что неудивительно, если вспомнить, сколько лет MR поставляется в качестве рендерера для Softimage и XSI), однако, тем
   не менее, уже сейчас в стандартной поставке Maya вы имеете
   хороший и стабильный продакшн рендерер.
 

Тем не менее, даже после встраивания MR в Maya вопрос с использованием внешних рендереров все еще
   не закрыт. Отчасти потому, что несмотря на растущую популярность и
   громкие пресс-релизы, Mental Ray пока не
   является таким же стандартом для студий, который представляет из
   себя Renderman. Ну и,
   конечно же, потому, что очень многие студии используют для
   рендеринга свои собственные наработки – достаточно вспомнить Blue Sky Studios со своим
   рейтрейсером CGI Production Studio или PDI/Dreamworks с
   внутренним рендерером (никогда не угадаете!) PDI Renderer, уж не
   говоря про многочисленные японские и французские студии,
   традиционно использующие для просчёта картинок собственные
   наработки.
 

Так или иначе, интерфейс к внешним рендерерам в
   Maya был всегда – собственно, было бы очень удивительно, если бы в
   продукте с такой продуманной архитектурой  его не было. Другой вопрос – что
   из себя этот интерфейс представляет.
 

Прежде чем продолжить свой рассказ, я хотел бы
   разрешить небольшую неоднозначность в определении и переводе
   терминов, чтобы не запутать вас и не запутаться самому. Итак, у нас
   есть 2 различных “интерфейса” – тот, что называется просто interface, и тот, который обычно принято обозначать как Visual Interface или GUI. Так вот, под
   первым мы будем понимать те параметры, функции, процедуры и
   объектные сущности, которые предоставляет разработчику плагинов
   или Mel-описателю Maya; под вторым –
   те окна, меню, кнопочки и прочие {\it интерфейсные элементы},
   которые Maya отрисовывает.
 

Так вот, несмотря на всю гибкость и всю
   направленность на реальные запросы студий интерфейса {\it внутреннего}, позволяющего встроить
   функциональность вызова внешнего рендерера в Maya, интерфейс {\it визуальный} до недавнего
   времени до установленной планки не дотягивал. У вас, несомненно,
   была возможность написать свой плагин, встроить его в Maya,
   создать для него окно настроек с помошью Mel, даже
   встроиться в главное меню программы – но вплоть до 5ой
   версии Maya у вас не было возможности встраиваться в святая
   святых рендеринга, окно Render Globals.
   Запрашивать значения переменных из этого окна было можно, а
   встроиться в него – нельзя (для любителей теории
   конспирации  - эту
   досадную мелочь исправили одновременно со встраиванием Mental Ray).
 

Отметим же торжество справедливости над косностью
   ума и реализуем возможность вставить в текст первую в этой главе
   картинку:
 
\gr{image001}

Вы уже заметили такие страшные слова, как “плагин”
   и “объектная сущность”, которые я употребил, описывая внутренние
   интерфейсы Maya. Более
   подробно про внутреннее устройство вы прочитаете в других главах,
   сейчас же мы вкратце рассмотрим сам процесс  работы Maya со внешними
   рендерерами.

\chapter*{Maya + внешние рендереры = ?}

 По вполне понятным как историческим, так и другим
    причинам (например, для более простой переносимости или для
    возможности запуска на кластерах) большинство внешних рендереров
    представляют собой программы,  выполняемые из  командной строки. Если вы ничего
    и никогда, кроме Макинтоша старых версий, в своей жизни не видели, то командная
    строка выглядит приблизительно так:

  \gr{image003}
  

 Мы познакомимся с ней подробнее чуть позже, а пока
    всего лишь отметим, что подавляющее большинство рендереров
    запускаются именно из командной строки.
  

 Небольшое отступление для наших читателей,
    пришедших из мира Autodesk/discreet/Kinetix – то есть для пользователей 3dsmax. Подавляющее
    большинство ВАШИХ рендереров – а вы ими совсем не обделены, стоит
    признать – реализовано в виде плагинов к самому Максу и интерфейса
    (опять это слово!) командной строки не реализуют. Очевидно, это
    имеет множество хороших сторон и как минимум одну плохую – ваш
    рендерер так просто к Maya не
    подключишь. Впрочем, мы отвлеклись.
  

 Таким образом, для того, чтобы заставить внешний
    движок просчитать картинку, вы должны произвести некоторые действия
    (приведём некий усреднённый алгоритм работы):
  

 	\begin{enumerate}
	\item                    Во-первых, ваш экспортер, т.е. тот
    плагин, который производит экспорт сцены из Maya во внешние
    файлы,  итеративно
    обходит всю сцену и экспортирует её геометрию и материалы (обычно -
    в некий новый файл собственного формата) .\hfil\break
    
Как известно, Maya внутри
    представляет все свои данные в виде набора нод; некоторые из них
    связанны в DAG (прямой незамкнутый граф). Наш плагин должен обойти
    все ноды в этом графе, передвигаясь от предков к потомкам; для
    каждой ноды мы должны определить, с кем она связана и какой тип
    информации она представляет, и в зависимости от того, поддерживает
    ли наш внешний рендерер данную информацию – использовать её или
    пропустить. Задача обхода сцены упрощается тем, что программист
    может заранее накладывать фильтр на граф перед тем, как проводить
    экспорт – например, если ваш рендерер поддерживает только
    полигональную геометрию, то и обходить вам нужно только
    соответствующие ноды. С другой стороны, в Maya ОЧЕНЬ много
    различных типов нод и, скорее всего, вы не захотите терять
    информацию только потому, что ваш рендерер не поддерживает тот или
    иной вид геометрии или материалов. Это значит, что вашему плагину
    также придётся заниматься преобразованием информации и приведением
    её в тот вид, который подходит для вашего рендерера. Очень многие
    рендереры используют свои собственные системы описания материалов,
    гораздо более простые или сложные по сравнению с теми, которые
    используются в Maya – но все они, как минимум, от неё отличаются, и
    правильный плагин также должен понимать природу этих различий и
    уметь их обходить.
  

\item                    Итак, ваш плагин продирается сквозь
    глубины DAG, сквозь все
    эти текстуры, материалы, полигоны, NURBS, SDS,
    кривые, локаторы, источники света, камеры, объёмные примитивы,
    трансформы, анимационные кривые и ключи, деформеры, динамику и
    системы частиц, слои и глобальные переменные. Некоторые рендереры
    воспринимают в виде входных данных файл, в котором описывается как
    сама геометрия, так и те материалы, которые присоединены к ней.
    Другие, напротив, разделяют эти понятия. Некоторые рендереры умеют
    хранить несколько кадров в одном файле, другие не умеют, или умеют,
    но не рекомендуют (просто исходя из того, что геометрия и материалы
    в сцене могут быть настолько сложны, что размеры полученных нами
    файлов будут очень велики).
  

 \item                    Все эти сакральные знания спрятаны внутри
    экспортера, который, пока вы читали этот абзац, уже закончил
    экспорт и оставил на диске один или несколько файлов, которые
    передаются в рендерер для просчёта. Не суть важно, что произойдёт в
    этом случае. Мы можем просто запустить локальную копию рендерера,
    передав ей файлы и какие-то другие параметры. Абсолютно
    естественно, что в процессе рендеринга может происходить как
    препроцесинг,  так и
    постпроцессинг данных, т.е. будут выполняться какие-то скрипты,
    программы или их комбинации, которые будут преобразовывать ваши
    данные перед рендерингом или по окончании оного. Если вы работаете
    над каком-то большим проектом, то (скорее всего) вы используете
    программно-аппаратное решение (в просторечии – “ферму”, официально
    – “renderfarm”), которое позволяет
    распараллеливать ваш рендеринг на несколько машин, находящихся в
    локальной сети, будь то рабочие машины сотрудников или  специальные компьютеры,
    выделенные для подобных расчетов. В таком случае вы не будете
    вызывать ваш рендерер напрямую, а запустите другую программу,
    которая уже и начнёт раздачу подпроцессов в вашей ферме.
  

 \item                    Как результат запуска постороннего
    движка, будет получена искомая картинка. Многие современные
    рендереры имеют возможность рендерить прямо в окно Maya. Благодаря
    этому, вы получаете достаточно удобное средство для
    предварительного просмотра вашего рендеринга.
  
\end{enumerate}


 Вот, собственно, и всё. Волшебство Maya +
    умелые руки программистов в сочетании с опытом работы во всех
    встречавшихся в процессе продуктах – и мы получаем неплохую основу
    для настоящего студийного pipeline, который
    объединяет в себе сильные черты всех имеющихся на вооружении
    продуктов.
  

 Кстати, давно хотел спросить – какого рода слово
    “пайплайн”? Считается, что это очень важное сакральное слово,
    которое нужно произносить несколько раз в день, особенно глядя на
    ошибки в своих перловых скриптах или на побитые картинки, оригиналы
    которых удалил композер. Только с обретением истинного смысла этого
    слова, а также после появления в штате студии собственного
    программиста, студия может считаться непомерно крутой. 
  

 Особенно правильным считается употребление слова
    “пайплайн” в письменном виде, в особенности на форумах Render.ru и в собственном резюме. Правда, я ни разу не видел
    рекомендаций по поводу использования этого мегаслова в книжках, и
    поэтому, прежде чем написать его ещё пару тысяч раз и задрать карму
    свою и читателей в необозримые выси – всё-таки, какого рода это
    слово? Как его правильно склонять?

\chapter*{Maya + Renderman = хорошо}
  

  Теперь, когда мы достаточно поверхностно (а я вас предупреждал!) рассмотрели теоретические основы подключения внешних рендереров к Maya, пришло время обратить
  внимание на практический пример, в качестве которого у нас с вами выступит связка Maya и Photorealistic Renderman, а именно конкретная реализация такой связки –
  продукт под названием MTOR.
  

 Почему мы с вами собираемся рассматривать именно эту связку?

В последнее время стал очевиден тот факт, что такой симбиоз продуктов – использование Photorealistic Renderman в качестве рендерера для Maya - стал стандартом де-факто
в индустрии визуальных эффектов; в особенности это касается визуальных эффектов для кино. Почему это именно так и почему считается, что такая связка является
стандартной - мы узнаем чуть позже, а пока сделаем некоторое отступление, глубокий вдох и остановимся на том, что такое Photorealistic Renderman и что такое Renderman
как понятие.
  

 Я, думаю, что вы уже обратили внимание на этот факт, как я легко и непринуждённо разделил две эти вещи. Такое разделение имеет под
    собой достаточно серьёзные основания.
  

 В свое время, когда мы (группа единомышленников, объединившаяся в рамках сообшества Renderman.ru) только начали заниматься Renderman и пропагандой этого стандарта в
русскоязычном интернете, приходилось достаточно часто сталкиваться с обращением новичков «Где скачать Renderman для Maya?». Я лично, будучи модератором посвященных
Renderman форумов, потратил многие часы своей жизни ответу на этот вопрос и его широко распространённую разновидность – Где Скачать Renderman для Макса? (Как показал
опыт, данное упражнение развивает как мышцы пальцев, так и любовь и терпимость к окружающим).

Оставив в стороне правовые стороны такой постановки вопроса (а именно ту его часть, которая “где скачать”), остановимся на том, что в его основе лежит широко
распространённое заблуждение.

Как оказалось, достаточно просто перепутать два понятия - понятие Renderman как стандарта и понятие Photorealistic Renderman (prman) как продукта, реализующего этот
стандарт. И основной причиной этой путаницы является cама компания Pixar, которая сначала создала стандарт Renderman, а потом начала распространение собственного
продукта с очень похожим именем. И что самое смешное, эта компания продолжает подливать масла в огонь беспорядку в наименовании, подготавливая к выпуску продукт,
который называется – Renderman for Maya! (Для окончательного запутывания народа Pixar’у осталось анонсировать Renderman for 3dsmax)
  
Что же  такое Renderman? 

Renderman - это стандарт описания трехмерных данных для их последующей визуализации. В сборнике часто встречающихся вопросов и ответов ньюсгруппы
comp.graphics.rendering.renderman написано так: “Renderman - это стандартный интерфейс между программами моделирования и рендеринга для создания качественного
фотореалистического изображения”. Работая над стандартом Renderman (а работа эта началась в доисторические времена - в 1987 году), компания Pixar поставила перед собой
достаточно амбициозную цель - пытаясь воспользоваться идеями формата файлов Postscript как некоей среды для обмена векторными двухмерными данными (а к тому моменту
Postscript был уже широко развит и использовался повсеместно), сделать аналогичный шаг и создать новый универсальный формат для обмена информацией между системами для
моделирования и анимации и системами для рендеринга, который был бы достаточно универсален для того, чтобы занять место Postscript в мире трехмерных приложений.
  

Основной особенностью стандарта Renderman является тот факт, что данный стандарт разделяет понятия геометрии в сцене и материала, который накладывается на данную
геометрию. Параметры и строение сцены описываются посредством Renderman Interface. В первоначальном варианте спецификации стандарта Renderman Interface являлся
интерфейсом для языка программирования C, с использованием которого вы могли написать программу, которая бы вызывала рендерер и передавала ему необходимые параметры.
Чтобы отрендерить прямоугольник, вам было необходимо написать приблизительно такую программу (текст программы адаптирован из книги RenderMan Companion):
  
\code{code/test.c}  

 Ничего особо страшного, но и приятного для человека
    с непрограммистским складом ума – тоже маловато. В качестве
    дополнительной возможности первой спецификации значился экспорт
    данных в промежуточный формат данных под названием RIB,
    что расшифровывается как Renderman Interface Bytestream. Вот
    так вот будет выглядеть та же самая сцена в этом формате (я получил
    её, окомпилировав приведенный выше код на С в исполняемый файл и
    запустив полученную программу):

\code{code/ex1.rib}  

 Ваша интуиция вас не подводит - впоследствии
    оказалось,  что в
    большинстве случаев гораздо удобнее работать именно с
    форматом RIB, который и
    стал основной частью Renderman Interface, хотя
    возможность использования C-интерфейса
    по-прежнему существует (для тех случаев, например, когда вам проще
    процедурно генерировать геометрию, чем моделировать её традиционным
    путём).
  

 Файлы в формате RIB, описывающие
    геометрию сцены, обычно имеют расширение (внимание!)
    *.rib. В нашем
    примере вы видели обычное текстовое представление этого формата,
    однако стандартом также предусматривается двоичное представление
    таких файлов. Такие двоичные файлы занимают гораздо меньше места и
    хорошо подходят для случаев, когда сцена не требует дальнейшей
    обработки и всё, что нужно сделать – передать файл рендереру. Как
    указано в спецификации формата, RIB изначально
    предполагался не как средство обмена геометрической и структурной
    информацией между различными системами моделирования и анимации, но
    как средство передачи данных от таких систем к рендерерам, то есть
    файлы такого формата не обязательно должны быть максимально
    удобными для импорта в программу-моделлер, но должны представлять
    максимум информации для последующего рендеринга.
  

 RIB является очень хорошо продуманным расширяемым форматом. Это
    сыграло большую роль в том, что, несмотря на свой приличный
    возраст, Renderman всё ещё опирается на RIB, время от
    времени добавляя новые возможности и команды – и сейчас, через
    почти что 30 лет со времени изобретения формата, мы можем
    использовать файлы данного формата c самыми
    современными версиями новых рендереров, задействуя новые
    возможности и технологии, о которых создатели стандарта не могли
    даже мечтать.
  

 Второй составной частью стандарта Renderman является язык Renderman Shading Language  или, в
    приблизительном переводе, язык закраски Renderman. Не в наших
    привычках пользоваться настолько корявым переводом, и потому в
    дальнейшем будем использовать общепринятую аббревиатуру - SL.
  

 Язык SL представляет из
    себя некоторое подмножество языка программирования С, которое
    позволяет нам описывать при помощи алгоритмов практически любые
    визуальные свойства материалов, присвоенных вашей геометрии,. Это
    значит, что мы не ограничены какими-то встроенными в программу
    моделями освещённости и стандартным набором процедурных текстур,
    как это сделано в очень многих рендерерах - в нашем распоряжении
    находится вся мощь современного языка программирования, специально
    адаптированного для этой задачи.
  

 Дабы избежать короткого замыкания в мозгах
    неискушённых пользователей Maya, подойдём к
    вопросу с другой стороны. В Maya ведь есть
    шейдеры? И в чём их отличие от шейдеров в Renderman?
  

 Ответим на первый вопрос, а заодно определимся с
    терминологией. Итак, шейдер – это некая программа на языке
    программирования, которая описывает визуальные свойства того или
    иного объекта (например, поверхности модели). В данный момент нам
    не так важно, на каком языке написан шейдер и каким образом он
    вызывается – главное, что мы знаем, как это сделано и как
    работает.
  

 Так вот с этой точки зрения майские материалы
    шейдерами не являются. Вы можете скомбинировать ноды в HyperShade в нужном порядке, получив тот или иной материал –
    но шейдера в результате вы не получите.
  

 Очень часто в литературе и разговорах встречается
    словосочетание “процедурная текстура”. Исходя из вышесказанного, и
    шейдеры, и материалы являются процедурными текстурами – в отличие
    от текстур непроцедурных, или говоря языком родных осин –
    картинок.
  

 Вообще говоря, идея процедурного описания
    визуальных и геометрических свойств  поверхности не являлась новой
    даже для 1987 года, в котором появился стандарт Renderman. Идея
    эта является простой экстраполяцией того факта, что при
    программировании на низком уровне мы ВСЁ РАВНО описываем и
    геометрию, и визуальные свойства объектов в виде неких алгоритмов
    (ключевое слово для программистов: OpenGL). Теперь
    же, когда мы поднялись на более высокий уровень и для рендеринга
    картинки уже не нужно писать больших и сложных программ (а
    достаточно нажимать большие кнопки с красивыми иконками)–
    оказалось, что процедурное описание закраски всё ещё остаётся
    полезным, поскольку является наиболее гибким способом такого
    описания. Процедурные текстуры обладают и другими неоспоримыми
    достоинствами, например, малым (по сравнению с обычными) размером
    файлов; отсутствием зависимости от разрешения получаемой картинки.
    Но несмотря на все преимущества подхода, до появления Renderman написание шейдеров носило в основном экспериментальный
    характер; именно Renderman сделал
    шейдеры полноценным индустриальным стандартом; язык написания
    шейдеров SL является, возможно, наиболее известным и
    распространённым среди других шейдерных языков.
  

 Как мы уже говорили, шейдер на SL - это программа
    на Си-подобном языке программирования, описывающая алгоритмически
    визуальное представление объектов, а если быть точнее – описывающая
    алгоритм расчета интегрированной освещенности и прозрачности для
    данной точки поверхности геометрического примитива. Для большего
    удобства в этом языке были сделаны некоторые изменения, например,
    были введены специальные типы данных, которые описывают цвет и
    позицию в трёхмерном пространстве; синтаксис языка и его
    стандартная библиотека процедур были упрощены, оптимизированы и
    дополнены по сравнению с оригинальным языком С для работы с такими
    типами данных; был веден набор глобальных переменных, несущих
    информацию о цвете источников света, положении и ориентации
    освещаемого сэмпла и так далее, существенно облегчающих
    программирование шейдеров.
  

 Простой шейдер на языке SL, реализующий
    металическую изотропную модель освещения поверхности, будет
    выглядеть так:
  
\code{code/ex1.sl}  

 Как можно видеть, любой более или менее сведущий в
    программировании (в особенности на языке С) человек может
    достаточно быстро научится писать данные шейдеры как вручную
    (многие студии и разработчики всё ещё предпочитают этот путь), так
    и с использованием специальных визуальных инструментов разработки
    шейдеров.
  

 При помощи шейдеров в Renderman можно
    контролировать не только модель отображения поверхности
    (surface shaders), но и влиять на её геометрию (displacement shaders),
    управлять поведением источников света (light  shaders),
    прохождением света через область сцены (volume shaders) и общим
    отображением информации в виртуальной камере (imager shaders):
  

  \gr{image005}
  

 Работа с шейдерами в SL также упрощена
    путём введения модели “чёрного ящика”. В переводе на человеческий
    язык это означает, что в каждый момент времени наш шейдер имеет
    дело с одной конкретной точкой в сцене и отвечает на вопрос “Что
    происходит в данной конкретной точке сцены”? Об этой точке шейдер
    знает всё: где она находится, какой цвет назначен геометрии в этом
    месте по умолчанию, какими геометрическими параметрами она
    обладает. Шейдер в такой модели представляет из себя “чёрный ящик”
    с некоторым набором “ручек” – вы засовываете исходные параметры в
    этот ящик, поворачиваете ручки в нужные позиции и получаете готовые
    данные, и так – для каждой точки в вашей сцене. Обратной стороной
    медали в данном случае служит тот факт, что в рамках шейдера вы в
    общем случае не имеете возможности обратиться к данным, которые
    относятся к СОСЕДНЕЙ точке этой же сцены, этого же полигона или
    шарика.
  

 Где-то там наверху я говорил вам, что в
    стандарте Renderman есть 2
    основные части? Я оговорился, их таки три, и на закуску я оставил
    самое интересное.
  

 Еще одной стороной стандарта, о которой мало кто
    упоминает,  является
    тот факт, что этот стандарт определяет не только форматы файлов
    которые применяются для обмена информацией между системами анимации
    и моделирования и  системой рендеринга, но и набор требований, который позволяет и
    другим разработчикам создавать различные рендереры, которые могли
    бы принимать подобные файлы.
  

 На самом деле,  стандарт никоим образом не
    указывает вам, какого рода рендеринг должен происходить и какого
    рода обработка должна производиться с вашими файлами. Ваш рендерер
    может быть рейтрейсером или использовать технологию сканлайн. Вы
    можете поддерживать Global Illumination,
    можете быть REYES (Renders Everything You Ever Saw), как prman, объёмным рендерером, гибридным, использовать A-Buffer, какие-то
    другие технологии или их сочетания – стандарт не оговаривает таких
    подробностей и  не
    ограничивает вас в деталях реализации. Всё, что нужно сделать для
    того, чтобы ваш собственный рендерер стал официально совместимым со
    стандартом Renderman – это
    выполнить некий набор специальных требований (описанный в
    спецификации стандарта и в книге Renderman Companion), а
    именно:
  
\begin{itemize}

\item поддерживать некоторое количество стандартных геометрических примитивов и понимать формат файлов RIB

\item поддерживать шейдеры во всём их разнообразии и, соответственно, файлы SL

\item поддерживать внутри себя иерархическую структуру сцены в соответствии со стандартом

\item содержать в себе стандартный набор из 15 шейдеров, также описанных в стандарте 

\end{itemize}

 а также делать некоторые другие, не настолько
    интересные и важные вещи. И это – весь список. Любые другие детали
    реализации движка рендеринга остаются на совести разработчика; в
    стандарте есть только небольшой список того, что было бы здорово
    иметь:

\begin{itemize}
 \item     Ray tracing
 \item      Global illumination
 \item      Level of detail
\item      Depth of field
\item      Motion blur
\item      Area light sources
\end{itemize}

 Подобная гибкость стандарта вкупе с его
    открытостью, продуманностью и массированным продвижением
    компанией Pixar не осталась незамеченной. Семена легли в
    благодатную почву всеобщего интереса к 3хмерной графике (области
    знаний на тот момент только развивающейся и новой, а потому вдвойне
    интересной), специальным эффектам в кино (помните – Terminator
    2... Jurassic Park…
    Как давно это было...) Добавим в коктейль отличный референсный
    продукт от самих Pixar – и получим
    воистину взрывоопасную смесь, которая немедленно
    сдетонировала.
  

 Вокруг стандарта Renderman немедленно образовалась новая ниша продуктов – рендереров,
    экспортеров из всевозможных систем анимации и моделирования,
    вспомогательных программ. Новичок в индустрии, стандарт сам {\it стал индустрией}. На
    данный момент существует больше десятка рендереров, являющихся
    совместимыми с Renderman (ссылка
    на постоянно обновляемый список будет приведена в конце главы);
    фактически не существует такого пакета 3D-моделирования
    который бы не выводил в RIB сам или при
    помощи плагинов.
  

 Более того, благодаря своей открытости и
    доступности Renderman вошёл в программу обучения многих
    американских университетов. На крупнейшем мировом форуме,
    посвященном компьютерной графике (я имею в виду, конечно же, Siggraph) Renderman’у посвящаются отдельные курсы, проходят
    собрания Special Interest Groups of Renderman
    и Birds of Feathers (не
    просите меня это перевести), на которых разработчики и пользователи
    стандарта и совместимых с ним продуктов встречаются, обмениваются
    новыми идеями. Renderman живет
    своей отдельной жизнью и пользуется огромной популярностью в
    студиях и среди непрофессиональных пользователей (благо, рендереров
    хватает на любой вкус, как высококачественных и дорогих, так и
    бесплатных с открытым доступом к исходному коду).
    Фактически, Renderman стал
    одним из столпов современной индустрии визуальных
    эффектов.
  

 Я очень хотел бы закончить наш краткий экскурс на
    этой мажорной ноте, но, как говорится, в жизни всё не так, как на
    самом деле.
  

 Историческое развитие индустрии компьютерной
    графики привело к тому, что Renderman не стал универсальным
    стандартом, которого придерживаются все и любые системы рендеринга.
    Это абсолютно нормальное явление, конкуренция хороша и нужна везде.
    Так или иначе, компания Pixar отказалась от роли технологического
    лидера и проповедника, сконцентрировавшись взамен на своей
    деятельности как анимационной студии (за что им огромное спасибо).
    Таким образом, компания прекратила продвижение стандарта Renderman, перестала выдавать лицензии на соовместимые рендереры
    и взамен стала продвигать тот самый референсный рендерер Photorealistic Renderman. Эра всеобщего благолепия и братства людей-цветов закончилась
    в августе 2002 года, когда Pixar подала в суд
    на разработчиков одного из конкурирующих рендереров,
    компанию Exluna, с
    обвинениями в нарушении патентного права. Таким образом,
    руководимая Стивеном Джобсом компания устранила одного из своих
    основных конкурентов на рынке Renderman-совместимых рендереров. Можно долго спорить, кто был прав или
    виноват в данном случае, но факт остаётся фактом – мир изменился
    навсегда, Renderman повзрослел.
  

 Ну вот, на мажорной ноте закончить не получилось,
    на минорной ноте обрывать историю - не хочу. Да и не нужно её
    обрывать, на самом деле, потому что на этом месте история
    заканчивается и начинается сегодняшний день. День, когда prman и стандарт Renderman  сам по себе имеют множество
    последователей по всему миру, и в том числе среди русскоязычных
    пользователей; день, когда технологии, описанные данным стандартом,
    нашли применение и признание как в зарубежных, так и в
    отечественных студиях. День, когда историю пишем мы.
  

 Не знаю, получилось ли у меня  на одном вдохе рассказать о
    стандарте Renderman. Мне
    очень хотелось рассказать о нем побольше; очень многое просто не
    влезло или не очень соответствует формату “книжки про Maya”.
    Я не рассказал вам многое из истории стандарта и продукта, я не
    приводил никаких сравнительных таблиц и не публиковал картинок и
    схем. Всё это можно найти в книгах и на ресурсах в Интернет, ссылки
    на которые я собрал в конце главы. Ну а теперь пришло время
    выполнить данное ранее обещание и объяснить, почему именно
    связку Maya-Photorealistic Renderman используют в своём пайплайне ведущие продакшн хаусы мира (не
    утруждая себя полным перечислением – вряд ли оно возможно вообще –
    укажем лишь ILM, Weta Digital, Disney, Sony Pictures Imageworks, Moving Picture Company, Framestore, Tippett Studio, Jim Henson Studios, наконец,
    сам Pixar – список уже можно назвать “места работы моей
    мечты”).
  

 Как только студия принимает решение использовать в
    своём пайплайне спецэффекты с использованием того или иного
    рендерера, к нему предъявляются некоторые достаточно специфические
    требования (в просторечии – “быть production-ready”):

	\begin{enumerate}

\item  Набор возможностей рендерера не должен
    ограничивать пользователя, это очевидно. Какие-либо ограничения на
    количество или качество входных или выходных данных недопустимы.
    Даже если вам кажется, что никто никогда не сделает сцену со 100
    миллиардами полигонов – вспомните Билла Гейтса с его знаменитой
    фразой о 640 килобайтах памяти, которых хватит всем (1981
    год).
  

\item                  Рендерер должен понимать множество типов
    входной геометрии. В современных условиях даже самый быстрый и
    качественный рендерер, оперирующий только полигонами, обречён на
    забвение.
  

\item                   Рендерер должен представлять широкие
    возможности настройки. Ничто так не радует TD (вы ведь
    знаете, что это такое? Technical Director – в
    просторечии, компьютерный разработчик спецэффектов), как мощный
    набор инструментов для баланса между качеством требуемого
    результата, скоростью рендеринга и требуемой для него
    памятью.
  

\item                   Более того, рендерер должен быть гибким.
    Лучшее доказательство гибкости рендерера – это когда его используют
    не по прямому назначению. Хороший пример – производство
    нефотореалистичной анимации (так называемый “картун” – cartoon animation) при
    помощи “фотореалистичных” рендереров.
  

\item                    Надёжность. Давайте посчитаем. 70 минут
    среднего полнометражного полностью 3хмерного мультфильма. Вы врядли
    работаете именно над таким проектом, но помечтать никто не мешает.
    Так вот, 70 минут или 4200 секунд. 24 кадра в секунду. Чуть больше
    100 тысяч кадров – и это только чистового рендерера, то есть с
    учётом проверочных рендерингов, внесённых по ходу изменений,
    раздельных рендеров по слоям и служебных рендеров (текстуры, тени,
    бейкинг другой информации) мы эту цифру можем запросто умножить на
    6. Кто-то прокричал “тридцать”? Мне нравятся ваше стремление к
    совершенству и ваши бюджеты, но мы останемся с нашей цифрой.
    Впрочем, вы таки подпортили мне настроение, да и цифра 6 мне
    сегодня решительно не нравится – поэтому давайте умножим на 7.
    700000 кадров. По часу, скажем, на кадр. 80 машино-лет непрерывного
    рендеринга. Так вот смысл наших расчётов состоит в том, что production-quality renderer не должен падать. Софта без ошибок не бывает, в особенности
    такого сложного и наукоёмкого, каким является рендерер – и тем не
    менее, ПАДАТЬ РЕНДЕРЕР НЕ ДОЛЖЕН. Каждое падение добавляет новые и
    новые перерендеринги к вашей задаче и забирает деньги (которых в
    бюджете и так мало) и время (которого до дедлайна вообще почти не
    осталось).
  
\item                   Надёжность. Вы не ошиблись и я не
    оговорился. Написать это во второй раз не помешает.
  

\item                   Производительность. Помните цифру
    наверху? 700 тысяч кадров. Которые вам нужно отрендерить на вашей
    старенькой ферме за полгода. И не стоит рассчитывать на закон Мура
    (“Мощность вычислительных устройств удваивается каждые 18 месяцев”)
    – в нашей индустрии действует другой закон, Блинна – “Вне
    зависимости от скорости компьютера, кадр считается одинаковое
    количество времени”. Как только вы проапгрейдите свои компьютеры
    – TD немедленно найдёт, чем занять их новые
    мощности.
  

\item                   Предсказуемость. TD должен быть в
    состоянии догадаться, насколько увеличится время рендеринга при
    добавлении того или другого элемента. Если вы добавили мелкий
    элемент в сцену и время обработки увеличилось вдвое – это не ваш
    выбор.
  

\item                   Последовательность. Ваш рендерер умеет
    рендерить все поддерживаемые примитивы с применением Motion Blur. В
    день X авторы программы добавили поддержку нового
    примитива, который пока что не понимает MB. В день X+2
    именно с этой проблемой вы столкнётесь в вашем проекте. Все
    “навороты” используемого вами решения должны понимать друг друга и
    взаимодействовать соответствующим образом – за исключением
    бессмысленных случаев – и, возможно, даже в этих бессмысленных
    случаях.
  

\item                 Качество результата. Мы поставили его в
    конце, и этот параметр иногда сложно измерить – но на самом деле,
    это самый главный параметр, самое главное требование, основа всех
    основ в рендеринге. Если вы делаете кино, то ваши картинки будут
    показаны на большом (на самом деле – на ОЧЕНЬ БОЛЬШОМ) экране, и
    каждый грязный и неправильный пиксель будет иметь размеры, которые
    можно будет определить при помощи большой деревянной линейки
    (такие, знаете, которыми так удобно размахивать, взявшись двумя
    руками). Любые артефакты, будь то плохой антиалиасинг, неправильная
    апроксимация геометрии, ошибки отсечения, полосы Маха –
    недопустимы. Более того, скорее всего вы делаете анимацию – а это
    означает, что также важно соответствие кадров друг другу во всех
    деталях. Неподвижные объекты не должны дёргаться, медленно
    движущиеся – прыгать; антиалиасинг должен быть правильным и
    одинаковым для объектов любого размера, находящихся на любой
    дистанции от камеры.
\end{enumerate}

 Ух! Объяснение у нас получилось очень длинное и
    очень простое – prman полностью
    удовлетворяет всем приведённым выше требованиям. А теперь сравните
    этот список с другим, относящимся к стандартному рендереру Maya.
  

 Убедил? Тогда продолжим.
\chapter*{Ужасная история в поезде}
 

Тихий зимний вечер. Мы вместе с классом возращаемся
   из поездки в Прибалтику. Поезд проносится мимо тёмных полей и
   лесов. Парни внизу играют в карты, на полке напротив тоже что-то
   происходит – но мне всё равно, я читаю. Мне только что дали
   почитать СуперПовесть.
 

Я чуть ли не единственный участник поездки, который
   ещё не успел прочитать эту повесть. Она хит сезона, она ходит по
   рукам, от неё невозможно оторвать, очередь выстроилась на неделю
   вперёд. Класс разделился на тех, кто прочитал Произведение - и со
   смаком обсуждает перипетии сюжетных поворотов - и тех, кто ещё не
   успел и может только тупо кивать, слушая эти обсуждения.
 

Страницы уже немного замусолены, этот десяток
   листков был выдран из какого-то новомодного журнала для молодёжи,
   который уже не боится печатать “про секс”, но всё ещё не научился
   правильно пользоваться компьютерной вёрсткой.
 

И вот оно, счастье – хозяин манускрипта едет со
   мной в одном купе – или это всё-таки был плацкарт? Память услужливо
   подсовывает мне любые детали, кроме этой. Я пользуюсь моментом – и
   встреваю без очереди, выпросив эти пожмаканые листики у хозяина и
   клятвенно пообещав, что быстренько прочитаю и верну.
 

О, что это была за повесть! Отложив в сторону её
   содержание (а именно, борьбу бравого зелёного берета за выживание в
   диких джуглях Амазонии), стоит отметить исключительно занятный
   художественный стиль. Ломаное повествование всё время бросало меня
   из одного ключевого момента в другой. Казалось, что в произведении
   совсем нет сюжетной линии, и лишь к тому моменту, когда я осилил
   примерно половину листков – скрючившись под еле светящейся
   лампочкой на второй полке ПЛАЦКАРТА, Я ВСПОМНИЛ, уже далеко за
   полночь, безумно уставший, борясь со сном, со слипающимися и
   немного слезящимися глазами, под монотонный стук колёс и
   покачивание вагона разогнавшегося поезда – я наконец-то понял, что
   произведение построено наоборот – сюжетная линия ведёт меня от
   конца к началу повествования, время от времени подбрасывая новые
   заморочки в своём движении к финалу апофеоза.
 

Я уже говорил, что на листиках не было номеров
   страниц?
 

Всю чудовищность произошедшего я осознал лишь в тот
   момент, когда увидел на последней странице ЗАГОЛОВОК повести. Как
   оказалось, всё это время я боролся с сюжетом и плохим освещением,
   читая повесть наоборот – от последнего листика к
   первому.
 

К чему бы я городил воспоминания школьных лет в
   книжке про Maya, спросите вы? Да просто потому, что, по моему
   искреннему мнению, начинать рассказ о Prman с описания
   функциональности Renderman Artist Tools (RAT) – это всё равно, что
   начинать читать интереснейшую книжку с конца. Нет, не так – это всё
   равно, что начинать читать учебник с конца. Это антипедагогично,
   антинаучно, антиморально и вообще неправильно.
 

И именно это мы собираемся сделать сейчас – начать рассказ с RATа.
 

Конечно, можно было бы всё-таки послушать зов
   своего сердца и сделать из этой главы академический учебник –
   начать с азов, постепенно продвигаясь ко всё более и более сложным
   вещам. Но так уж получилось, что самое интересное и самое
   презентабельно выглядящее в prman – это 2 большие разницы. Да и, в
   конце концов, мы же не ставим себе целью усыпить нашу аудиторию
   посреди главы?
 

Поэтому придётся поступиться принципами и построить
   обзор возможностей комбинации prman+Maya в точности так, как я
   когда-то читал ту самую повесть – то есть задом наперёд. Сначала мы
   с вами рассмотрим RAT, входяшие в него утилиты, их особенности и
   алгоритм работы с ними. Затем пойдём назад (или всё-таки вперёд?),
   опустимся чуть поглубже в потроха рендерера и узнаем, как оно всё
   там внутри работает и вызывается.
 

Кстати говоря – несмотря на то самое происшествие в
   поезде, я всё ещё не могу избавиться от дурной привычки читать с
   конца журналы. Хотя в последнее время поступаю так всё реже и
   реже.
\chapter*{RAT или Рендерим Визуально}
 
Прежде всего, введём в употребление некоторые
    аббревиатуры, используемые в Пиксаровских продуктах.
    Итак:
  
     prman –
      PhotoRealistic renderMAN или, официально, Pixar’s
      RenderMan – собственно, рендерер от Пиксар. Я уже использовал это
      обозначение раньше, теперь просто приведу его для
      закрепления.

     RAT
      – Renderman Artist Tools – набор инструментов и
      плагинов для связи Maya и prman. Если кто-то называет его “крысой”
      – знайте, перед вами новичок. Закоренелый трёхмерщик непременно
      назовёт его “рат” (а не “рэт”, как этого следовало бы ожидать, и уж
      никак не “ЭрЭйТи”).

     MTOR
      – Maya To Renderman – собственно плагин
      для Maya, который позволяет ей рендерить при помощи
      Renderman-совместимых рендереров. Входит в состав RAT.

 Продуктовая линейка Пиксар на момент написания этих
    строк включает в себя 3 продукта: Renderman Pro Server (включающий
    в себя prman, Irma и Alfserver), RAT (в составе MTOR, Slim, Alfred,
    “it”) и Renderman For Maya. Нужно было бы, конечно, для
    лицензионной чистоты расставить в предыдущей фразе значки ™ в
    произвольном порядке, ну да ладно.

 Ещё раз - для того, чтобы хоть что-то отрендерить,
    вам понадобится Pro Server. Для того, чтобы связать Maya и Pro
    Server, нужен RAT:

 \gr{image007}

 Renderman for Maya - отдельный новый продукт,
    являющийся боковой веткой эволюции prman и RAT, предназначенный для
    более удобной увязки этих продуктов и нацеленный, в первую очередь,
    на людей, ранее с Renderman не сталкивавшихся. Мы не будем
    рассматривать Renderman For Maya в этой главе детально, поскольку
    на момент подготовки этой главы продукт всё ещё не был выпущен в
    продажу и не был доступен для бета-тестирования.

Для
    любознательных: само название
    RenderMan имеет очень интересную историю. В самом начале у
    основателей Pixar был vision – они хотели создавать “железо”,
    аппаратные решения для рендеринга. В идеале, такой микрокомпьютер
    должен был быть очень маленьким и носимым чуть ли не в кармане. Из
    таких компьютеров создавались бы “render walls”, объединяющие свои
    усилия в рамках одной задачи; производством такого железа могли
    заниматься кто угодно – но оно было бы полностью совместимым,
    поскольку подчинялось бы стандарту Renderman. По аналогии с
    появившимся в то время карманным плейером для кассет фирмы Sony,
    который назывался Walkman, будущее устройство назвали
    Renderman.

Новый микрокомпьютер
    так никогда и не появился в продаже, но имя – и стандарт –
    прижились.
  
\section*{MTOR}

 Первое, с чем столкнётся пользователь Maya, установивший комплект из Pro Server и RAT (а работать с ними он будет именно в такой связке)
– это MTOR. В интерфейсе MAYA это проявится вот таким вот образом:

\gr{image009}

 то есть в вашем меню появится новый пункт – RenderMan:

\gr{image011}

 Что произойдёт после того, как вы (предварительно
    загрузив какую-нибудь существующую сцену), выберете пункт меню
    RenderMan=>Render? А произойдёт вот что: MTOR последовательно
    обойдёт все элементы сцены и {\it постарается} вывести их в файл
    формата RIB. Кроме того, он пройдётся по геометрии и источникам
    света и назначит для них шейдеры в RIBе – но сделает это очень
    хитро: фактически, MTOR не будет учитывать текстуры и сложные
    материалы, эмулируя лишь некоторый стандартный набор шейдеров
    (Ambient Light, Directional Light, Point Light, Spot Light, Phong,
    Blinn, Lambert – эти шейдеры легко опознать по сигнатуре имени
    файла mtor*.slo). Наконец, будет запущен рендеринг, который вызовет
    окно Alfred и выведет результаты в окне it.

 Как видите, всё переплелось и взаимосвязалось.
    Попробуем разобраться в клубке из аббревиатур и программ, и сделаем
    это с помощью простого тестового примера.

 Набросайте в сцену несколько примитивов и добавьте
    источники света – пусть это будет простой майский
    directLight:

 \gr{image013}

 В общем, уже можно рендерить. Что мы и делаем:
    RenderMan=>Render.

 Мы только что вызвали на выполнение MTOR – плагин,
    позволяющий Maya рендерить  с помощью Renderman-совместимых
    рендереров. Плагин этот встраивается в интерфейс Maya, но поскольку
    был написан достаточно давно, ещё во времена первых версий Майки,
    когда встраиваться в Render Globals и Rendering Window не
    позволялось – проявляется в виде меню и вызываемого из этого меню
    набора собственных окон.

 На плечах MTOR лежит задача вывода геометрии в
    формат RIB и вызова других программ из набора. Также он
    осуществляет поддержку Renderman-compatible subdivision surfaces в
    интерфейсе Maya – видите в меню RenderMan последний пункт? Это
    оно.

 Сильной стороной MTOR является его
    программируемость. Встраиваясь в Maya, MTOR добавляет несколько
    новых команд Mel; более того, будучи сам написан на языке
    программирования Tcl, MTOR позволяет также выполнять скрипты на
    этом языке, что делает его необычно гибким в применении и адаптации
    к различным пайплайнам.

 По моему мнению, RAT – это попытка технарей из
    Pixar усидеть сразу на двух стульях, а именно - сделать
    одновременно удобный для неопытного и расширяемый опытным
    пользователем продукт. В общем, где-то табурет у них и получился –
    пользоваться более-менее удобно, ручки все на месте, настроить под
    свои требования – тоже можно. С другой стороны, у такого подхода
    есть и свои проблемы, но о них потом.
  
\section*{Alfred}

 А у нас тем временем (сцена несложная и много
    времени на экспорт в RIB не ушло) появляется окно
    Alfred:

 \gr{image015}

 Alfred – система распределения рендеринга по
    локальной сети. Даже если вы не используете сетевой рендеринг, эта
    система будет установлена и включена по умолчанию. Это может
    пригодиться, например, если у вас многопроцессорная система; в
    таком случае вы сделаете вид, что у вас на самом деле 2 компьютера
    – и voila.

 Alfred – достаточно продвинутый сетевой инструмент,
    который используют во многих студиях не только в составе RAT, но и
    для сетевого рендеринга из Maya, 3dsmax и даже After Effects, Shake
    и Nuke, благо под Renderman он не заточен, а в качестве файла
    задачи в нём выступает обычный Tcl скрипт с командами. Но несмотря
    на такую гибкость и на то, что Alfred используется в работе {\it самими} Pixar, качество
    кода этого продукта от релиза к релизу заметно ухудшается и потому
    в последнее время появилась тенденция ухода студий на
    альтернативные (в том числе и собственной разработки) сетевые
    диспетчеры.

 Раз уж мы заговорили об Альфреде, обозрим и его
    лучшего друга – Альфсервер.
  \section*{Alfserver}

 Alfserver – это собственно агент, выполняющий
    задания Alfred. Название этой программы несколько сбивает с толку,
    поскольку мы привыкли, что сервером называется одна головная
    программа, к которой подсоединяется множество клиентов. В данном
    случае центрального рендеринг-сервера, раздающего задачи клиентам,
    нет и применяется обратная аналогия – клиент-диспетчер задач один
    (это Alfred) и он сам распределяет задачи по серверам (Alfserver),
    установленных на рендеринг-машинах в ферме. Очевидно, что клиентов
    может быть много (скажем, клиент может быть установлен на
    компьютере у каждого аниматора и работать по ночам, пока аниматору
    снятся анимационные кривые), и все они будут распределять свои
    задачи по серверам.

 Для
    продвинутых: конечно, всё не так
    просто. При работе с Alfred существует
    возможность использовать утилиту {\it maitre\_d}, которая будет
    выступать в качестве центрального сервера – но основноё принцип
    работы от этого не изменится, поскольку утилита эта сама
    распределять задачи не может, а всего-лишь собирает информацию о
    занятости систем и прочих метриках и раздаёт её всем желающим – и
    локальные диспетчеры используют эти метрики для оптимизации
    нагрузки рендеринг-фермы.
  \section*{It}

 А тем временем мы почти сразу же после появления
    окна Alfred видим новое окно – it - в котором быстро проявляется
    наша картинка:

 \gr{image017}

 “it” (Image Tool)– в оригинале вьюер для
    графических файлов, в последних версиях вобравший в себя
    функциональность просмотрщика последовательностей, каталогизатора
    файлов и даже композера. Я более чем уверен, что большинство
    пользователей RAT не используют даже 50\% возможностей этой
    программы – а зря.

 По умолчанию it выступает в качестве display target
    при рендеринге; это удобно, если вы хотите посчитать несколько
    различных картинок подряд, а затем сравнить их или выбрать лучшую
    из серии.

 Сравним полученную картинку с результатом
    рендеринга при помощи родного движка от Maya:

 \gr{image019}

 Очень похоже, не правда ли? Ничего, в общем,
    удивительного – сцена примитивная и потому нашему крутому рендереру
    prman негде было развернуться во всей своей красе.

 Вы наверняка заметили, что размеры картинок,
    полученных при рендеринге через RAT и родным рендерером Maya, не
    совпадают. Самое время узнать, каким образом настраивается RAT – а
    именно, вызвать RenderMan=>RenderMan Globals.

 \gr{image021}

 Подробное описание всех функций этого окна оставим
    для документации по RAT – кстати говоря, несмотря на некоторую
    запутанность и огромный размер, она обладает одним немаловажным
    преимуществом – в ней есть ВСЁ.

 Пытливый читатель уже обнаружил параметр Display
    Resolution. На нашем скриншоте этот параметр уже выставлен в 0, 0 –
    таким образом мы всегда будем рендерить в размере выбранного
    viewport’а Maya; по умолчанию же этот параметр выставлен в 640x480.
    Кстати говоря, обратите внимание на значки i рядом с именами параметров –
    это контекстная справка. Про возможность установки (0,0) я узнал
    именно из неё.

 Для
    любознательных: как видите, разрешение
    рендеринга в Maya и RAT не сихронизированы. Оставим для других
    обсуждение правомерности этого поступка; сами же скажем, что задачу
    эту можно достаточно просто решить с помощью механизма скриптования
    MTOR – для этого достаточно немного покопать в документации в
    сторону RibBox, mattr и Майского нода resolution.
  
\section*{SLIM}

 Вернёмся к MTORу. Чего он делать, к сожалению, не
    умеет – так это конвертировать майские материалы в рендерменовские
    шейдеры. По идее разработчиков RAT, этого от программы и не
    требуется, поскольку считается, что все шейдеры будут либо писаться
    руками, либо создаваться при помощи SLIM – визуального конструктора
    шейдеров.

 О SLIMе можно говорить очень долго. После
    Hypershade он может показаться несколько архаичным, но тем не менее
    является достаточно удобным конструктором для шейдеров.

\gr{image023}

 SLIM, как и Hypershade, реализует древовидную схему
    построения шейдеров из блоков, но не использует при этом
    визуализацию в виде дерева (такая визуализация появилась в
    последних версиях, но не является особо функциональной и потому
    зачастую попросту не используется).

 Для
    любознательных: древовидная схема для
    визуального построения шейдеров – да и вообще для визуального
    построения алгоритмов и программ, которыми и являются шейдеры –
    идея абсолютно не новая и уже серьёзно заезженая. В том или ином
    виде, визуальные конструкторы шейдеров в виде деревьев используются
    уже практически во всех современных системах анимации и
    рендеринга.

 На самом
    деле, когда я занимался историей этого вопроса, для того чтобы
    кратко осветить его в этой вставке для любознательных, я был
    буквально ошеломлём тем количеством информации, которая доступна в
    Интернете по поводу Визуального Программирования – а ведь
    визуальное построение шейдеров суть визуальное программирование и
    есть.

 Как
    оказалось, первая графическая компьютерная система,
    продемонстрированная в 1963 году Сазерлендом (она называлась
    SketchPad – для тех, кто уже запустил верный Google) уже была
    оснащена визуальным редактором для быстрого построения программ – и
    количество разработанных с тех пор систем визуального
    программирования огромно.

 Так почему же
    мы до сих пор не строим все наши программы визуально? Почему
    программисты шейдеров в той же Пиксар до сих пор пользуются vi и
    emacs, создавая свои сотни тысяч строк кода шейдеров (цифры
    реальные)? Этому можно посвятить отдельную статью, а то и книгу;
    скажем лишь, что у визуального представления программ есть некий
    порог сложности, начиная с которого их становится сложно
    редактировать и эта возрастающая сложность редактирования и
    понимания нивелирует все достоинства метода.

 Создаваемые в SLIM материалы назначаются на
    геометрию в Maya, добавляя к шейпам новые кастомные атрибуты с
    необходимой для MTOR информацией.

 Более того, таким же образом (через кастомные
    атрибуты из SLIMа) вы можете добавлять в сцену свой собственный
    RIB-код при помощи так называемых RIBbox (мы их уже упоминали в
    отступлении для любознательных).Дальше – больше: расширяемый, как и
    сам MTOR, при помощи Tcl, SLIM позволяет вставлять в сцену
    TCLbox’ы, которые будут исполняться на этапе экспорта сцены в RIB и
    имеют полный доступ как ко внутренним интерфейсам самой Maya, так и
    к командам MTOR и SLIM.

\begin{quotation}
Для любознательных: основной атрибут шейдера называется slimSurf; короткое имя, под которым он фигурирует в скриптах и в файлах *.ma –
sss. Выглядит в Maya Ascii-файле это примерно так:
\end{quotation}

\code{code/ex1.ma}

 Вернёмся к нашим тестовым баранам. Давайте теперь
    немного усложним нашу сцену. Поработайте немного над вашими
    примитивами. Раз уж заговорили про парнокопытных, то у вас должно
    получиться что-то вроде этого:

 \gr{image025}

 Не получилось? Не расстраивайтесь, у меня тоже не
    получилось, поэтому я просто взял готовую модель (автор модели –
    Ник Габченко, за что ему отдельное спасибо) из файла sheep.ma (вы
    можете найти его на диске).

 Эта модель – полигоны, переведённые в subdivision
    surfaces при помощи команды Modify=>Convert=>Polygons to
    Subdiv. Давайте опробуем SLIM на деле и назначим овечке материалы.
    Renderman=>Slim=>New Palette.

 \gr{image027}

 В меню SLIM выбираем File > Create Appearance
    > Surface > Velvet – пусть овечкина голова будет бархатной.
    Двойной клик по шейдеру Velvet, клик по области Preview:

 \gr{image029}

 Как видите, это довольно сильно напоминает Attribute Editor для
    стандарных майских материалов.

 Темноватый бархат получается, сделаем его немного
    светлее. Изменяем Kd, поскольку амбиентного источника у нас в сцене
    не будет (говорят, иметь их в сцене вредно), изменяем цвет, всё
    время проверяем результат, кликая на шарике. Добившись вменяемого
    результата, назначаем созданный материал: в Maya выбираем голову
    овечки, затем в основном окне SLIMа выбираем Appearance=>
    Attach. Рендерим (вы уже вынесли пункт меню RenderMan=>Render на
    полку?)

 \gr{image031}

 И чтобы уже совсем сделать наш краткий туториал
    похожим на все остальные 1327 туториалов для начинающих,
    заполонивших Интернет, покажем, как собственно делаются в SLIMе
    более сложные составные шейдеры.

 Для
    любознательных: число 1327 выбрано
    случайно. Кстати, о случайных числах – вот вы знаете, например, как
    отличить TIFF файл по сигнатуре? Оказывается, очень просто – у него
    3им и четвёртым байтом записано число 42 – как сказано в
    спецификации, an arbitrary but carefully chosen number (42) that
    further identifies the file as a TIFF file. Я думаю, вопрос о том,
    почему выбор не пал на  43 или 41 – неправомерен, поскольку они не являются 5ым Каталонским
    Числом.

 Так вот, рядом с Velvet color кликнем на жёлтом
    квадратике и в выпавшем меню выбираем Connection. В этом же строке
    справа в вываливающемся списке выбираем Noise. Должно получиться
    что-то такое:

 \gr{image033}

 После клика по кнопке Noise мы попадаем в редактор
    уже самого шума.

 И опять для
    любознательных: придумал нойз и
    первым реализовал шейдеры в продакшн рендеринге один и тот же
    человек – Кен Перлин.

 Быстренько задираем величину frequency в заоблачные
    дали (а именно, где-то в 10), кликом по диагональной стрелочке
    возвращаемся к исходному Velvet и сразу же видим результат на нашей
    овечке:

  \gr{image035}

 Исправить ситуацию с UV-меппингом головы овечки,
    из-за которой нойз так постыдно размазался, мы оставим в качестве
    домашнего задания для читателей, а сами закончим наш туториал на
    самом интересном месте и продолжим рассматривать программы и
    утилиты, входящие в состав RAT. Их осталось не так уж и много.
    Честно скажем, осталась одна
  
\section*{Irma}

 Irma – ре-рендерер. В общем, идея проста – вы
    рендерите один раз с помощью Ирмы, которая кэширует необходимую
    информацию о сцене, освещении и шейдинге. Все последующие разы
    ре-рендеринг использует эту информацию и происходит гораздо
    быстрее, без перегенерации RIB-файлов и шейдеров. Irma понимает
    изменения параметров шейдеров, позиции источников света и изменения
    в координатных системах – такие изменения будут просчитаны очень
    быстро. Ближайший аналог Irma из мира Maya – это, конечно же,
    IPR.

 Что бы не говорили, а бОльшую часть интересных
    новых идей уже кто-то придумал до нас. Взять, например, ту же Ирму
    с IPRом. Самая первая реализация шейдеров в продакшне – та самая,
    которую реализовал на Фортране Кен Перлин – использовала те же
    идеи: предварительный рендеринг всего, что можно, в буффер и
    последующее наложение шейдеров. С другой стороны, с тех пор утекло
    очень много воды, и в современных рендерерах бОльшую часть времени
    тратится именно на шейдеры. Тем полезнее для вас будет
    Ирма.

 \chapter*{prman или Закапываемся.}

 В 1642 году известный голландский мореплаватель
    Абель Тасман во главе экспедиции из двух кораблей отплыл из
    Джакарты на поиски новой земли на стыке Тихого и Индийского океана.
    В ходе своих скитаний путешественник открыл Тасманию, Новую
    Зеландию, острова Фиджи – и при этом умудрился проплыть мимо
    Австралии!


 Я ничего не имею против Новой Зеландии (скорее
    наоборот), но человек, пытающийся использовать prman исключительно
    при помощи RAT, не рассматривая взаимодействие этих программ,
    уподобляется славному голландскому первооткрывателю – как Тасман
    всё-таки открыл Австралию через 2 года после первого путешествия,
    так и незадачливый исследователь Renderman всё равно откроет для
    себя командную строку. И вот тогда он, наконец, поймёт, сколько
    всего интересного скрывал под поверхностью океана
    айсберг.


 К сожалению, не многие проходят этот путь, цепляясь
    за визуальные инструменты и пытаясь как можно дольше не погружаться
    в глубины EXE-файлов и их параметров. Их позывные слышны издалека:
    “Я художник, я не понимаю этого вашего всего, оно мне не нужно,
    пусть программисты копаются, а мне сделают мегакнопку”.


 И очень даже напрасно, хочу вам сказать, потому что
    истинная сила всех standalone рендереров – именно там, в командной
    строке.

\section*{prman.exe}

 Большинство Renderman-совместимых рендереров
    состоит из 3х программ – собственно рендерера, компилятора шейдеров
    и компилятора текстур.


 Не столь важно, получилось ли так исторически или
    все рендерман-совместимо-рендерерописатели смотрели на Pixar в
    качестве образца – но подобная тройственная архитектура сохраняется
    с теми или иными отклонениями во всех таких рендерерах. Отчасти в
    этом есть некая дань стремлению к максимальной оптимизации процесса
    и разделению рендерера на независимые модули; отчасти – это отклик
    Unix-овского наследия; отчасти – это слепок самой идеи, заложенной
    в спецификацию Renderman (согласно которой, процедурные материалы
    (шейдеры) и описание геометрии хранятся в разных файлах,
    соответственно, в *.SL и в *.RIB). Так или иначе, тенденция есть и
    она сохраняется даже в случае с самыми современными рендерерами –в
    их поставке вы тоже обнаружите 3 exe-шника: рендерер, компилятор
    шейдеров и конвертер текстур.


 В этом изысканном трио основная программа – это,
    конечно, сам рендерер – prman.


 Вызовите свою любимую командную строку (если вы
    используете Windows, то для этого нужно в пункте стартового меню
    Run запустить cmd.exe) и уже в новом окне командной строки
    запустите на выполнение prman.exe. 

\gr{image037}


 У новичков может возникнуть впечатление, что
    программа зависла, потому что сразу после запуска ничего не
    произошло и обратно в командную строку мы не вернулись. На самом
    деле мы столкнулись ещё с одной особенностью программ, изначально
    написанных в расчёте на консоль Unix – эти программы предназначены,
    в том числе, и для работы в режиме пайпинга, то есть передачи
    данных от одной программы к другой. Так вот, если такую программу
    запустить на выполнение, не указав параметров, то она будет
    ожидать, что данные будут поступать на вход от других программ (для
    продвинутых: из stdin) или будут набираться с клавиатуры. Значит,
    всё, что нам нужно сделать – это запустить программу с
    параметрами.


 Остановим этот экземпляр рендерера при помощи
    клавиатурной комбинации Ctrl-C. Возьмём образец текстового
    RIB-файла из начала нашей главы и сохраним его как test1.rib.
    Стойте, не нужно судорожно листать наше повествование в обратном
    направлении, вот необходимый код (он также есть на
    диске):

\code{code/rib_example/test1.rib}

Повторим наш экзерсис с рендерером, на этот раз
    передав программе в качестве параметра имя test1.rib:


 \gr{image039}


 Как видите, в результате получилось... хм... ничего
    у нас не получилось. В чём причина? Причин на самом деле несколько,
    и все они достаточно очевидны:
  
     В нашей примитивной сцене нет источников света.
      Поэтому сцена получилась полностью неосвещённой (умное название для
      эффекта “ничего не видно”)
     В нашей сцене используется (это не очень очевидно)
      полигональный объект, расположенный таким образом, чтобы полностью
      покрывать видимое пространство (это совсем неочевидный факт, но
      можете поверить мне на слово).
  

 Использовать в качестве упражнения на рендеринг в
    книжке полигон, находящийся в чёрной-чёрной сцене – это, в общем,
    готично, но при этом не совсем педагогически правильно. Поэтому нам
    с вами нужно немножко поработать над нашим RIBом, чтобы
    получить хоть сколько-нибудь удобоваримую картинку. Я сделал эти
    исправления вручную в текстовом редакторе, вы можете просто
    скопировать текст из книги или из файла test2.rib:

	\code{code/rib_example/test2.rib}


 Записываем этот файл поверх старого и повторяем
    вызов рендерера:

 \gr{image041}


 Вот, совсем другое дело. Поздравляю вас с
    приобщением к великой могучей командной строке и обществу её
    последователей и почитателей.


 Если вы хорошо рассмотрели скриншот с командной
    строкой, то обратили внимание, что команда, которую мы выполняли,
    выглядит вот так:

\begin{lstlisting}[frame=single, framerule=0pt, framesep=10pt, xleftmargin=10pt, xrightmargin=10pt, basicstyle=\ttfamily \small, backgroundcolor=\color{light-gray}]
prman test1.rib
\end{lstlisting}


 Это не единственный возможный способ передачи
    информации в рендерер; как один из примеров, мы можем
    воспользоваться тем самым пайпингом (или туннелированием, как его
    называют некоторые несознательные юниксоиды-русофилы):

\begin{lstlisting}[frame=single, framerule=0pt, framesep=10pt, xleftmargin=10pt, xrightmargin=10pt, basicstyle=\ttfamily \small, backgroundcolor=\color{light-gray}]
type test1.rib | prman
\end{lstlisting}


 Результат будет в точности такой же. Но это ещё не
    всё. Вы же знаете, что туннелировать можно не только 2 программы,
    но и целые цепочки программ?


 В качестве примера (скриптоненавистники могут
    перевернуть страницу) напишем небольшой скрипт на Perl, который
    будет заменять объявление материала Plastic на объявление материала
    Stone. Делаем новый файл, называем его magic.pl и пишем нечто
    подобное:

\begin{lstlisting}[frame=single, framerule=0pt, framesep=10pt, xleftmargin=10pt, xrightmargin=10pt, basicstyle=\ttfamily \small, backgroundcolor=\color{light-gray}]
foreach (<>) {
  s/plastic/stone/;
  print;
}
\end{lstlisting}

 
Наша цепочка выполняемых программ в данном случае
    немного изменится и будет выглядеть так (я переименовал
    RIB-файл):

\begin{lstlisting}[frame=single, framerule=0pt, framesep=10pt, xleftmargin=10pt, xrightmargin=10pt, basicstyle=\ttfamily \small, backgroundcolor=\color{light-gray}]
type test2.rib | perl magic.pl | prman
\end{lstlisting}
  
 Читаем написанное слева направо: распечатать в
    консоль файл test2.rib и передать результат в программу perl,
    которая выполняет файл magic.pl, который заменяет все нахождения
    слова plastic на слово stone и опять печатает в консоль. Вывод этой
    консоли передаётся собственно рендереру:

\gr{image043}

 
Для того, чтобы это сработало на вашей машине (если вы не сидите сейчас за Unix-терминалом), вам понадобится установить на своём
компьютере интерпретатор языка программирования Perl (например, ActivePerl).


 Лирическое
    отступление: Perl, по искреннему
    мнению многих, и я к нему присоединяюсь – язык программирования для
    истинных криптоманьяков. Расшифровывать свои собственные скрипты
    через полтора года с момента  последнего их использования –
    занятие, доставляющее нереальное удовольствие. Так прямо себе и
    представляю: Блетчли-Парк, Вторая Мировая война, взломавший Энигму
    Тьюринг никак не может разобраться в перловом скрипте из 4х
    строк.


 Вы уже слышите, как тихонько зазвучало в воздухе
    исконно русское слово “pipeline”? Просто подумайте о тех
    невероятных возможностях, дверь к которым мы только что приоткрыли.
    Получается, что геометрию для рендерера совсем не обязательно
    выводить из Maya (или 3dsmax, или Houdini, или на самом деле
    неважно откуда именно) – вы можете сами писать небольшие программы
    или скрипты, которые будут или создавать новые файлы RIB, или
    модифицировать уже существующие. Более того, файлы эти можно
    запросто открывать в текстовых редакторах, делать замены,
    переставлять куски местами, экспериментировать с различными
    возможностями – и всё это без запуска и использования безумных
    систем моделирования и анимации. Да что уж там – изучение текста
    сгенерированных RIBов представляет из себя захватывающее
    интеллектуальное психоделическое путешествие в мир 3хмерной графики
    и во внутреннее устройство вашего любимого моделлера. Как
    подействует это изменение в модели на RIB? Что изменится, если
    сделать то или это? Какая картинка получится в том или ином
    случае?


 Лирическое
    отступление: Что уж говорить – я
    сам прошёл через это увлечение текстовыми рендерерами, но у меня
    оно началось не с Renderman-совместимых. Первым рендерером, который
    я запустил на своём новеньком, свежесобранном (Тёма, спасибо тебе!),
    с иголочки 486DX4-100, был Vivid. Где-то в то же время на моём
    жестком диске гостили ещё и DKB с PovRay’ем в ранних версиях
    (собственно, PovRay является развитием DKB, если мне не изменяет
    память), но Vivid – это было наше всё. Быстрый рейтрейсер, простой
    входной формат данных, полноцветный выход – целых 24
    бита!


 Не один вечер
    провёл я в экспериментах с картинками и скриптами, в попытках
    сделать тот или иной элемент, материал, эффект. Рендерер этот был
    действительно быстрым, быстрее PovRay чуть ли не в десятки раз
    (жаль, он не запускается под Windows XP – уж очень хочется
    проверить эти мои старые ощущения на новом железе). Такая скорость
    позволяла мне за вечер перепробовать множество вариантов, получить
    множество красивых – и не очень – результатов. В общем, это была
    любовь с первого зеркального шарика (стоящего на шахматной доске –
    а как же без этого).


 Ау, [название
    удалено по понятным причинам] продукт отечественного 3хмерного
    софтостроения! На совести вашего криво написанного деинсталлятора –
    все мои картинки, исходники к ним, равно как и всё остальное
    содержимое моего славного боевого друга и товарища - диска С!
    Надеюсь, в славном 199x-ом году вам славно
    поикалось.

 
Первый
    Renderman-совместимый был потом, и был это – BMRT. Но это уже –
    современная история, с Интернетом, Амазоном, Пентиумами и
    ньюсгруппами.


 Впрочем, я немного увлёкся.
\section*{Другие RIBы}

 Отметим, что у RIBов бывает не только текстовое, но
    и бинарное (оптимизированное для быстрой загрузки рендерером) и
    упакованное (оптимизированное для уменьшения занимаемого дискового
    пространства – используется библиотека упаковки GZIP)
    представление. Объединённого бинарно-упакованного представления
    нет, но его несложно сымитировать при помощи той же самой командной
    строки:
  
\begin{lstlisting}[frame=single, framerule=0pt, framesep=10pt, xleftmargin=10pt, xrightmargin=10pt, basicstyle=\ttfamily \small, backgroundcolor=\color{light-gray}]
catrib -binary test2.rib | gzip -cq9 > test2.gbin
\end{lstlisting}

 Слева направо: программа catrib (из поставки Pixar
    Renderman Pro Server) переводит файл test2.rib в бинарный формат и
    результат передаёт в программу gzip (вам придётся скачать её из
    Интернета), которая упаковывает её и передаёт дальше, в файл
    test2.gbin.
  

 Чтобы теперь отрендерить файл test2.gbin, развернём
    цепочку программ в другом направлении:
  
\begin{lstlisting}[frame=single, framerule=0pt, framesep=10pt, xleftmargin=10pt, xrightmargin=10pt, basicstyle=\ttfamily \small, backgroundcolor=\color{light-gray}]
gzip -cd test2.gbin | prman
\end{lstlisting}
  

 Что и требовалось доказать.
  

 Для
    продвинутых: Я вам по секрету скажу
    – на самом деле можно делать и так:
  
\begin{lstlisting}[frame=single, framerule=0pt, framesep=10pt, xleftmargin=10pt, xrightmargin=10pt, basicstyle=\ttfamily \small, backgroundcolor=\color{light-gray}]
prman test2.gbin
\end{lstlisting}
  

 Prman, а
    также некоторые другие рендереры, понимают такие файлы и отлично с
    ними работают.

  \section*{shader.exe}

 Вернёмся к нашему шарику. Переименовываем файл в
    test3.rib и вносим небольшие изменения – убираем строку,
    определяющую материал (Surface “plastic”):
  
\code{code/rib_example/test3.rib}


 Различия в файле минимальны, но зато какое различие
    в результате:
  

 \gr{image045}
  

 В отличие от некоторых других рендереров, у prman
    есть понятие “шейдера по умолчанию” (этот шейдер хранится в файле с
    характерным названием defaultsurface.sl; у
    некоторых рендереров этот материал встроен в код рендерера –
    “захардкожен”; аналогом из мира Mayaявляется
    lambert1, назначаемый на все вновь создаваемые объекты). Даже если
    вы не назначили материал на свою геометрию, она всё равно
    отрендерится с использованием материала по умолчанию.
  

 Для
    продвинутых: в лучших FX-домах
    иногда в качестве шейдера по умолчанию устанавливают закраску
    модели красным цветом. Таким образом, при первом же тестовом
    рендеринге будет сразу видно, на какую из моделей вы забыли
    наложить материалы. Если геометрии в сцене действительно много – подобный
    трюк будет очень полезен.
  

 Ранее в своих экспериментах мы использовали шейдеры
    Plastic и Stone. Вообще говоря, согласно спецификации стандарта, со
    всеми Renderman-совместимыми рендерерами должны поставляться
    следующие шейдеры (вы помните, что источники света в Renderman тоже
    определяются шейдерами):
  
\begin{itemize}
	\item null
	\item constant
	\item      matte
	\item      metal
	\item      shinymetal
	\item      plastic
	\item      paintedplastic
	\item      ambientlight
	\item      distantlight
	\item      pointlight
	\item      spotlight
	\item      depthcue
	\item      fog
	\item      bumpy
	\item      background
\end{itemize}

 Так что поиграться уже есть с чем, но мы хотим
    сделать что-то своё. Представим себе, что вы где-то из Интернета
    скачали шейдер, который обучает вас использованию в Shading
    Language функции noise:
  
\code{code/rib_example/noisetest.sl}

 Для
    продвинутых: Нойз (“шум”) – это
    отличный и обычно первый из приходящих в голову способов добавить
    нерегулярности в “стерильную” компьютерную картинку.
  

 Сохраните его в текстовый файл под названием
    noisetest.sl рядом с файлом test3.rib. Немного правим test3.rib,
    добавляя в него ссылку на новый материал:
  
\code{code/rib_example/test3.rib}

 Запускаем на рендеринг – и получаем сообщение об ошибке:
  
\begin{lstlisting}[frame=single, framerule=0pt, framesep=10pt, xleftmargin=10pt, xrightmargin=10pt, basicstyle=\ttfamily \small, backgroundcolor=\color{light-gray}]
S01001 Cannot load shader "noisetest".  (WARNING)
\end{lstlisting}

 Что происходит? Мы же вроде бы всё правильно
    сделали, указали материал, положили шейдер рядом – что не так? Ан
    нет, не всё так просто. Перед непосредственным использованием
    шейдер необходимо привести в состояние, которое будет максимально
    удобным для рендерера – а попросту говоря,
    откомпилировать.
  

 Для
    любознательных: Конечно же, рендерер
    мог бы и сам понять, чего от него хотят, и откомпилировать шейдер
    внутри себя. Но это бы противоречило духу и одному из основных
    принципов Unix (“для каждой задачи – своя небольшая программа”),
    исторической справедливости и не дало бы  нам в руки могучее оружие
    пайпинга – а оно нам доступно и в этом случае.
  

 Для компиляции используется вторая программа нашего
    могучего триумвирата –
  

 shader.exe. Возможности и гибкость вызова этой
    программы абсолютно аналогичны возможностям и гибкости
    prman.exe  - если мы
    хотим, то можем соорудить цепочку программ, которые будут
    генерировать тексты шейдеров на языке SL. Но мы сделаем
    просто:
  
\begin{lstlisting}[frame=single, framerule=0pt, framesep=10pt, xleftmargin=10pt, xrightmargin=10pt, basicstyle=\ttfamily \small, backgroundcolor=\color{light-gray}]
shader noisetest.sl
\end{lstlisting}

 Программа отработала и в нашей директории рядом с
    файлов noisetest.sl появился noisetest.slo. Не стоит в него
    заглядывать – там интересно, но не настолько, чтобы мы отвлекались.
    Попробуем отрендерить картинку ещё раз и:
  

 \gr{image047}
  

 Для
    продвинутых: Далеко не очевидным,
    особенно для начинающих, является тот факт, что шейдеры вообще
    нужно компилировать. Даже если не затрагивать исторические корни
    языка шейдеров – а именно язык программирования C и модель работы с
    программами, написанными на нём  – достаточно будет одного лишь
    довода в пользу такого требования – скорость исполнения. Язык С
    достаточно сложен для того, чтобы его быстрая его интерпретация
    (выполнение программы без предварительной компиляции) была почти
    невозможной. Поэтому создатели рендереров делают всё возможное,
    чтобы ускорить процесс обработки шейдеров внутри своих программ,
    перенося эту задачу на компиляторы шейдеров. Prman внутри себя
    использует архитектуру выполнения SIMD, для которой компилятором
    shader генерируется специальный бинарный код. Другие рендереры
    поступают по-другому, например, RenderDotC превращает SL-код в
    минипрограмму на языке программирования C++, которая затем
    компилируется уже в файл DLL – и скорость выполнения шейдера таким
    хитрым образом увеличивается во много раз.
  

 Вернёмся к нашему RIBу. Вот строка из него, которая
    определяет новый подключённый материал:
  
\begin{lstlisting}[frame=single, framerule=0pt, framesep=10pt, xleftmargin=10pt, xrightmargin=10pt, basicstyle=\ttfamily \small, backgroundcolor=\color{light-gray}]
Surface  "noisetest"
\end{lstlisting}

 Это строку можно модифицировать, чтобы передать
    шейдеру параметры, например, вот так:

\begin{lstlisting}[frame=single, framerule=0pt, framesep=10pt, xleftmargin=10pt, xrightmargin=10pt, basicstyle=\ttfamily \small, backgroundcolor=\color{light-gray}]
Surface "noisetest" "freq" 100
\end{lstlisting}

 \gr{image049}
  

 Дальше – больше. Поскольку язык SL является
    потомком языка C, то нам доступны такие возможности C, как
    препроцессор и макросы. Проиллюстрируем это на простом примере.
    Немного изменим наш шейдер
  
\code{code/rib_example/noisetest2.sl}  

 и строку вызова его компиляции:
  
\begin{lstlisting}[frame=single, framerule=0pt, framesep=10pt, xleftmargin=10pt, xrightmargin=10pt, basicstyle=\ttfamily \small, backgroundcolor=\color{light-gray}]
shader -DFREQD=2 noisetest.sl
\end{lstlisting}

 Обратите внимание на то, что мы ввели в шейдер
    новую символьную константу (для удобства она обозначена большими
    буквами - FREQD), которую можем переопределять вне шейдера, в
    командной строке компилятора. Результат рендеринга будет,
    естественно, другим (мы вернули строку подключения шейдера в
    исходное состояние):
  

 \gr{image051}
  

 Учтите, что если бы мы вызвали компилятор без этого
    параметра, например, так
  
\begin{lstlisting}[frame=single, framerule=0pt, framesep=10pt, xleftmargin=10pt, xrightmargin=10pt, basicstyle=\ttfamily \small, backgroundcolor=\color{light-gray}]
shader noisetest.sl
\end{lstlisting}

 то получили бы ошибку – ведь константа останется
    неопределённой.
  

 Уже вкусно? Сейчас будет ещё вкуснее. Поскольку в
    наших руках почти что язык C, то мы можем вынести куски кода в
    подключаемые файлы (included files) или, наоборот, использовать уже
    готовые подключаемые файлы в своём шейдере. Один из самых
    интересных для изучения начинающими include-файлов можно найти в
    RManNotes, небольшом онлайновом пособии по Renderman Shading
    Language (его перевод на русский язык доступен на сайте
    Renderman.ru). Мы же воспользуемся стандартными возможностями из
    поставки Renderman Pro Server. Подключаем к нашему шейдеру
    библиотеку материалов:
  
	\code{code/rib_example/noisetest3.sl}  


 и вызываем компиляцию:
  
\begin{lstlisting}[frame=single, framerule=0pt, framesep=10pt, xleftmargin=10pt, xrightmargin=10pt, basicstyle=\ttfamily \small, backgroundcolor=\color{light-gray}]
shader -I%RMANTREE%\lib\shaders -DFREQD=2 noisetest.sl
\end{lstlisting}  

 и рендеринг:
  

 \gr{image053}
  

 Для
    продвинутых: -D и –I – это флаги
    препроцессора языка С. Подробнее о них можно почитать в справке или
    в любой книжке по С. Обратим наше внимание на то, что в
    традиционных компиляторах препроцессор и компилятор – это две
    разных программы, которые связываются пайпингом. В некоторых
    Renderman-совместимых рендерерах используется аналогичный подход,
    например, в BMRT и Aqsis. Есть такой препроцессор и у shader; его
    назвали slpp и спрятали в
    поддиректорию %{\it rmantree%\etc.}
  

 А как вы смотрите насчёт того, чтобы выстроить
    цепочку из нескольких программ-генераторов шейдеров? Или, например,
    написать внутри шейдера небольшую программу на Perl, затем
    обработать шейдер ещё одной небольшой программой на Perl – и
    получить таким образом динамически настраиваемый шейдер? А не
    изволите ли анимированных шейдеров? Возможностей – море, нужно
    только помнить, что вы ничего не можете сделать с самим шейдером {\it после} компиляции – только
    передать параметры из RIBа.

  \section*{txmake.exe}
  

 В этом случае всё совсем просто. Парадигма
    тотального ускорения рендеринга и разумного потребления памяти
    требует, чтобы всё, что поступало на вход рендерера, было
    максимально оптимизировано. Для RIBов таким вариантом является
    бинарный формат, для SL – откомпилированный код. Для текстур,
    которые вы можете использовать в своих шейдерах, тоже есть такой
    формат, и перевести текстуру в такой формат можно при помощи
    последней программы из нашей шустрой троицы –
    txmake.exe.
  

 Для
    продвинутых: Каким образом
    собственные форматы растровых текстур позволяют ускорить рендерер –
    и использовать более гигабайта текстур для одной модели? Всё очень
    просто:
  
     Текстура
      хранится не сплошным куском, а в виде тайлов – то есть побитой на
      квадратики. Соответственно, в память грузится не весь массив
      данных, а только тот квадратик, к которому идёт
      обращение.
     Внутри файла
      текстуры хранится не одно изображение, а несколько –
      последовательно уменьшающихся в размерах и детализации. Для
      объектов, нахожящихся на бОльшем расстоянии от камеры, используется
      текстура с меньшим разрешением – которая грузится быстрее и
      занимает меньше памяти.
     Прочие
      хитрости, вроде оптимизации данных под конкретный рендерер – так
      например prman при
      конвертации текстур может привести их размеры к величине, кратной
      степени двойки.
  
  

Таким
    образом, txmake при
    конвертации текстур создаёт многостраничный затайленый файл
    формата TIFF – благо
    формат позволяет. Просмотреть такой файл можно чем угодно,
    хоть ACDSee. Но если вы
    хотите воспользоваться текстурой размером в
    16000x16000 пикселей, а
    у вас на машине нет 2 гигабайт памяти – лучше использовать так
    называемый “старый” формат текстур Pixar, который
    можно получить при помощи того же txmake со
    специальными параметрами вызова – и который является ещё более
    оптимизированным вариантом.
  

 В Renderman Pro Server растровые изображения
    используются не только в качестве текстур, но и, например, как
    карты теней, если вы не пользуетесь рейтрейсингом. Обработка всех
    этих растров лежит на могучих плечах txmake.
  

 Одно небольшое отличие txmake от своих консольных
    собратьев состоит в том, что текстуры сами по себе текстовыми
    файлами не являются, то есть с ними наши фокусы с пайпингом
    работать не будут.
  

 Вот, собственно, и всё, что я хотел про неё
    рассказать. Ах, да, я забыл пример. Похулиганим
    немножко?
  

 По причинам, которые мы только что указали, пусть
    текстура будет квадратной, 512x512 пикселей. Напишем на ней...
    напишем на ней.... чего бы такого написать, чтобы никого не обидеть
    и чтобы литературный редактор пропустил? Вот, придумал:
  

 \gr{image055}
  

 Просто и со вкусом.
  

 Добавляем вызов текстуры в наш шейдер (по этому поводу переименованый в textured\_noise.sl):

\code{code/rib_example/textured_noise.sl}  


 Я несколько упростил наш шейдер по сравнению с
    последней итерацией (нам сейчас ни к чему все эти изыски с
    препроцессингом). Идея шейдера простая – показывается или надпись,
    или нойз (этакая модуляция нойза текстурой или, переходя на язык
    родных осин, перемножение двух изображений – очень распространённый
    приём).
  

 Для
    продвинутых: с педагогической точки
    зрения, код нашего шейдера следовало бы переписать несколько
    иначе:
  
\begin{lstlisting}[frame=single, framerule=0pt, framesep=10pt, xleftmargin=10pt, xrightmargin=10pt, basicstyle=\ttfamily \small, backgroundcolor=\color{light-gray}]
surface textured_noise ( float freq=100; )
{
  color tex = texture("texture.tif", s, t);
  float noi = noise(freq*s,freq*t);
  Ci  = tex * noi;
}
\end{lstlisting}
  

 В
    соответствии с рекомендуемым некоторыми источниками подходом,
    шейдеры необходимо разбивать на слои в соответствии с
    накладываемыми эффектами, и затем производить финальные операции
    уже над полученными ранее слоями. У этой модели есть свои
    преимущества и недостатки, сторонники и противники. Уточним лишь,
    что оптимизирующий компилятор шейдеров из комплекта {\it Renderman} {\it Pro} Server компилирует
    вышеприведённые шейдеры в почти одинаковый код.
  

Вносим также некоторые изменения в RIB:
  
\code{code/rib_example/test4.rib}

 Что мы сделали: исправили имя шейдера и немножко
    повернули наш шарик, чтобы надпись нормально читалась.
  

 Компилируем шейдер:
  
\begin{lstlisting}[frame=single, framerule=0pt, framesep=10pt, xleftmargin=10pt, xrightmargin=10pt, basicstyle=\ttfamily \small, backgroundcolor=\color{light-gray}]
shader textured_noise.sl
\end{lstlisting}
  

 Рендерим картинку:
  
\begin{lstlisting}[frame=single, framerule=0pt, framesep=10pt, xleftmargin=10pt, xrightmargin=10pt, basicstyle=\ttfamily \small, backgroundcolor=\color{light-gray}]
prman test4.rib
\end{lstlisting}

 и получаем огромное количество сообщений об
    ошибках, потому что рендерер не распознал нашу текстуру как
    правильный прооптимизированный формат. Строго говоря, какая-то
    картинка у нас получилась и так, но поскольку нас интересует
    академическая правильность процесса, то сделаем два дополнительных
    действия.
  

 Во-первых, подготовим картинку к
    рендерингу:
  

\begin{lstlisting}[frame=single, framerule=0pt, framesep=10pt, xleftmargin=10pt, xrightmargin=10pt, basicstyle=\ttfamily \small, backgroundcolor=\color{light-gray}]
txmake texture.tif texture.tx
\end{lstlisting}  

 То есть: берём нашу картинку и переводим её в новый
    формат, как водится, оптимизированный для рендеринга.
  

 Во-вторых, изменим шейдер:
  

\code{code/rib_example/textured_noise2.sl}  
  

 Для
    любознательных: Вас никто не
    заставляет называть свои файлы тем или иным образом. Вы можете
    хранить свой шейдеры в файлах с расширением *.shader; свою
    геометрию – в файлах с расширением *.rib, *.ri или *.renderman;
    инклуды в файлах *.h или *.inc. Просто стремитесь к тому, чтобы
    всегда придерживаться одного и того же логичного и удобного для вас
    именования файлов. Система именования, которую я использую в этой
    главе, считается стандартной и широко используется в литературе и
    документации.
  

 Для
    продвинутых: в процессе обработки
    текстур RAT включает в название файла параметры командной строки,
    которые использовались при вызове txmake. Например, вот так:
    metal024.tif.ppu.tx. Расшифровку скрытого в этом имени тайного
    значения оставим в качестве задания для самостоятельной работы для
    вас, продвинутые.
  

 Теперь вроде бы всё правильно – компилируем,
    запускаем рендерер и получаем в результате:

 \gr{image057}
\section*{Могучая кучка}
Собственно, из трёх программ, prman, shader и
   txmake, состоит вычислительное ядро Renderman Pro Server. Как мы
   уже говорили раньше, подавляющее большинство Renderman-совместимых
   рендереров придерживается такой же архитектуры, просто называя
   файлы другими именами. Приведём лишь некоторые из них:
 
%\begin{tabular}{ l c c c r }
\begin{tabular}{ | c | c | c | c | c | }
\hline
\textbf{Продукт} & \textbf{Рендерер} & \textbf{Компилятор} & \textbf{Готовый} & \textbf{Обработчик}\\
  &   & \textbf{шейдеров} & \textbf{шейдер} & \textbf{текстур}\\
\hline
Prman & prman & shader & *.slo & txmake \\
\hline
AIR & air & shaded & *.slb & mktex \\
\hline
RenderDotC & renderdc & shaderdc & *.dll & texdc \\
\hline
3delight & renderdl & shaderdl & *.sdl & tdlmake \\
\hline
Aqsis & aqsis & aqsl & *.slx & teqser \\
\hline
\end{tabular}   
 

Ссылку на постоянно обновляющийся список рендереров
   вы найдёте в конце главы, но уверяю вас – у участников списка всё
   обстоит аналогичным образом.
 

Для
   любознательных: вы наверняка знаете,
   что в комплекте поставки Maya есть
   программа под названием {\it render.}exe, которая
   позволяет рендерить майские сцены из командной строки?Я не буду
   проводить аналогии – хотя бы потому, что остальных программ из
   нашей троицы у Maya нет в силу
   особенностей архитектуры.
 

А у нас настало время систематизировать наши знания
   о внутренностях Renderman-совместимого рендерера. Проще всего будет
   это сделать в виде схемы, показывающей путь данных внутри
   системы:
 

\gr{image059}
 

Как видите, мы умудрились ужать несколько десятков
   страниц убористого технического текста в одну простую
   диаграмму.
 

Каким же образом в этот процесс встраивается MTOR?
   Добавим в схему недостающие элементы (а заодно уберём из неё
   временные файлы *.TX и *.SLO):
 

\gr{image061}
 

Как видно из обновлённой диаграммы, майская сцена
   обрабатывается МТОRом и превращается в файлы геометрии *.RIB;
   материалы, разработанные в SLIM, конвертируются в шейдеры *.SL. С
   текстурами чуть сложнее, но ненамного – вызовы их обработчиков
   также настраиваются в SLIM.
 

{\it Замечание:} Для большей простоты
   понимания мы не указали в этой схеме Alfred.
 

А как же та самая пресловутая совместимость между
   рендерерами в рамках стандарта, о которой так громко говорили
   большевики всю эту главу? Нет ничего проще: берём страничку
   соответствия файлов между различными рендерерами и делаем в схеме
   соответствующую замену. На практике, полностью взаимозаменимых
   рендереров не бывает (как бы их авторы не старались), но
   спецификация стандарта даёт нам уверенность в том, что особых
   дополнительных усилий такая замена не потребует.
 

А теперь представьте себе, что между любыми
   элементами этой новой диаграммы можно встроить другие программы и
   скрипты на различных языках программирования, которые будут
   преобразовывать ваши данные.
 

Вы считаете, что и этого мало? Ну тогда давайте
   вернёмся к нашей схеме и описанию нашей триады и дополним её
   некоторыми немаловажными деталями, которые мы не затронули в первом
   заходе.
 
    Утилита txmake принимает на входе файлы формата
     TIFF. На самом деле, список поддерживаемых форматов гораздо длинее,
     и включает в себя: Maya IFF, SGI, SUN, TGA, Alias, GIF, JPEG, LBM,
     BMP, ICO, форматы систем X11 (например, майские иконки в файлах
     XPM), PhotoCD, 24битные raw bitmaps и некоторые другие, а начиная с
     версии 12 – становящийся де-факто стандартом индустрии
     OpenEXR.

    В поставке Renderman Pro Server есть утилита
     sho.exe, которая позволяет конвертировать между всеми этими
     форматами.

     Кстати говоря, в поставке Maya есть утилита imgcvt.exe, которая
     делает почти то же самое – конвертирует между форматами.
     Преимущество sho в том, что она может ещё и показывать файлы на
     экране.

     Признайтесь честно – когда вы в последний раз заглядывали внутрь
     Maya/bin?

    Язык шейдеров SL можно расширить за счёт своих
     собственных процедур. Сделать это можно, написав свою собственную
     DLL (в нашем случае называемую на юниксоидный лад: DSO). Поверьте
     мне – это не так сложно, как кажется, и для этого совсем не
     обязательно знать язык C или C++ - я с успехом писал DSO на
     Паскале. Классические примеры расширения SL – frankenrender (вызов
     одного рендерера из шейдера, исполняемого в другом рендерере;
     именно так в стародавние времена в prman встраивали raytracing –
     просто вызывали BMRT) и Vtexture (использование векторных файлов в
     качестве текстур).

    Вы уже знаете, что формат RIB можно генерировать
     своими собственными программами; это очевидно, поскольку формат это
     простой и в основной своей ипостаси – текстовый. О чём мы в нашей
     главе не говорили – так это о том, что этот формат также можно
     расширять при помощи динамических генераторов RIB-кода, вызовы
     которых встраиваются непосредственно в сами RIBы. Такие генераторы
     можно писать как в виде DLL/DSO, так и непосредственно как
     исполняемые файлы, или даже скрипты на Perl.

     {\it Комментарий в сторону:} Я уже говорил, что Perl -
     язык истинных криптоманьяков. Хорошим подтверждением этой гипотезе
     является тот факт, что базовым скриптовым языком в ILM и Google
     избран Python.

    Начиная с версии 11.5, prman позволяет встраивать
     ваши собственные фильтры внутрь рендерера для более глубокого
     процессинга RIB. Опять таки, эти фильтры представляют из себя
     DLL/DSO с определённой схемой вызовов. Начиная с версии 12.0, эти
     фильтры описаны в документации.

    prman может рендерить во множество файловых
     форматов: TIFF, Maya IFF, Softimage, TGA, SGI, CIN, PIC, Alias. Но
     этот список можно легко увеличить при помощи своих собственных
     плагинов, в этом случае называемых display drivers. Экспериментируя
     с prman в нашей главе, мы рендерили сразу в окно просмотра; такая
     возможность реализуется при помощи плагина d\_windows.dll. Пробуя на
     зуб MTOR, мы считали картинку непосредственно в окно программы it –
     делается это при помощи драйвера d\_socket. Разрабатывая ShaderMan
     (о нём поговорим чуть попозже, но уже совсем скоро), я поставил
     перед собой задачу рендерить напрямую в окно своей программы – и
     решил эту задачу, написав специальный display driver. Стандартная
     для prman 12 возможность считать в файлы формата OpenEXR может быть
     реализована и в более старых версиях рендерера – при помощи таких
     же драйверов (их можно скачать с сайта OpenEXR или найти
     через Google). Хотите
     считать в PNG, JPEG2000 или свой собственный über-формат? Просто
     напишите драйвер.

Почему множество людей, FX-домов и студий во всём
   мире используют Renderman Pro Server? Сведём всё предшествующее
   описание, все эти десятки страниц, картинок и схем в один
   список:

\begin{itemize} 
    \item Стандарт Renderman
    \item Скорость работы
    \item Соответствие тем нормам, которые мы определили как необходимые для продакшн рендерера
    \item Удобство и простота в освоении
    \item Расширяемость
\end{itemize}

Добавим в эту гремучую смесь такие свойства
   рендерера, как:

\begin{itemize} 
    \item быстрый motion blur – почти не влияющий на скорость рендеринга (сравните с другими)
    \item настоящий высококачественный displacement
    \item SL – который сам стоит целого списка
\end{itemize}
и вы получите тот динамит, который взорвал индустрию спецэффектов. Технологию с великим прошлым, динамичным настоящим и полным оптимизма
будущим.
\chapter*{Альтернативы?}
 

Почему же множество людей, FX-домов и студий во
   всём мире НЕ используют
   Renderman Pro Server? Тому есть множество причин.
 

Начнём с очевидной - цены.
 

На этом шаге можно было бы сразу и закончить,
   потому что по цене комплект из Renderman Pro Server, RAT + годичная
   подписка на услуги службы поддержки почти догнал (чуть было не
   написал – автомобиль) Maya Unlimited – надеюсь, вы оценили. Один
   этот факт ставит жирный крест на продукте для маленьких студий,
   фрилансеров и студентов, которые хотят поэкспериментировать с
   рендерером дома – они выбирают либо совместимые альтернативы, либо
   совершенно другие продукты.
 

Но я хотел заострить ваше внимание на другом
   аспекте проблемы. Да, prman – очень универсальный рендерер. Но,
   согласно поговорке, когда вы берёте в руки молоток, то все предметы
   вокруг вас начинают казаться гвоздями.
 

И поэтому мы сделаем небольшой шаг в сторону от
   нашего магистрального направления (“Maya+Renderman=счастье”) и
   попытаемся объять необъятное – рассмотреть альтернативные prman’у
   рендереры, которые можно использовать совместно с Maya.
 

Вообще говоря, говорить о том, что какой-то
   рендерер не совместим с Maya, особенно после того, как вы
   познакомились с главой, посвященной Mel – глупо. Возможности Mel в
   области вывода данных из Maya настолько велики, что фактически любой рендерер,
   запускаемый из командной строки, можно с теми или иными затратами
   прикрутить к Майе. И это будет работать – а если вас перестанет
   устраивать скорость работы скрипта - то вы просто перепишете ваш
   скрипт в виде плагина (или попросите кого-то переписать ваш скрипт
   в виде плагина). А это означает, что Maya-совместимым является
   почти любой из существующих на рынке рендереров, запускаемых из
   командной строки. Но, как говорил Козьма Прутков, нельзя объять
   необъятное – у нас здесь не Большая Советская Энциклопедия, в конце
   концов. И поэтому мы сделаем небольшую выборку из огромного списка
   и расскажем вам о нескольких внешних рендерерах.

 \section*{Какие рендереры?}
 

Как-то вечером я задался вопросом – какие вообще
   рендереры существуют и почему они существуют вообще? Вопрос
   настолько же риторический, насколько и философский – по аналогии,
   можно спросить, почему так много моделей лопат или молотков есть в
   магазине.
 

Во-первых, очень часто большая (а иногда – и не
   очень большая) студия самостоятельно изготавливает для себя рабочий
   инструментарий – начиная с рендерера и заканчивая системами
   моделинга, анимации, цветокоррекции и композитинга. В таком случае
   сотрудники этой студии получают максимальный контроль над
   результатами своего труда.
 

Во-вторых, очень часто существующие решения не
   справляются с поставленными для них задачами, неважно, из-за своих
   особенностей или из-за особенностей таких задач. Для таких задач
   (типичные примеры – аниме, волосы, жидкость и пламя) достаточно
   часто пишутся специальные рендереры – которые умеют считать только
   один вид геометрии или один спецэффект – но делают его хорошо и
   очень быстро.
 

Ну и на закуску остаются – исследовательские и
   студенческие проекты, источник вдохновения и исходного кода, поток
   новых идей и инноваций.
 

Чем же закончился тот вечер, когда я попытался
   объять необъятное? Простой схемой, на которой я перечислил все
   известные мне (пусть только по названию) рендереры и провёл между
   некоторыми из них связи – будь то родственные, технологические или
   какие-то ещё. Картинка перед вами. 

\gr{image063}
 

Тёмно-серым цветом на этой схеме обозначены некие
   ключевые продукты, от которых я отталкивался в выстраивании системы
   отсчёта – краеугольные камни. Три таких камня очевидны – это
   Renderman. Mental Ray и видеокарты (OpenGL/DirectX и прочее). Но
   есть и четвёртый камень в этой схеме, и этот четвертый элемент -
   Gelato.


 \chapter*{Gelato}
  

 Появления этого продукта ожидали многие. Ещё не
    было достоверно известно, что он существует, но все понимали, что
    внутри Nvidia что-то варится. Не могло не вариться – как не может
    критическая масса внутри атомной бомбы не взорваться.
  

 Но – будем рассказывать по порядку.
  

 В июле 2000го года три человека – Ларри Гритц, Мэтт
    Фарр и Крэг Колб (Larry Gritz, Matt Pharr, Craig Kolb) объединили
    свои усилия, чтобы создать новый Renderman-совместимый рендерер.
    Ларри внёс в копилку знаний свой рендерер BMRT и опыт работы в
    Pixar над prman’ом и интеграцией prman и BMRT (помните, как я,
    рассказывая о возможностях расширения SL, кратко упомянул
    Frankenrender?). Мэтт положил в пул свою идею (и написанную на её
    основе диссертацию) о кардинальном ускорении процесса raytracing’а
    – и опыт работы в Pixar над prman. Крэг – опыт, накопленный в ходе
    разработки рендерера Rayshade. Объединившись под вывеской Exluna
    (“На Луну!”) (для любознательных – латынь язык интересный и не
    всегда однозначный; переводом названия Exluna могут также служить
    «с луны» или «не на луне») и выбрав в качестве логотипа мультяшный
    космический кораблик, разработчики сделали почти невозможное – в
    августе 2001го года представили на суд публике новый рендерер –
    Entropy. Качественный, быстрый, Renderman-совместимый рейтрейсер
    превосходил prman по набору поддерживаемых алгоритмов рендеринга и
    стоил всего полторы тысячи долларов (не случайно считается, что
    именно после появления Entropy в Pixar серьёзно озаботились
    встраиванием ray tracer’а в Renderman Pro Server – хотя скорее
    всего, это не так и работы шли параллельно) – а значит, был гораздо
    дешевле.
  

 Entropy был хитом и раскупался, как пирожки. Почва
    под ногами prman зашаталась, и ответный ход не заставил себя ждать
    – к сожалению, он был не совсем приятным. 5го марта 2002го года
    Pixar подала на Exluna в суд, обвиняя комнанию в нарушении
    патентов, а 16го мая – во втором иске обвинила трио основателей в
    плагиате (ведь два из них ранее работали в Pixar и видели исходный
    код prman). Тщетно адвокаты молодой компании пытались бороться –
    куда стартапу из 10 человек на втором году жизни было тягаться
    против гиганта Pixar (с Диснеем за спиной). И несмотря на то, что
    причин для судебного разбирательства не было и патенты не
    нарушались – лучшим выходом для Exluna оказалось прекратить продажи
    своего продукта, продаться Nvidia с потрохами и дать победителям
    урегулировать споры между собой.
  

 К этому моменту сложилась, можно сказать,
    интересная ситуация – в Nvidia одновременно оказались несколько
    разработчиков рендереров: Джакопо Панталеони (Jacopo Pantaleoni),
    автор LightFlow; троица из Exluna; Дэниэл Векслер (Daniel Wexler),
    создатель рендерера, который PDI/Dreamworks использовали в фильмах
    Antz и Shrek. По совокупности одновременно находящихся в компании
    именитых рендерописателей Nvidia вышла на первое место в
    индустрии  – и это
    могло означать только одно – у Entropy готовится
    преемник.
  

 Таким преемником оказался новый рендерер Gelato,
    анонсированный 19 апреля 2004го года.
  

 Основной особенностью Gelato, из-за которой его так
    любит пресса и так не любят некоторые разработчики и TD, является
    то, что это первый продакшн рендерер, который использует
    возможности вашей видеокарты для рендеринга – и более того, не
    работает, если у вас не установлена определённая видеокарта –
    конечно же, с чипом от Nvidia. Чтобы понять, насколько серьёзное
    преимущество оказывается в руках у тех, кто берёт на вооружение
    новый рендерер, отвлечёмся немного и поговорим о процессорах,
    Законе Мура и современных реалиях.
  

 Как гласит закон Мура, число транзисторов на
    современном компьютерном микропроцессоре удваивается каждые 2 года.
    Этот закон был сформулирован в 1965ом году, работает до сих пор и
    по оценкам исследователей – будет работать ещё с десяток лет.
    Казалось бы, всё отлично - до недавних пор формулировка закона
    означала, что каждые два года мы получаем в свои руки как минимум
    вдвое больше производительности – но, как показывает опыт, в
    последнее время рост производительности центральных
    микропроцессоров (CPU) практически прекратился, несмотря на
    продолжающийся рост сложности их конструкции, числа транзисторов и
    тактовой частоты. Производители микропроцессоров (Intel, AMD, Sun,
    IBM и другие) стараются изо всех сил, придумывая и воплощая в жизнь
    новые технологии и идеи – но предложить что-то кардинально новое, в
    очередной ускоряющее вычисления в 2 раза, увы – не в состоянии. В
    ближайшее время нас ждёт нашествие многоядерных решений, в которых
    за счёт фактического объединения 2х процессоров в один будет
    сделана попытка обмануть закон Мура и опять удвоить
    производительность – но проблема налицо – сделать персональный
    компьютер с центральным процессором, в 2 раза быстрее сегодняшнего,
    сложно и очень затратно (хотя, как показывает практика -
    можно).
  

 С другой стороны, в мире процессоров, применяющихся
    в видеоускорителях (GPU) всё складывается гораздо интереснее –
    требования по поддержке всё новых и новых стандартов (DirectX,
    OpenGL, шейдеры), по увеличению качества рендеринга во всё новых и
    новых приложениях и особенно – в играх – привели к тому, что сугубо
    специализированные ускорители отрисовки треугольников превратились
    в универсальные процессоры данных, позволяющие программировать себя
    и обладающие рядом уникальных свойств. При этом для GPU закон Мура
    не выполняется – потому что скорость у них не удваивается каждые 2
    года, а чуть ли не удесятеряется – ведь эти процессорами не тяготит
    груз совместимости и необходимость запускать операционную систему.
    Было бы глупо не попытаться использовать такой мощный
    дополнительный универсальный вычислитель для ускорения работы
    основного процессора – и такие попытки делались и делаются, и всё
    больше программ умеют использовать “дармовые” мощности. Но первыми,
    кто смог оптимизировать для GPU готовый к производству рендерер,
    стали сотрудники Nvidia.
  

 Итак, для запуска Gelato вам понадобится
    современный мощный компьютер под управлением Windows XP или Linux –
    и современная-же мощная профессиональная видеокарта на чипсете от
    Nvidia – в частности, Quadro FX. Прежде чем мы попробуем новый
    рендерер в бою (вы тоже можете это сделать – тестовая версия Gelato
    доступна для скачивания на сайте; единственное ограничение этой
    версии – водяной знак на результатах рендеринга), рассмотрим кратко
    основные черты этого продукта.
  
     Gelato – полноценный production renderer
      (удовлетворяющий нашему списку требований), взявший за основу
      Entropy и BMRT и добавивший в них множество новых
      свойств
     Gelato – современный рендерер. В нём нет поддержки
      многих устаревших режимов и видов геометрии, которые не нужны при
      выполнении реальных задач. Что там говорить – такой геометрический
      примитив, как сфера, появился в рендерере уже в самом конце
      бета-теста по настоятельным просьбам тестеров – привыкшим к
      использованию шариков в своих тестовых сценах. С другой стороны,
      Gelato поддерживает все современные технологии – ray tracing,
      ambient occlusion, caustics и многие другие.
     Gelato не является Renderman-совместимым рендерером
      (Nvidia решила не играть в кошки-мышки с Pixar),  но взял от былой совместимости
      всё самое лучшее. Например, в этом рендерере есть свой собственный
      язык шейдеров GSL – на 90 процентов совместимый с языком SL и
      переводимый из него при помощи несложного конвертера (неопытный
      пользователь может перепутать шейдеры, настолько они похожи).
      Возвращаясь к предыдущему пункту – над языком GSL основательно
      поработали, улучшив его новыми конструкциями и убрав устаревшие – и
      теперь в роли догоняющей оказалась уже Pixar.
  
  

 Если посмотреть правде не куда нибудь, а в
    конкретное место – перечислять инновации Gelato вот таким сухим
    списком скучновато не только для вас, но и для меня самого. Давайте
    лучше попробуем сделать что-нибудь в Maya и отрендерить в Gelato –
    а заодно и поговорим о достоинствах этого рендерера.
  

 Итак, в поставке Gelato (для тех, кто не знает –
    это такой сорт мягкого мороженого) идёт (продолжая вкусную тему)
    Mango  – плагин к Maya,
    который (в отличие от MTOR) встраивается в неё почти без проблем и
    позволяет использовать всю мощь обоих продуктов.
  

 Создадим новый проект, новую сцену, импортируем
    нашу овечку (файл sheep.ma на диске). Овечка у нас полигональная,
    если вы ещё не заметили – так что быстренько переводим её в
    Subdivision Surfaces: Modify => Convert => Polygon to Subdiv.
    Рендерим стандартным движком Maya – движок немного притормаживает –
    наконец получаем картинку:
  

 \gr{image065}
  

 Все наши предыдущие картинки с овцой были на чёрном
    фоне, который уже немного надоел – поэтому мы вставили в сцену
    большой шарик, который и дал нам серенький фон. Надоест и этот фон
    – уберём и его и будем опять смотреть на чёрный задник, но пока
    нормально и так. Да, на чём мы там остановились? Ах, да,
    Gelato.
  

 Если рендерер у вас установился правильно, то всё,
    что вам нужно сделать – это включить плагин обычным способом и
    вызвать окно Render Globals:
  

 \gr{image067}
  

 Переключаем рендерер, запускаем просчёт опять,
    слышим, как заработал вентилятор на видеокарте, ненадолго появилась
    и сразу же исчезла командная строка – и получаем прямо в окошке
    Render View результат:
  

 \gr{image069}
  

 Не нужно иметь зрение горного орла, чтобы увидеть,
    что рендеринг прошёл гораздо быстрее и в результате мы получили
    гораздо более качественную картинку – не меняя при этом установок
    по умолчанию в обоих рендерерах. При этом Gelato подхватил все
    установки Maya и воспроизвёл их своими средствами.
  

 Что произошло за кадром, пока мы слушали песню
    вентилятора нашей Nvidia GeForce QuadroFX 3000 (кстати, пока я это
    писал – вентилятор уже выключился)? Заглянем в директорию c:\\temp –
    в ней появились 3 новых файла:
  
gelato.pyg
gelato.bat
gelato\_perspShape.pyg
  

 Внутри этих файлов находится самое интересное –
    геометрия, которую Mango экспортировал из Maya в собственный формат
    Gelato. Правильнее говоря – Mango эскпортировал в формат PYG – один
    из форматов, поддерживаемых Gelato. Казалось бы, разница между
    двумя фразами минимальна, но она показывает глубокое философское
    различие между Gelato и другими рендерерами; одно из тех различий,
    на которых мы остановимся, и которое может вас удивить.
  

 Различие это состоит в том, что в Gelato нет
    формата сцены по умолчанию. Формат шейдера есть (GSL), а формата
    сцены – нет. Вместо этого пользователю представляется хорошо
    документированный программный интерфейс API и возможность написать
    в соответствии с этим интерфейсом плагин к рендереру,
    который  и будет
    заниматься вопросами загрузки геометрии. Понятное дело, что взять и
    просто так выпустить на рынок рендерер, в который по умолчанию
    нельзя загрузить геометрию вообще – это глупость, поэтому для
    Gelato придумали свой собственный формат PYG, представляющий из
    себя – программу на языке Python. В общем, смотрите сами, перед
    вами кусочек файла с овечкой:
  

World()
PushTransform()
SetTransform(((0.736097, 0, -0.676876, 0),
                (-0.130758, 0.981164, -0.142198, 0),
                (-0.664126, -0.193179, -0.722232, 0),
                (17.6053, 12.845, 21.3661, 1)))
Light("mayaDefaultLight", "distantlight", "point to", (0.5, -0.5, 1))
PopTransform()
PushAttributes()
Attribute("string name", "sheep:sheep")
AppendTransform ( ((1, 0, 0, 0),
                    (0, 1, 0, 0),
                    (0, 0, 1, 0),
                    (0, 0, 0, 1)) )
  

 Очень похоже на RIB (что неудивительно) – но это не
    обычный текстовый файл, а минипрограмма. Значит, мы можем взять – и
    сделать из неё настоящую программу, например, написать цикл или
    загрузить данные из другого файла или прочитать ввод пользователя
    из консоли или даже загрузить данные из Интернета – в нашем
    распоряжении вся мощь языка программирования Python!
  

 Энтузиасты Renderman немного скривились – в их
    руках такой мощной игрушки нет. А мы тем временем нанесём им ещё
    один удар – загрузив с сайта Nvidia плагин к Gelato, который
    позволяет использовать в качестве файлов геометрии – RIB-файлы.
    Действительно – а почему бы и нет, если есть открытый API?
    Попробуем эту возможность в бою – временно откладываем в сторону
    Maya и вызываем командную строку:
  

gelato test2.rib
  

 и немедленно получаем на экране окно с результатами
    рендеринга:
  

 \gr{image071}
  

 Как видно, Gelato без особых проблем справился с
    RIBом – благо шейдер “Plastic”, который мы использовали с нашей
    тестовой сцене, входит в поставку – равно как и все остальные
    шейдеры из стандарта Renderman - сказывается Entropy-шное
    прошлое.
  

 Это самое прошлое не раз будет напоминать о себе в
    наших изысканиях. Раз уж мы начали копаться в командной строке –
    посмотрим, например, из чего же состоит наш новый друг. Ба! Налицо
    уже набившая нам оскомину триада – рендерер, компилятор шейдеров и
    конвертор текстур – на этот раз они называются gelato.exe, gslc.exe
    и maketx.exe.
  

 Внутри директории gelato/bin ещё много всего
    интересного, например, утилита topyg, которая позволяет перевести
    любой из поддерживаемых установленных вами плагинами форматов
    геометрии в PYG – с её помощью можно сравнить одну и ту же сцену в
    формате PYG и RIB и увидеть явные аналогии. А мы возвращаемся к
    нашим баранам – я хотел сказать, к нашей овечке - в
    Майку.
  

 Немножко покрутим камеру, чтобы видеть одну только
    голову нашей модели – экспериментировать мы будем именно с головой.
    Выделяем голову, создаём в Hypershade новый материал – Create =>
    Materials => Lambert. Назначаем этот материал на голову овцы и
    начинаем с ним играться. Я не долго думал над примером и, по
    аналогии с предыдущими экспериментами, подключил к ламбертовскому
    diffuse – Noise:
  

 \gr{image073}
  

 Снова запускаем рендеринг – а рендерером у нас всё
    так же стоит Gelato – и с удивлением смотрим на
    результат:
  

 \gr{image075}
  

 Gelato подхватил настройки материалов из
    Hypershade, сконвертировал их в свои – и правильно отобразил новый
    материал. Оказывается, Mango знает о существовании Hypershade, в
    его поставке есть набор шейдеров для многих Hypershade-овских нод,
    из которых и конструируются готовые – шейдеры для Gelato? Нет, не
    шейдеры. Мы в очередной раз встретились с инновационным решением –
    Gelato позволяет накладывать несколько шейдеров на один объект,
    объединяя их в группы и более того – позволяя строить из таких
    шейдеров более сложные конструкции, передавая параметры и
    результаты работы из одного шейдера в другой. Давайте ещё раз
    посмотрим внутрь файла c:\\temp\\gelato.pyg:
  
\begin{lstlisting}[frame=single, framerule=0pt, framesep=10pt, xleftmargin=10pt, xrightmargin=10pt]
ShaderGroupBegin()
Shader( "surface", "maya|_place2dTexture", "place2dTexture1" )
Shader
    ( "surface", "maya_noise", "noise1", "float amplitude", 0.6529,
    "float depthMax", 4, "float frequency", 11.57, "float
    frequencyRatio", 2.1158, "float noiseType", 0, "float ratio",
    0.90082, "float threshold", 0.22316 )
ConnectShaders
    ("place2dTexture1", "outUV", "noise1", "uvCoord")
ConnectShaders
    ("place2dTexture1", "outUvFilterSize", "noise1",
    "uvFilterSize")
Shader( "surface", "maya_lambert", "lambert2", "color _color", (0.20932,
    0.166365, 0.9091) )
ConnectShaders("noise1", "outAlpha", "lambert2", "diffuse")
ShaderGroupEnd()
\end{lstlisting}
  

 Манго определил, что мы используем три ноды –
    place2dTexture, noise и lambert. Для этих нод в поставке Mango есть
    соответствующие шейдеры (загляните внутрь
    \$GELATOHOME/mango/shaders, чтобы узнать, какие ноды поддерживаются
    на данный момент). Далее создаётся группа шейдеров, в которой при
    помощи процедуры ConnectShaders воссоздаётся то построение, которое
    мы с вами только что делали визуально в HyperShade.
  

 Кто-то ещё считает, что SLIM – самое
    удобное для создания шейдеров средство? Пусть бросает читать на
    этом месте, а мы тем временем продолжим перестройку сознания и
    галоп по возможностям Gelato. Вернёмся в директорию c:\\temp и
    покажем, как сделать быстрый рендерер ещё быстрее. Удаляем
    gelato.bat, а бывшее его содержимое копируем в командную строку и
    запускаем:
  

gelato –iv gelato\_perspShape.pyg gelato.pyg
  

 Попутно открылась ещё одна возможность – несколько
    PYG-ов, заданных в качестве параметров, будут выполнены в той
    последовательности, в которой вы их задали – удобно для разделения
    обших настроек сцены и собственно геометрии. Но мы хотели ускорить
    просчёт сцены. Запускаем рендерер ещё раз, с чуть другими
    параметрами:
  

gelato
    -iv -preview 0.1 gelato\_perspShape.pyg gelato.pyg
  

 Результат появился практически мгновенно – мы
    используем специальный режим предварительного просмотра в несколько
    ухудшенном качестве. Можно ли отрендерить ещё быстрее? Да,
    можно:
  

gelato  -iv  -shade defaultsurface -preview 0.1 gelato\_perspShape.pyg gelato.pyg
  

 Мы отключили все шейдеры и вместо них использовали
    defaultsurface. Результат выглядит ужасно,  но получается очень быстро даже
    для самых сложных сцен и показывает гибкость настроек
    предварительного просмотра:
  

 \gr{image077}
  

 Мы могли бы и дальше расписывать особенности
    Gelato, упрощающие жизнь трёхмерщика, приводя всё новые и новые
    примеры – но вместо этого просто обозначим один факт: этот рендерер
    является более расширяемым, чем prman, и вся его философия – это
    философия расширяемости. Хотите собственные операторы в GSL,
    аналогичные таковым в Renderman SL? Есть. Хотите рендерить в
    собственные форматы картинок? Есть открытый простой API, с помощью
    которого можно не только выводить в ваш формат данных (как в случае
    с display drivers у Renderman), так и импортировать эти данные – и
    изначально поддерживаются TIFF, Maya IFF, JPEG, PNG, PPM, TGA, HDR,
    DDS (сжатые текстуры для использования в DirectX) и, конечно же,
    OpenEXR. Хотите использовать собственный формат сцены? Пишете
    плагин или переводите в RIB или PYG. Хотите встроить рендерер в
    pipeline своей студии? У вас в руках мощнейший язык
    программирования Python, поддерживаемый этим рендерером. Хотите
    рендерить по слоям? Обсчитывать геометрию в shading grid, а потом
    использовать Gelato исключительно для решейдинга? Планка,
    установленная Renderman Pro Server,  поднята на новую высоту новым
    продуктом Nvidia. И индустрия это чувствует – интерес к рендереру
    очень высок и, несмотря на молодость и новизну, поддерживающие его
    конвертеры, утилиты и плагины появляются, как грибы после
    дождя.
  

 Должен признаться. Вернее, я должен быть признаться
    ещё в начале нашего рассказа про Gelato. Мне нравится этот
    рендерер. Он мне действительно очень нравится, почти так же (а по
    моему даже больше, прим. Редактора), как и prman. У меня есть
    видеокарта Quadro FX и я бета-тестер Nvidia – соответственно,
    имеющий доступ к внутренней информации (которая разглашению не
    подлежит) и бета-релизам (про которые я вам умудрился ни слова не
    рассказать, хотя очень хотелось). Мне тяжко осознавать, что я
    предаю свою искреннюю любовь к Renderman-у, но иногда, по вечерам,
    я тихонько закрываю дверь своей комнаты и вместо того, чтобы
    экспериментировать с prman’ом – исследую Гелато. Поэтому – ещё раз
    – мне нужно было признаться с самого начала в том, что я несколько
    пристрастный в данном случае человек – но если вы хотите
    беспристрастности – читайте отладочные дампы или судебные
    стенограммы.
  

 Что же у нас в сухом остатке? На рынке появился
    новый продукт, который в полной мере отвечает обозначенным нами
    критериям production-качества, прост для новичков, очень удобен и
    гибок в использовании в руках опытных пользователей. Но для того,
    чтобы воспользоваться всеми этими преимуществами, вам понадобится
    современный компьютер, оснашёный мощной профессиональной
    видеокартой последнего поколения от Nvidia – то есть денег этой
    компании вы заплатите два раза, сначала за рендерер, а потом за
    видеокарту (и третий раз – за саппорт). Стоит ли этих денег как
    минимум 2х-кратное (полученное мной в ходе антинаучных
    экспериментов, на одинаковых простых тестовых сценах, на одном и
    том же компьютере, gelato 1.0R3 в сравнении с prman 11.5.3)
    ускорение просчёта сцены? Решать вам.
  

 Для
    продвинутых: говорят, аборигены в
    лесах Амазонии знают, как запустить Gelato на игровых видеокартах,
    равно как на ноутбуках  без Quadro FX. Это сакральное знание передаётся ними на закрытых
    форумах в виде патчей к драйверам Nvidia, которые позволяют обычной
    видеокарте эмулировать хай-эндовую – благо элементная база у них
    одинаковая. Даже и не знаю, что сказать по этому
    поводу.
 \chapter*{Jot}
  

 1го февраля 2003го года я оконфузился.
  

 Чувствовал я себя в этом процессе очень глупо, не
    очень уютно и вообще как-то некомфортно. Эту историю я ещё никому
    не рассказывал, потому что немного стыдно до сих пор. По прошествии
    нескольких лет я потихоньку начинаю понимать, что всё было не так
    уж и плохо и зря я так серьёзно отношусь ко всяким мелким
    происшествиям, тем более, что стыдиться-то собственно и нечего -
    но, как говорится в бородатом анекдоте – ложки-то нашлись, а осадок
    остался.
  

 И, чтобы выветрить осадок, для излития души я
    решил, что выбирать стоит только между мегафоном на набитой народом
    площади и книгой по Maya. Сами видите, что я выбрал.
  

 1го февраля 2003го года школа Реалтайм проводила в
    Москве очередной семинар по компьютерной графике ViAGra (Visual Art
    and Graphics). На этом семинаре у меня было небольшое выступление
    на тему “Нефотореалистичный рендеринг”. Следует отметить, что я уже
    давно интересуюсь этой темой и экспериментирую с различными NPR
    техниками, используя в своей работе Renderman-совместимые рендереры
    (а в последнее время и Gelato) как источники информации для
    собственных программ и алгоритмов. И для своего выступления на
    Виагре я подготовил не только доклад, но и небольшой показ своих
    собственных достижений, которыми (втайне) очень
    гордился.
  

 А вот в конце доклада я в качестве обзора вставил в
    презентацию несколько картинок, показывающих различные
    нефотореалистичные техники рендеринга, имитирующие анимационную
    закраску, псевдоживопись, чертёжные линии, гуашь с акварелью и
    прочие интересности.
  

 Одна из этих картинок была результатом рендеринга в
    Jot.
  

 В те времена этот университетский проект ещё не
    назывался Jot, не был выложен в публичный доступ и не обрёл свою
    микроармию поклонников. Но уже тогда по рукам интересующихся
    бродила видеозапись, демонстрирующая все возможности Jot’а в
    действии.
  

 Честно вам скажу – я не помню, откуда в моей
    демонстрации взялся этот файл с видео. Возможно, я сам его привёз.
    Вполне может быть, что кто-то подошёл и поставил свой диск – я
    просто не помню.
  

 Единственное, что я вообще помню сейчас про эти 3
    минуты, пока на экране шёл показ прототипа Jot – это охи и ахи со
    всех сторон, голос девушки из середины зала, которая застонала
    “дайте мне это, я хочу делать это!” и возглас “Ну вот, это же
    гораздо круче того, что вы показывали”.
  

 Вот эта последняя фраза меня и доконала. Надеюсь,
    никто этого не заметил – но я сильно расстроился. Немного смешно
    вспоминать это сейчас, честное слово – но это так.
  

 Прошло время. Контур и заливку сейчас рендерят все,
    кому не лень – он есть в стандартной поставке Maya, он есть во
    многих рендерерах (в том числе и Renderman-совместимых) в качестве
    дополнительной опции. Студия Disney на одной из последних выставок
    Siggraph показала свой внутренний рендерер Inka, рассказала
    принципы его работы и внутреннего устройства – и с тех пор только
    очень ленивый студент не написал идеальный картунный движок.
    Проблема исчезла сама собой, вместе с ней ушла эйфория увлечения
    нерешёнными задачами и новыми технологиями, наступила пора готовых
    решений. Но даже сейчас, когда я запускаю Jot, внутри меня
    начинается какое-то противоречивое шевеление. Я продолжаю
    восторгаться его возможностями, ужасаться корявости его интерфейса,
    и я вспоминаю тот февральский день, когда весь зал ахал и охал,
    пусть и не от моих картинок.
  

 Я отлично понимаю, отчего все они так реагировали.
    Университетский проект, результат долгих лет исследований студентов
    и профессоров, Jot – классический нефотореалистичный рендерер “как
    в мультфильме” с возможностью работы из командной строки, простым
    форматом файла и изюминкой – интерактивным редактором стилей
    закраски. Интерфейс этого интерактивного редактора ужасен, но стиль
    его работы и – самое главное – результаты его работы завораживают.
    Именно поэтому в качестве второго претендента на обзор в этой главе
    я избрал Jot – хороший пример того, как пристыковать к Майке
    посторонний специализированный рендерер.
  

 Скачайте и установите Jot с сайта http://jot.cs.princeton.edu/.
    Настройка продукта – процесс достаточно сумбурный, но если вы
    внимательно прочитаете документацию и всякие там README – доступный
    и быстрый, поэтому здесь не задокументированный. Нам же интересно –
    как вывести наши данные, чтобы их понял Jot.
  

 Смотрим в исходный файл одного из
    примеров:
  

 vertices           { {{4.984928608 -4.984928608 -7.771561172e-016} ….
  

 faces    { {{160 1 161 }{78 2 79 }{44 4 45 }{26 7 27 }{17 8 19 }…
  

 То есть сцена в формате JOT описывается в виде
    точек (вершин), из которых затем собираются треугольники. Ничего
    военного, очень похоже на формат файла Wavefront OBJ – и похоже,
    что кроме этой геометрии, нас в файле ничего не интересует. Засучив
    рукава, быстренько пишем небольшой скрипт на Mel для экспорта
    данных в нужный формат (текст скрипта любезно предоставлен Егором
    Чащиным и Сергеем Цыпцыным):
  
\begin{lstlisting}[frame=single, framerule=0pt, framesep=10pt, xleftmargin=10pt, xrightmargin=10pt]
global proc export_jot(string $_filename)
{
       string $list[] = `ls -sl`;
       string $item;
       for($item in $list)
       {
             select -hi -r ($item);
             string $hi[] = `ls -sl`;
             string $it;
             print ("EXPORTING:\n");
             string $filename = $_filename;
             int $FP = fopen($filename, "w");
             for($it in $hi)
             {
                   string $type = `nodeType ($it)`;
                   if($type == "mesh")
                   {                                       
                        fprint $FP "#jot\n\nTEXBODY \t{\n";
                        fprint $FP ("\tname \t"+$item+"."+""+$it+"(celia) \n");
                        // code for WTM
                        fprint $FP "\txform \t{{1 0 0 0 }{0 1 0 0 }{0 0 1 0 }{0 0 0 1}}\n";
                          fprint $FP "\tmesh_data      { \n\t\tLMESH     {\n";
                        fprint $FP "\t\t\tvertices   { {";
                        int $nv[] = `polyEvaluate -v $it`;
                        int $nf[] = `polyEvaluate -f $it`;
                        int $ii,$jj;
                        string $varr="";
                        for($ii=0;$ii<$nv[0];$ii++)
                        {
                              float $vt[] = `xform -q -ws -t ($it+".vtx["+$ii+"]")`;
                              $varr+="{"+$vt[0]+ " "+$vt[1]+ " "+$vt[2]+ "}";
                        }
                        fprint $FP ($varr);
                        fprint $FP "}} \n";                     
                        fprint $FP "\t\t\tfaces \t{ {";
                        string $farr="";
                        for($ii=0;$ii<$nf[0];$ii++)
                        {
                              string $face[] = `polyInfo -fv ($it+".f["+$ii+"]")`;
                              string $buf[];
                              int $nn=tokenize($face[0], " ", $buf);
                              $farr+="{";
                              for($jj=2; $jj<$nn-1; $jj++) $farr+= $buf[$jj]+" ";
                              $farr+="}";
                        }
                        fprint $FP ($farr);   
                        fprint $FP "}} \n";
                        fprint $FP "\t\t\t} } \n\t} \nCREATE \t{ ";
                        fprint $FP ($item+"."+$it+"(celia)\n\t} ");                    
                    }
             }
             fclose($FP);     
       }
 }
\end{lstlisting}
  

 Как видно из текста скрипта, чтобы не
    заморачиваться с триангуляцией сцены, мы будем выводить
    исключительно полигональные объекты – помните об этом. Тэээкс, где
    там наша многострадальная овечка? Создаём новую сцену, импортируем
    овцу. Чтобы не очень много ждать в процессе экспорта, оставляем от
    бедного животного одну голову, переводим в сабдивы, а затем сразу
    же переводим в обратно в полигоны – Modify => Convert =>
    Subdiv to polygon. Объявляем в Maya наш скрипт, вызываем его на
    выполнение командой в Script Editor:
  

 export\_jot("sheep.jot");
  

 и получаем файл sheep.jot. Всё, что нам осталось
    сделать – запустить jot и насладиться результатом:
  

 \gr{image079}
  

 Как говорится в только что придуманой мной
    поговорке – овечья голова это ещё не вся овца. Что не даёт нам
    экспортировать всю модель? Излишняя медлительность скрипта, который
    вынужден обходить три с половиной тысячи вертексов. Такова природа
    вещей – писать скрипты на Mel’е легко и весело, а выполнять тяжело
    и грустно,  поскольку
    они интерпретируются, и выполняются  заведомо медленнее
    откомпилированного оптимизированного кода на C++ - и потому для
    экспорта больших и сложных моделей годятся с большой натяжкой. Дело
    мастера боится – закатываем рукава и пишем плагин, который будет
    делать всё то же самое, что и наш замечательный скрипт, но в
    десятки раз быстрее. Если с программированием на C++ и
    использованием Maya API у вас туговато – используем план Б и
    загружаем с диска готовый плагин (в качестве приглашённого
    программиста выступил Вячеслав Богданов). Для совсем ленивых я
    записал на диск полностью готовую к экспорту сцену – sheep\_jot.ma.
    Нам же осталось только привести готовую картинку:
  

 \gr{image081}
  

 Для продвинутых и
    неленивых: Пара советов.
    Обращайте внимание на то, куда смотрят нормали вашей поверхности. И
    помните о том, что при помощи приведённого выше
    Mel-скрипта  мы можем
    экспортировать в Jot только один объект – поэтому не забывайте
    объединять все полигоны в сцене в единый объект перед
    экспортом.

 
\chapter*{GRUNT}
  

 Пришла пора от овечек перейти к другим животным на
    ту же букву О – к оркам. Мы попробовали на вкус два рендерера,
    каждый со своими премудростями, особенностями и сильными сторонами.
    Gelato – универсальный продукт, хорошо приспособленный для
    большинства ставящихся в реальном продакшне задач; Jot –
    университетский проект; он умеет рендерить только картунные линии,
    но делает это хорошо и включает в себя интерактивный редактор таких
    линий и их стилей. Вы можете попробовать применить оба продукта в
    своей повседневной работе – а теперь я бы хотел рассказать о
    рендерерах, применить которые в своей работе вы врядли сможете. Это
    внутренние разработки иностранных студий, которые используются ими
    в производстве кино и рекламы. Почему эти студии занимаются такими
    разработками? Что их не устраивает в prman’е, Mental Ray’е и прочих
    общедоступных продуктах? Попробуем ответить на этот
    вопрос.
  

 Надеюсь, все присутствующие смотрели хотя бы один
    из фильмов трилогии “Властелин Колец”? Тогда вы, как и я, были
    впечатлены огромными батальными сценами, в которых участвовали
    тысячи, если не десятки тысяч сгенерированных компьютером бойцов –
    орков, людей, эльфов и прочих сказочных персонажей.
  

 Новозеландская студия Weta Digilal, автор
    спецэффектов в трилогии, использует в качестве основы своего
    пайплайна две “коробочных” программы – Maya и Renderman Pro Server.
    Но как и всякая другая серьёзная студия, Вета старается
    минимизировать свои расходы и максимально ускорить производственный
    процесс – именно поэтому в батальных сценах использовался
    специализированный рендерер под названием GRUNT, созданный
    сотрудником студии Джоном Эллитом (Jon Allitt).
  

 Этот рендерер (а его название расшифровывается как
    General Renderer of Unlimited Numbers of Things) имеет одно
    неоспоримое преимущество перед prman’ом – он использует A-buffer
    модель рендеринга. В ситуации с батальной сценой с участием десятка
    тысяч персонажей, Renderman Pro Server пытается оптимизировать
    расход памяти и бьёт картинку на квадратные кусочки, каждый из
    которых будет просчитываться отдельно. На тот случай, если уже
    обсчитанная геометрия потребуется в следующем кусочке, информация о
    ней сохраняется в кэш – иначе накладные расходы на
    выгрузку/загрузку/пересчёт одного и того же серьёзно замедлят
    процесс.
  

 Для
    продвинутых: по-научному
    кусочки  эти называются
    “бакеты”. Prman в процессе
    рендеринга использует бакеты размером по умолчанию 16x16 пикселей и
    затем обрабатывает сцену не всю целиком, а по таким вот кусочкам –
    и по окончании работы с очередным куском занимаемая им память
    освобождается. Так вот проблема состоит в том, что на огромных
    сценах, в особенности с полупрозрачными и мелкими объектами, с
    большим displacement, с применением motion blur и Depth of field
    (ну то есть в типичном кино или телевизионном проекте) – в бакете
    накапливаются очень большие объёмы информации. Добавьте к этом кэш
    обработанной геометрии, о котором мы говорили выше.  Конечно, изворотливые
    пользователи придумали множество различных решений и обходных
    манёвров – но суть проблемы от этого не меняется.
  

 Для нашей батальной сцены  даже на машинах с несколькими
    гигабайтами оперативной памяти prman вылетает из-за нехватки оной –
    просто потому, что кэш становится слишком большим. GRUNT этого
    недостатка лишён – он поддерживает в памяти только A-buffer (то
    есть, Z-buffer, набор нормалей, координаты точки на поверхности,
    цвет и некоторую другую информацию для каждого пикселя картинки) и
    обходится без кэша моделей, просто подгружая каждого нового орка,
    рендеря его, обновляя A-buffer и затем выгружая орка. Чтобы
    упростить понимание системы – представьте себе композитинг по
    Z-буферу – если у вас есть что-то, что находится дальше
    Z-координаты – вы его не рендерите и вас слабо интересует, что
    именно изображено на соответствующем пикселе – если это не стекло,
    конечно. Но поскольку орки и эльфы в трилогии Толкиена не ходят в
    стеклянных доспехах – считайте, что вам очень сильно
    повезло.
  

 Любопытно, что GRUNT начинался в 1992 году как
    универсальный рендерер (и анимационная система) для операционной
    системы OS/2 и активно использовался при производстве мультфильмов
    и телевизионной рекламы. В 1992м году 32 мегабайта памяти на
    персональном компьютере были огромным размером – и выбор технологии
    A-буфера дал возможность даже на таком маленьком объёме памяти
    рендерить очень большие сцены. Позднее, когда Джон присоединился к
    Weta Digital, ему пришло в голову, что массовые батальные эпизоды –
    отличное поле деятельности для его прежде универсального
    рендерера.
  

 Итак, основное преимущество специального рендерера
    – это используемая в его ядре технология, которая позволяет
    рендерить удивительно сложные сцены. Что ещё даёт для студии
    обладание таким инструментом?
  

 Во-первых, это возможность гибкой настройки для
    определённых задач. Ведь если у вас в руках есть исходный код
    продукта и более того – вы его автор – то никакой prman не даст вам
    подобной гибкости. Обратите внимание – я говорю именно о гибкости,
    а не мощности или скорости работы.
  

 Во-вторых – возможность серьёзной оптимизации под
    свои задачи. От продукта никто не требует расширяемой модели
    освещения и поддержки Renderman-шейдеров – и шейдеры для него
    пишутся на C++ и оптимизируются по самое немогу. Программой часто
    пользуются на этапе настройки сцены – и поэтому встроен специальный
    механизм, который позволяет сбрасывать промежуточные результаты
    рендеринга и сразу же их просматривать. Pipeline компании заточен
    под композитинг при помощи Shake – и GRUNT умеет считать в слои.
    Почти все кадры рассчитываются с использованием “фермы” – и
    рендерер умеет делить сложную сцену на несколько простых для такого
    просчёта.
  

 Основным поставщиком информации и моделей для GRUNT
    является другой специальный продукт – Massive, который отвечает за
    правдоподобное поведение компьютерных персонажей на поле боя.
    Massive посредством простых текстовых файлов передаёт в рендерер
    как обычную (для рендерера) информацию (например, направление
    движения каждого объекта, его размер и положение), так и необычные
    вещи – например, в какую одежду нужно одеть того или иного
    персонажа. Если бы для всего этого использовались RIBы, то объём
    занимаемого дискового пространства был бы огромным. В данном
    случае, все необходимые алгоритмы и схемы встроены непосредственно
    в рендерер, который анализирует входные данные и на лету собирает
    необходимую модель, которую затем рендерит. Ту же самую текстовую
    начинку умеет генерировать посредством простых Mel-скриптов и Maya
    – и на этом этапе разница между prman и GRUNT для TD, работающих в
    Вете, стирается.
  

 Как показывает практика, число компаний,
    занимающихся производством компьютерной графики, анимации и
    рекламы, и при этом имеющих возможность позволить себе пользоваться
    своими собственными разработками – исчезающе мало. Сказывается как
    себестоимость таких разработок, которая ставит под вопрос
    возможность их окупить, так и высокий порог сложности такой задачи.
    Появление на рынке Maya с внутренним языком программирования Mel
    сильно уменьшила этот порог – практически невозможно найти
    пользователя Maya, который бы не писал собственных скриптов. С
    собственными рендерерами, несмотря на все преимущества, ситуация
    куда сложнее – они остаются вотчиной крупных студий или отчаянных
    чудаков, как в нашем следующем примере.
  
\chapter*{Spore}
  

 Рассказ об этом, не побоюсь громкого слова,
    революционном рендерере будет одной из самых сложных для меня
    частей в главе. Всё дело в том, что готовя материал к публикации, я
    попытался пообщаться напрямую с людьми, непосредственно связанными
    с разработкой тех или иных продуктов. В большинстве случаев мне это
    удалось – не скрою, многим разработчикам очень приятно и лестно,
    когда результат их труда освещается в книге (пусть и
    русскоязычной).
  

 Многим, но не Ричарду Бейли.
  

 Мы общались с доктором Бейли в течение нескольких
    недель, обменявшись не одним десятком писем по e-mail. И за всё это
    время я не узнал ничего нового про Spore. То есть вообще ничего.
    Можете мне поверить, я действительно старался. Поэтому всё, что вы
    прочитаете здесь о творении Ричарда, взято исключительно из
    открытых источников; впрочем, даже эта горстка информации
    впечатляет.
  

 Что же делает Sporе настолько особенным, чтобы мы
    посвятили ему главу в своей книге?
  

 Spore – это узкоспециализированный рендерер,
    предназначенный для одного типа геометрии – партиклов. Но зато
    оптимизирован он под этот тип настолько, что позволяет рендерить в
    одной сцене недостижимое для других систем число частиц – более
    миллиарда. Такое количество взаимодействующих партиклов переводит
    эффекты, которые можно достичь при помощи Spore, в кардинально
    новую плоскость – сам автор рендерера не боится говорить, что
    перешёл на уровень фотонных эффектов и теперь занимается не
    рендерингом, а световой скульптурой.
  

 Результаты работы маленькой студии Ричарда Бейли
    под названием Image Savant выглядят немного психоделически –
    впрочем, это не удивительно, потому что все свои инструменты – а
    студия работает исключительно на своих собственных программах и
    рендерерах – Ричард рассматривает как пробы в живописи и
    любительские арт-проекты. Впрочем, для небольшой студии и
    психоделически выглядящего “любительского” портфолио у Image Savant
    неплохой список заказчиков и фильмов – назовём лишь поверхность
    планеты Солярис в одноимённом американском ремейке и внутренности
    Земли в фильме The Core.
  

 Поскольку Spore является единственным источником
    существования небольшой студии, все детали внутреннего устройства
    этого рендерера доктор Бейли хранит в строгой тайне. Известно лишь,
    что, имея полный доступ к исходному коду – и являясь автором этого
    кода – Ричард фактически настраивает свой продукт для каждого
    проекта и, возможно, для каждого кадра. В ходе работы над фильмом
    Solaris Image Savant выдала на гора 45 минут высококачественной
    16ти-битной анимации в разрешении 2k за несколько (2-3) месяцев,
    что говорит об экстраординарной скорости и гибкости рендерера.
    Скорее всего, Spore не предусматривает шейдеры в том виде, к
    которому мы привыкли после Maya, Renderman и Mental Ray, но, как и
    GRUNT, позволяет использовать некие заранее предусмотренные модели
    закраски. Опять же, полный доступ к коду продукта позволяет
    получать промежуточные результаты на любой стадии просчёта кадра –
    мало какой рендерер похвалится такой гибкостью.
  

 Так о чём же мы могли говорить с Ричардом в ходе
    нашего онлайнового интервью, спросите вы, если я ничего не узнал о
    его рендерере? О многом. Например, о том, что единственная
    программа, которую используют в своей работе сотрудники Image
    Savant и которую они не написали сами – это Shake, причём версии 1,
    без визуального интерфейса, исключительно через скрипты - и
    переходить на более свежие версии эти мазохисты отказываются
    категорически. О том, что в современном мире кинопродакшна вполне
    можно взять и открыть студию из 3х человек, которые займут свою
    собственную нишу и будут обеспечены работой на многие годы вперёд –
    не имея ни копейки денег и ни одного заказа на старте. О том, что
    хорошие, “правильные” программы пишутся чаще всего не
    программистами. О том, каким будет следующий рендерер
    студии  и что он
    позволит сделать (втайне надеюсь, что у них опять всё получится). О
    том, наконец, насколько сложен в изучении русский язык для
    англоязычных – надеюсь, наша книга поможет Ричарду в решении и этой
    задачи.
  

 Вероятнее всего, мы никогда не увидим Spore
    продающимся в виде коробочного продукта и те из вас, кто не
    поступит на работу в студию Image Savant, никогда не узнают, есть
    ли у этого продукта интерфейс к Maya. Впрочем, это не так уж и
    важно. Не знаю, как вам, а мне просто достаточно знать, что
    возможность рендерить миллиард партиклов на кадр – есть. Значит,
    если немного постараться, то можно будет обработать и превратить в
    потрясающе красивый кинокадр и 10 миллиардов частиц – а кто знает,
    может, из такой разработки родится и ваша студия?
\chapter*{Renderman – альтернативы RAT}
  

 Возвращаясь к теме Renderman + Maya, я хотел бы
    сказать несколько (тысяч) слов о альтернативах RAT. Ни для кого не
    секрет (и я попытался передать это ощущение в своей главе), что
    связка Renderman Pro Server + RAT + Maya близка к идеалу; ни для
    кого также не секрет, к сожалению, что идеальной она не является. О
    некоторых проблемах данной связки мы уже говорили; кратко
    просуммируем сказанное:
  
     высокая цена комплекта
     ухудшающееся качество кода в последних версиях
      RAT
     неудовлетворённость пользователей интерфейсом и
      возможностями
     и ещё раз - высокая цена – увы, для многих этот
      фактор критичен.
  
  

 И поэтому неудивительно, что существуют несколько
    альтернативных RAT’у решений, позволяющих решать тот же круг задач,
    но не отягощённых присущими продукту компании Пиксар проблемами. В
    преддверии самой интересной части этой главы, посвященной
    оптимизации рендеринга и хитростям, применяемым в реальном
    продакше, мы не будем тратить ваше время на подробные описания и
    уроки, взамен предъявив вам список основных решений, их
    положительные и отрицательные черты и наше заключение по каждому из
    продуктов. Логичным представляется разделить наш краткий обзор на
    несколько основных направлений:
  
     Альтернативные prman’у рендереры
     Плагины и средства для экспорта
      геометрии
     Программы для работы с шейдерами
  
  

 И опять я вас обманул. У нас будет ещё и пункт
    номер ноль.

  \section*{Альтернативные интерфейсы для MTOR}
  

 Именно так – альтернативные интерфейсы. Для MTORа.
    Казалось бы, в чём проблема?
  

 А проблема очень простая – существующий визуальный
    интерфейс RenderMan Globals, написанный в закудыкины времена на
    языке программирования Tcl, выглядит чужеродным объектом в теле
    Maya. Более того, для продвинутых пользователей продукта это
    торжество программирования со множеством окон и табов не кажется
    оптимальным. Именно поэтому существуют несколько различных скриптов
    на Mel, которые предлагают альтернативный подход к настройкам MTOR
    – более удобный, скоростной и “родной” – ведь интерфейс самой Maya
    тоже написан на Mel. Как образец творческого подхода к вопросу,
    можно указать скрипт Юрия Мешалкина под названием mtorRender (вы
    можете найти его на диске):
  

 \gr{image083}
  

 Человеку, только начинающему разбираться с
    Renderman Pro Server и MTOR, будет полезнее пока оставаться в
    знакомом окружении родного интерфейса RenderMan Globals (как
    минимум для того, чтобы не было проблем с прохождением туториалов),
    но когда его рамки начнут вам мешать – обратитесь к альтернативным
    решениям.

  \section*{Альтернативные Renderman-совместимые рендереры}
  

 В списке Renderman-совместимых рендереров \footnote{http://www.dotcsw.com/links.html},
    любезно поддерживаемом одним из разработчиков такого рендерера
    (Dot C Software, продукт
    под названием Render Dot C), на апрель
    2005го года находилось 23 продукта, из которых доступны для покупки
    или (в случае open source) бесплатного скачивания – 13 штук. Все
    они обладают теми или иными достоинствами и недостатками, например,
    бесплатный Pixie – использует OpenGL и, соответственно, ресурсы
    видеокарты для ускорения просчёта сцены; недорогой AIR обгоняет
    всех конкурентов по набору возможностей и по этому параметру
    единственный приблизился к prman. Флагманом
    Renderman-совместимо-рендеростроения с открытым исходным кодом
    является Aqsis – достаточно медленный, но старательно пытающийся
    стать реальным противовесом коммерческим приложениям. Являясь
    Renderman-совместимыми (а чтобы называться таким, рендерер должен
    строго придерживаться основных положений стандарта), все эти
    продукты могут быть использованы в качестве замены Renderman Pro
    Server, в особенности если вы не использовали какие-то
    специфические для prman особенности SL и RIB – более того, вы
    можете их использовать вместе с MTOR и SLIM, специальным образом
    настроив последние.
  

 Но меня не оставляет мысль, что единственным
    реальным соперником prman’а среди Renderman-совместимых является
    де-юре не-Renderman-совместимый Gelato.Я был бы счастлив не
    разделять этот скептицизм с вами – но таковы факты.
 
 \section*{Экспорт из Maya в RIB – Mel}
  

 При всей своей простоте формат файлов RIB в том
    виде, в котором он описан в спецификации стандарта – достаточно
    объёмен и включает в себя много различных возможностей и удобств. С
    другой стороны, Maya поддерживает
    много различных видов геометрии, материалов и эффектов. Поэтому
    задача вывода сцены из Maya в RIB посредством написания скриптов на
    Mel является исключительно академическим упражнением – на простых
    сценах рендерер не может показать себя во всей своей красе; на
    сложных сценах слабым звеном становится сам интерпретируемый скрипт
    (выросший к тому моменту до нереальных размеров). Тем не менее, для
    простых экспериментов или для случаев, когда пасуют готовые
    продукты - этот подход хорош.
  
\section*{Экспорт из Maya в RIB - не Mel}
  

 Меня всегда удивляло, сколько всего интересного
    поставляется в комплекте с Maya. Про универсальный конвертер
    графических файлов мы уже говорили; а вот знаете ли вы, что в
    стандартной поставке Майки можно найти – плагин для экспорта
    геометрии в RIB? Удивительно, но факт – ribExport.mll поставляется
    со всеми версиями Maya, начиная с первой, и позволяет делать
    простой экспорт геометрии – и более того, плагин этот поставляется
    с полным исходным кодом. Ни для каких серьёзных задач данное
    решение не предназначено; качество и читаемость RIB-ов, которые оно
    производит, оставляет желать лучшего. Тем не менее, такая
    функциональность в рамках Maya существует – и при очень большом
    желании вы можете взять исходники и попробовать их модифицировать
    для своих нужд. Но мой вам совет – при первой же возможности
    воспользуйтесь такими плагинами, как MayaMan или Liquid.
  
\section*{MayaMan}
  

 Одними из первых, кого не удовлетворили качество и
    ценовые запросы MTOR, стали антиподы – австралийцы из студии Animal
    Logic. Уже одно происхождение продукта по всей видимости даёт ему
    хорошую фору по отношению к конкурентам – находясь на другой
    стороне нашего шарика, антиподы всё время висят вниз головой, что
    помимо постоянного притока крови к голове даёт немалую
    раскрепощённость и отсутствие дурных привычек, присущих голивудской
    индустрии.
  

 Однако, шутки в сторону. Бесплатная оценочная
    версия доступна на сайте www.animallogic.com – вы можете
    попросить временную лицензию, и скорее всего, вам её дадут – мне,
    как видите, лицензию дали.
  

 Для того, чтобы попробовать MayaMan в деле,
    откройте свою любимую сцену или создайте новую и подключите плагин.
    В главном меню Maya появился новый пункт:
  

 \gr{image085}
  

 Как обычно в нашей главе, импортируем бедную
    овечку, переводим её в сабдивы и начинаем искать, как же нам её
    отрендерить: MayaMan => MayaMan Globals (имя команды в меню
    ничего не напоминает?):
  

 \gr{image087}
  

 Первое же отличие, которое бросается в глаза – это
    поддержка различных Renderman-совместимых рендереров. MayaMan
    изначально разработан с тем расчётом, чтобы быть совместимым с
    максимально возможным числом рендереров, предоставляя им одинаковые
    возможности – и при этом максимально скрывая разницу между ними.
    Данная стратегия дала свои результаты – учитывая различные версии
    одинаковых продуктов, плагин поддерживает 53 различных рендерера,
    от самых старых и уже не встречающихся в природе – до самых
    продвинутых и ещё не вышедших из стадии предварительного
    тестирования.
  

 Для
    продвинутых: я получил эти цифры,
    изучив содержимое директории /mayaman/renderers. Файлы, находящиеся
    в этой папке, содержат все необходимые настройки для всех
    поддерживаемых рендереров, и представляют из себя весьма
    увлекательное чтиво.
  

 Нас пока что удовлетворяют настройки по умолчанию,
    поэтому просто вызываем MayaMan => Preview (я предварительно
    добавил в сцену большой светлый шарик, чтобы убрать чёрный
    фон):
  

 \gr{image089}
  

 Какой конфуз! В отличие от MTOR, MayaMan на данный
    момент не поддерживает subdivision surfaces от Maya Unlimited.
    Однако мы всё равно можем отрендерить полигоны как сабдивы – и для
    этого воспользуемся кастомными атрибутами, которые MayaMan
    позволяет добавить к любому объекту в сцене.
  

 Для
    продвинутых: если не считать конфуз
    с майскими SDS, официально MayaMan поддерживает и экспортирует в
    RIB любую геометрию, кроме PaintFX – впрочем, PaintFX в RIB не
    экспортирует ни один из известных плагинов. Этот вопрос – “Как
    отрендерить Paint Effects в Renderman” в своё время был самым
    задаваемым в русских форумах, посвященных Renderman; нам пришлось
    даже вынести его в FAQ. Впрочем, и у этой проблемы есть возможное
    решение – вывести данные из PaintFX в виде кривых не составляет
    особого труда, загвоздка лишь в применяемых кистях и их
    отрисовке.
  

    Удаляем из сцены нашу овечку и импортируем её заново. На этот раз
    мы не будем переводить её в сабдивижны штатными майскими
    средствами; вместо этого, выделяем ту геометрию, которую нужно
    отобразить в виде SDS и вызываем пункт меню MayaMan => Add Model
    Attributes. Затем в окне редактирования атрибутов в разделе
    SubDivision Surface выставляем нужную галочку и:
  

 \gr{image091}
  

 получаем наше животное в целости и сохранности. Но
    это не так интересно – на протяжении нашей главы мы уже с дюжину
    раз различными методами рендерили бедного барашка; попробуем теперь
    узнать, какие дополнительные преимущества перед MTOR имеет MayaMan.
    Для этого назначим на нашу модель любой материал из HyperShade и
    посмотрим, что произойдёт (в данном случае мы назначили Blinn с
    присоединённым к каналу цвета brownian):
  

 \gr{image093}
  

 Оказывается, как и Mango, MayaMan поддерживает
    материалы HyperShade и конвертирует их в шейдеры RenderMan; в
    отличие от Mango, MayaMan действительно создаёт новые шейдеры –
    впрочем, делает это очень хитро, давайте посмотрим получившийся у
    нас шейдер (его можно найти в поддиректории нашего проекта
    /mayaman/sheep1/shaders/; я убрал заголовок шейдера для
    краткости):
  
\begin{lstlisting}[frame=single, framerule=0pt, framesep=10pt, xleftmargin=10pt, xrightmargin=10pt]
{
  

   PROFILE("begin");
  

#pragma
    nolint
  

   SURFACE_TEMPS;
  

   APPLY_SLICE_PLANE else {
  

     PROFILE("SETUP 0 BLINN");
  

     BLINN_SETUP(v_0_);
  

     if(only_do_opacity == 0) {
  

       PROFILE("SETUP 1
    BROWNIAN");
  

       BROWNIAN_SETUP(v_1_);
  

       PROFILE("SETUP 2
    PLACE3DTEXTURE");
  

       PLACE3DTEXTURE_SETUP(v_2_);
  

       PROFILE("SIMULATE 2
    PLACE3DTEXTURE");
  

       PLACE3DTEXTURE_SIM(v_2_);
  

       v_1_placementMatrix  =
    v_2_worldInverseMatrix[0];
  

       PROFILE("SIMULATE 1
    BROWNIAN");
  

       BROWNIAN_SIM(v_1_);
  

       v_0_color            = color(v_1_outColor);
  

     }
  

     PROFILE("SIMULATE 0 BLINN");
  

     BLINN_SIM(v_0_);
  

     Ci = v_0_outColor;
  

     Oi = Os;
  

   }
  

   DEEP_SHADOW_OPACITY_HANDLER
  

   PROFILE("end");
  

   SURFACE_DEBUG_HOOK
  

 }
\end{lstlisting}
  

 Не очень похоже на обычный шейдер, написанный на
    Renderman SL, правда? Тем не менее, это обычый шейдер – вся
    хитрость состоит в том, что в этом шейдере очень широко
    используются такие возможности языка SL, как макросы и подстановка
    значений. Если мы пропустим этот файл через препроцессор cpp, то
    сможем увидеть истинное лицо нашего шейдера – впрочем, я не буду
    сюда его вставлять из-за достаточно большого размера – книжка-то у
    нас не резиновая.
  

 Для
    продвинутых: те же, кто не
    поленится заглянуть в получившиеся 40 килобайт (почти 1200 строк)
    кода на SL, узнают много нового о том, как на Renderman SL
    сымитировать ноды из Maya HyperShade, да и вообще – про внутреннее
    устройство Maya и про рендеринг вообще.
  

 Итак, MayaMan конвертирует материалы из Maya в
    Renderman SL – и тем самым серьёзно упрощает работу для тех, кто
    только начинает пробовать на зуб новый рендерер. Кроме того, вы
    можете использовать свои собственные шейдеры в качестве материалов
    – и даже больше, вы можете встраивать материалы, написанные на
    языке SL, в виде специальных нод HyperShade – и эти ноды будут
    использованы при рендеринге через MayaMan.
  

 Нам же осталось указать, что всё возможности и
    настройки MayaMan доступны как через Mel (посредством ноды под
    весёлым названием MayaManNugget), так и из командной строки – при
    помощи программы mayaman\_batch\_m6.exe (или mayaman\_batch\_m65.exe, в
    зависимости от используемой вами версии Maya).
  

 Для
    продвинутых: несколько простых
    скриптов на Mel, которые упростят вашу работу с MayaMan. Во-первых,
    узнаем, какие вообще параметры есть у ноды MayaManNugget – и,
    соответственно, у всего плагина:
  
\begin{lstlisting}[frame=single, framerule=0pt, framesep=10pt, xleftmargin=10pt, xrightmargin=10pt]
string \$MayaManOptions[] = `listAttr MayaManNugget`;
for ($eachParameter in $MayaManOptions) print(eachParameter  + "\n");
\end{lstlisting}  

 Во-вторых,
    небольшой скрипт, который можно вытянуть на полку, чтобы быстро
    переключаться в режим предварительного просмотра – и ухудшеного
    качества (о значении всех этих устанавливаемых параметров мы будем
    говорить уже очень скоро):
  

{\it   } setAttr
    "MayaManNugget.ShadingRate" 16;
  

   setAttr "MayaManNugget.PixelSamplesX"
    1;
  

   setAttr "MayaManNugget.PixelSamplesY"
    1;
  

   setAttr "MayaManNugget.PixelFilterX"
    2;
  

{\it    setAttr}  "{\it MayaManNugget}{\it .}{\it PixelFilterY}"
    2;
  

 И в-третьих,
    антипод скрипта “во-вторых”:
  

{\it   } setAttr
    "MayaManNugget.ShadingRate" 1;
  

   setAttr "MayaManNugget.PixelSamplesX"
    3;
  

   setAttr "MayaManNugget.PixelSamplesY"
    3;
  

   setAttr "MayaManNugget.PixelFilterX"
    3;
  

{\it    setAttr}  "{\it MayaManNugget}{\it .}{\it PixelFilterY}"
    3;
  

 Суммируя сказанное: MayaMan специально разработан в
    расчёте на использование с несколькими Renderman-совместимыми
    рендерерами, у него более удобный (на мой взгляд) интерфейс, чем у
    MTOR, но беднее возможности экспорта геометии. MayaMan поддерживает
    материалы, разработанные в HyperShade, используется в студии Animal
    Logic в работе над фильмами и является, по сути, самой сильной
    альтернативой MTORу из существующих.
  

 Помимо MayaMan, австралийские разработчики также
    предлагают аналогичные плагины для 3dsmax и Softimage с
    соответсвующими названиями – MaxMan и SoftMan. Если вы пользуетесь
    “максом” или всё ещё не перешли с “софта” на XSI –хороший повод для
    того, чтобы попробовать Renderman, теперь есть и у вас.
 
 \section*{Liquid}
  

 Существует целая каста людей, которые по
    принципиальным соображениям пользуются исключительно бесплатным
    программным обеспечением, предоставляемым исключительно с исходным
    кодом. Бальзамом на душу таким людям будет наш следующий плагин для
    Maya – Liquid.
  

 Разработанный Колином Донкастером (Colin Doncaster), лИквид был первоначально
    предназначен для нужд студии Weta Digital (мы ещё не раз встретим
    название этой студии в нашей главе), а именно – для фильма
    “Властелин Колец”. Подобное требование наложило свой отпечаток на
    внутреннюю структуру и функциональность плагина, который был
    максимально заточен под большие сцены, сложные написанные руками
    шейдеры и большое количество текстур – и не очень-то дружественен
    по отношению к неопытным рендерменщикам. 
  

 Впрочем, в 2002м году Колин покинул студию Вета и
    забрал Liquid с собой. После долгих раздумий, он выложил исходный
    код плагина на сайте http://liquidmaya.sourceforge.net/ -
    и с этого момента фактически прекратил участие в судьбе своего
    проекта. Впрочем, его программа попала в хорошие руки и целая
    группа разработчиков занялась доводкой и улучшением продукта,
    выкладывая всё новые и новые версии плагина и его исходного кода.
    Отрадно отметить, что среди новых родителей Liquid’а есть и
    русскоязычные программисты.
  

 Из всех рассмотренных нами плагинов для работы с
    Renderman, Liquid является, по сути, самым неудобным в повседневной
    работе. С другой стороны, у этого плагина огромный потенциал –
    изначально ориентированная на задачи реального кинопроизводства
    архитектура, заложенная в эту архитектуру скорость работы и
    гибкость дают надежду на дальнейшее развитие этого продукта –
    например, некоторые студии вовсю используют Liquid в качестве
    основы для своих собственных экспортеров и студийных
    пайплайнов.
  

 Впрочем, в списке этих студий больше нет Weta
    Digital. После ухода Колина оказалось, что второй и третий фильм
    трилогии “Властелин Колец” серьёзно повысили требования к плагину –
    а исправлять его было уже некому, да и копирайт на код остался у
    автора. Поэтому в студии внедрили новое решение, по иронии судьбы
    (а скорее, обыгрывая предыдущий продукт - Liquid) названное
    Solid.
  

 А мы тем временем от средств вывода геометрии
    перейдём к редактированию шейдеров. Вот вы, молодой и красивый (а
    ещё лучше – молодая и красивая), сидите перед компьютером, в
    плейере загружен последний альбом любимой группы, пакет с соком
    открыт, вы кладёте руки на клавиатуру – и сталкиваетесь с одной из
    самых неприятных сторон Renderman Shading Language – писать на нём
    шейдеры вручную трудно и долго, в особенности для тех из вас, кто
    привык к красивым интерфейсам и большим кнопкам. Множество компаний
    и просто умельцев прикладывает массу усилий для того, чтобы
    облегчить труд шейдерописателя – начиная с модулей для подсветки
    синтаксиса SL в самых популярных текстовых редакторах (вот вы
    смеётесь, а это очень важно и удобно) и заканчивая визуальными
    конструкторами, позволяющими создавать шейдеры при помощи мыши.
    Наибольший интерес для новичков (равно как и просто очень занятых
    людей) представляет именно последняя категория программ.
    Воспользуюсь служебным положением  и обойду вниманием такие входящие
    в эту категорию продукты, как ShadeTree, Vshade и Shrimp. Вместо
    этого мы поговорим о программе, автором которой я являюсь – о
    ShaderMan.
  \chapter*{ShaderMan}
  

 Когда-то очень-очень давно, когда деревья были
    высокими, а в prman ещё не встроили raytracing, я поймал себя на
    мысли, что в процессе написания шейдеров на SL всё время повторяю
    одни и те же шаги и постоянно использую одни и те же куски кода в
    качестве образцов. Более того, у меня скопилась целая библиотека
    таких кусков и готовых шейдеров, на которых я тренировался и
    учился,  изучая чужой
    код и воплощённые в жизнь идеи (эта библиотека существует по сей
    день; в ней 685 файлов и у меня сложилось впечатление, что в
    некоторые из них я ещё ни разу не смотрел). Так вот, я начал
    подумывать о том, чтобы каким-то образом автоматизировать свою
    работу, написав специальную программу, упрощающую создание шейдеров
    – но я не был уверен в том, как должна работать такая программа, на
    какие принципы опираться и какой визуальный интерфейс является
    наиболее удобным и быстрым.
  

 Примерно в это же время появился новый продукт
    комнании (тогда ещё) Softimage под названием XSI, в котором похожая
    идея с визуальным редактированием материалов была воплощена в
    Render Tree – и после первого же скриншота я понял, в каком
    направлении мне следует двигаться.
  

 Как автор программы, я  могу долго (подозреваю, что очень
    и очень долго, но проверять не рискну) философствовать о том, какие
    идеи заложены в её основу, как именно используется та или иная
    функция, в чём смысл всего происходящего и почему аборигены съели
    Кука. Но вместо этого – давайте попробуем сделать простой шейдер и
    при этом сэкономить на типографской краске и обойтись минимумом
    скриншотов.
  

 Запускаем ShaderMan. В окошке Welcome, если вы его
    не закрыли и не отключили, выбираем кнопку Start New Shader; если
    всё-таки закрыли – то выбираем из меню File=>New… Имя шейдера по
    умолчанию (defaultname0) вполне подойдёт.
  

 В рабочей области программы появился небольшой
    элемент – в терминологии ShaderMan он называется кирпичиком
    (brick).
  

 \gr{image095}
  

 Аналогия простая – из таких вот кирпичей мы
    собираемся построить нечто полезное и красивое. С версией 0.7.0.0
    программы поставляется 345 различных кирпичиков – вполне
    естественно, что выводить их все простым списком было бы глупо и
    некрасиво по отношению к пользователю, поэтому все элементы
    рассортированы по тематическим подгруппам – которые вы и видите в
    окне слева в виде дерева (или по правому клику на свободном
    пространстве в окне ShaderMan – в виде меню).
  

 Для
    любознательных: эта древовидная
    структура всего-лишь повторяет структуру поддиректорий /brick/. Все
    брики являются файлами в формате XML с расширением – никогда не
    угадаете - *.br. Раз уж вы это читаете, а я это пишу – XML читается
    как ИГЗЭМ{\bf ЭЛЬ.}
  

 Чтобы добавить новый элемент в наш шейдер, можно
    воспользоваться левым деревом (и перетянуть-бросить – в смысле
    драг’н’дропнуть - необходимые элементы из него), контекстным меню
    рабочего стола или главным меню программы. Не суть важно, как –
    добавим несколько бриков. Какие?
  

 В первую очередь (поскольку мы не экстремалы, а
    всего-лишь начинающие шейдерописатели), это будет готовый шейдер
    (папка ready shaders) или готовая модель освещения (shading
    models). Впоследствии вы сможете сами строить новые модели
    освещения, пользуясь бриками более низкого уровня – предоставляется
    и такая возможность – а пока возьмём на вооружение уже готовые
    наработки и добавим в шейдер кирпичик OrenNayar.
  

  \gr{image097}
  

 Для
    любознательных: в наборе есть и
    стандартные для Maya lamber и blinn, но мы специально используем
    модель освещения Орена и Найара. Названная так в честь своих
    авторов, данная модель является логическим продолжением широко
    известной модели Ламберта и часто применяется при имитации
    различных тканей.
  

 Через контекстное меню переводим оба брика в режим
    с локальными изображениями (Show local preview) и, если у вас
    правильно настроен рендерер – видим нечто подобное:
  

 \gr{image099}
  

 Что-то явно не так в Датском Королевстве – все
    элементы на месте, но элемент Surface – чёрный, а элемент OrenNayar
    – показывает образец шейдера. Вывод логичен – необходимо
    подсоединить выход из одного брика к входу из другого, что мы
    мышкой и делаем, после чего сразу же тыкаем в картинку на
    Surface:
  

 \gr{image101}
  

 Вот, собственно, и всё. Окинем взглядом имеющуюся
    диспозицию:
  

 \gr{image103}
  

 В правом верхнем углу окна имеем специальную
    область просмотра результатов рендеринга. Достаточно выбрать мышкой
    из списка чего-нибудь нешарообразное, например, чайник (teapot) и
    кликнуть в эту область – и сразу появится большое превью, с которым
    можно делать всякие разности, например, смотреть по каналам RGB или
    установить рендеринг не всей картинки, а только её части – в общем,
    имеется в наличии некий мутант, собранный из возможностей fcheck,
    it и других просмотрщиков.
  

 Для
    продвинутых: для того, чтобы prman
    мог рендерить сразу в окно ShaderMan, мне пришлось написать
    специальный display driver, который затем переписывался ещё не раз,
    в том числе полностью – для Entropy, которая не поддерживала
    стандартные драйвера дисплея.
  

 \gr{image105}
  

 Внизу под областью просмотра у нас – набор
    параметров шейдера, настройки для preview и специальные редакторы
    для RIB-кода (некий аналог RIBbox из MTOR). В самом низу окна –
    консоль, в которую рендереры выводят статусную информацию – иногда
    полезно поглядывать в это окошко, чтобы вовремя обнаружить
    возможные ошибки при рендеринге.
  

 А мы тем временем через меню View отключаем все эти
    навороты и остаёмся с голым рабочим столом программы. При помощи
    контекстное меню добавляем новый брик – fractal/fractal.
    Присоединяем его выход к входу Kd OrenNayar – кстати, о Kd. Везде,
    где только можно было, я старался придерживаться стандартных имён
    параметров шейдеров, которые приняты на практике, например, Kd –
    это коэффициент диффузии. Но в определённый момент столкнувшись с
    вопросами новичков, в особенности пришедших из Maya HyperShade, я
    придумал опцию View => Friendly Shader Parameters, которая даёт
    параметрам шейдеров более дружественные и понятные
    имена:
  

 \gr{image107}
  

 Кроме визуального, а ля HyperShade, режима
    редактирования шейдеров в ShaderMan реализован ещё один
    режим, a la Slim. Дабл-кликните (это в смысле - дважды) на любом брике,
    например, на достославном Oren-Nayar:
  

 \gr{image109}
  

 Далее всё работает почти как в Slim – прямо из
    этого окна можно присоединять новые элементы, смотреть на
    результаты на различных стадиях – в общем, каждый выбирает свой
    метод редактирования шейдера.
  

 Каким образом результат нашей работы можно
    использовать в Maya? Собственно в Maya – почти никак, потому что
    конечным результатом работы в ShaderMan является шейдер SL. Кроме
    этого, вы можете экспортировать темплейты Slim - и уже их
    непосредственно подключать к MTOR – либо специальные файлы для
    Liquid – и использовать их. Ну и наконец, можно отрендерить шейдер
    в текстуру, которую затем подсасывать в майские шейдеры.
  

 На этой мажорной ноте придётся оборвать наше
    повествование. Я много чего  не рассказал вам о ShaderMan.
    Например, о маркинговом меню (идея которого, которую я впервые
    увидел в продуктах Alias, поразила меня до глубины души). Или о
    потенциальной расширяемости программы, основанной на простом
    структурированом текстовом формате XML (запомнили? {\it ИГЗЭМЭЛЬ!}) и усиленной языком
    программирования Tcl. Или о встроенной поддержке более чем десятка
    Renderman-совместимых рендереров – и лёгком добавлении новых (ещё
    раз повторяем волшебное слово – правильно, XML). Или о
    дружественности продукта для новичков и одновременной мощности для
    опытных (я тоже раньше думал, что это почти нереально). Или про –
    о, вспомнил. О пункте меню File=> Import.
  

 Найдите на диске один из шейдеров, с которыми мы
    экспериментировали раньше, например, textured\_noise.sl:
  

surface
    textured\_noise ( float freq=100; )
  

{
  

  Ci = texture("texture.tx", s,
    t)*noise(freq*s,freq*t);
  

 }
  

 Вызываем из главного меню ShaderMan
    File=>Import=>Shader source as new brick…, выбираем только
    что найденный нами файл и через несколько секунд (в течение которых
    запускается и выполняется конвертер из шейдера SL в формат брика
    BR) на рабочем столе ShaderMan появляется новый брик –
    textured\_noise. Присоединяем его к Surface Color кирпичика
    Oren-Nayar и немедленно получаем:
  

 \gr{image111}
  

 Таким образом мы можем загрузить в виде нового
    брика практически любой готовый шейдер, написанный на языке
    Renderman SL, и в дальнейшем использовать его как часть своих
    собственных шейдеров. Но это ещё не всё. В том же меню ShaderMan
    File=>Import выбираем другой подпункт – RIB file as template.
    Находим нашего первенца, test1.rib – тот самый, неосвещённый с
    единственным одиноким полигоном перпендикулярно камере:
  

Display
    "RenderMan" "framebuffer" "rgb"
  

Format
    256 192 -1
  

ShadingRate
    1
  

WorldBegin
  

   Surface "plastic"
  

   Polygon "P" [0.5 0.5 0.5 0.5 -0.5 0.5 -0.5
    -0.5
  

                 0.5 -0.5 0.5 0.5]
  

 WorldEnd
  

 Пока не случилось непоправимое, быстренько
    открывает этот файл в тектовом редакторе, вспоминаем, чему мы
    учились всю эту главу, и добавляем источники света (для ленивых или
    тех, кто читает книжку, сидя в метро – файл
    test\_shaderman.rib):
  

Display
    "RenderMan" "framebuffer" "rgb"
  

Format
    256 192 -1
  

ShadingRate
    1
  

WorldBegin
  

     LightSource "ambientlight" 500
  

                 "lightcolor" [0.051 0.051 0.051]
  

     LightSource "distantlight" 501 "from" [1
    1.5 -1] "to" [0 0 0]
  

     LightSource "distantlight" 502
    "lightcolor" [0.2 0.2 0.2]
  

                 "from" [-1.3 -1.2 -1.0] "to" [0 0 0]
  

     AttributeBegin
  

       Attribute "identifier" "name"
    ["polyShape"]
  

       Surface "plastic"
  

       Polygon "P" [0.5 0.5 0.5 0.5
    -0.5 0.5 -0.5
  

  -0.5 0.5 -0.5 0.5 0.5]
  

     AttributeEnd
  

WorldEnd
  

 Кроме источников света мы также добавили аттрибут
    “identifier”, который позволяет нам дать имя объекту – такие
    аттрибуты добавляют, в том числе, все экспортеры из Maya в RIB. И
    только теперь импортируем в ShaderMan.
  

 Для
    продвинутых: те же самые операции
    по импорту файлов можно проделать, просто перетянув их мышкой из
    Windows Explorer.
  

 Вроде бы ничего не поменялось и не произошло, кроме
    разве что сообщения в консольном окне (и строке статуса) ShaderMan
    - /rib\_example/test\_shaderman.rib : done. Для того, чтобы понять
    всю важность произошедшего, включите обратно область preview и
    попробуйте выбрать в выпадающем списке объект:
  

 \gr{image113}
  

 Мы только что загрузили внешний RIB файл в качестве
    сцены для использования в ShaderMan – и программа предоставила нам
    возможность выбрать объект, на котором мы будем тестировать наш
    шейдер (не случайно мы дали нашему объекту имя).
  

 Попробуем проиллюстрировать все эти возможности,
    модицифировав диаграмму передачи данных и файлов в рамках RAT.
    Стоит обратить внимание на то, что некоторые стрелочки в схеме –
    двунаправленные; это означает, например, что ShaderMan может как
    экспортировать, так и импортировать файлы SL.
  

 \gr{image115}
  

 Название ShaderMan появилось не сразу. Первые
    публичные релизы программы (а раздавал я их сначала только
    русскоязычным пользователям) назывались TheTool – собственно,
    веб-страница продукта до сих пор носит такое название -
    http://www.dream.com.ua/thetool.html. И идея с новым названием и
    вообще выкладыванием своего творения на публику тоже пришла не
    сразу – изначально это был внутренний проект, программа для себя,
    чтобы сделать свою работу интереснее и проще. Лишь потом, получив
    много полезных и приятных откликов от первых пользователей
    ShaderMan’а, я озаботился всем тем, что отличает нормальную
    программу от кустарной поделки – написанием помощи, созданием
    нормальной веб-страницы, инфраструктурой для поддержки
    пользователей. И ещё – именно из-за того самого первого фидбэка
    программа осталась бесплатной. В конце концов, всё, что мне было
    нужно от своего детища – это получить инструмент для дальнейших
    исследований и получить удовольствие от самого процесса написания
    программы. И если ShaderMan оказался полезен не только для меня, но
    и для (многочисленных, насколько я могу судить по количеству
    скачиваний) пользователей – это очень приятно.
  

 Ведь так?

\chapter*{Трюки, ухищрения или старинное китайское искусство Chi Ting – в массы.}
  

 Вы слышали раньше про старинное китайские искусство
    Chi Ting? Первая информация об исследованиях этого искусства
    появилась в 1988 году в статье Джима Блинна (того самого, который
    Blinn shading model) под названием "THE ANCIENT CHINESE ART OF
    CHI-TING" \footnote{http://research.microsoft.com/users/blinn/CHITING.HTM},
    однако Джим указал, что первенство в придумывании названия для
    этого артефакта принадлежит Биллу Этре. Первое же упоминание об
    использовании данного искуства в отношении Renderman, следует
    полагать, появилось в Siggraph-92 Renderman Course Notes (ну вот
    как это название на русский перевести прикажете, а? – пусть будут
    Материалы Учебного Курса по Renderman)
  

 Ладно, ладно, я больше не буду издеваться –
    название Chi Ting, если его прочитать в соответствии с правилами
    английского языка, созвучно английскому же слову cheating – то
    есть, жульничать, надувать.
  

 Работая в Лаборатории Реактивного Движения (NASA
    Jet Propulsion Laboratory) над анимационным роликом, рассказывающем
    о пролёте станции исследования дальнего космоса Вояджер над
    Сатурном – вы наверняка видели этот ролик – Джимм заметил, что, на
    самом деле, зрителя мало интересует тот факт, что созданная вами
    анимация была построена на точных физических принципах и отображает
    истинную картину мира. Более того, использовать “взрослую физику”
    при производстве такой анимации зачастую попросту невозможно (в
    силу нашего незнания, неготовности или физической невозможности
    проводить расчёты такой сложности). Обратите внимание – я не говорю
    про физические эксперименты и визуализацию их результатов, я говорю
    про Вояджер, красиво пролетающий мимо колец Сатурна.
  

 Так почему бы в таком случае не срезать несколько
    углов и не обмануть зрителя, упростив расчёты или изменив алгоритм,
    например? Если в стакане красиво плещется вода, то
    среднестатистическому посетителю кинотеатра (а
    телерекламосмотрителю – и подавно) всё равно, настоящая там вода
    была в стакане, сымитированная на суперкомпьютере при помощи
    суперпродвинутых метафизических über-формул – или это просто
    “грязный хак”, который просто выглядит как вода в
    стакане.
  

 Для
    любознательных: буквально на днях
    прочитал весьма поучительную историю о том, как делаются рекламные
    фотографии сухих молочных завтраков – на языке потенциального
    противника они называются crisps, у нас пусть будут - мюсли.
    Общеизвестно, что мюсли – штука катастрофически хрупкая, и обычно
    на донышке пачки вместо той красотищи, что изображена на коробке,
    обнаруживается лишь сухое мелкодисперсное месиво. Так вот для того,
    чтобы сделать фотографию для оформления коробки, на пол высыпают
    целый ящик мюслей, после чего вручную ювелирными щипчиками отбирают
    из огромной кучи 50-60 абсолютно целых жирных кусочков (обычно к
    моменту начала ковыряния в куче их где-то столько целых и
    остаётся). Берётся сосуд, в него опять же вручную, по одному
    укладываются отобранные ранее кусочки, затем промежутки заполняются
    ярко-белым гелем для волос (в отличие от клея, он не портится от
    яркого света и высокой температуры, создаваемых софитами) – это
    будет видимость молока. В зависимости от сорта сухого завтрака
    добавляются всякие вишенки или шоколадки – на самом деле внутри
    коробки их нет, но на фотографии они обязательно будут. И вот уже
    эта несъедобная конструкция фотографируется.
  

 Как мне кажется, стандарт Renderman и
    поддерживающие его рендереры являются средствами, которые чуть ли
    не подталкивают пользователя к оптимизации сцены и использованию
    всевозможных ухищрений. Сочетание простого текстового формата,
    обширных возможностей настройки качества и скорости рендеринга и
    достаточно высокая скорость рендеринга дают возможность постоянно
    экспериментировать с различными настройками, добиваясь оптимального
    баланса качества и скорости. Как уже стало заведено в этой главе,
    мы не будем сильно углубляться в дебри настроек и фокусов, показав
    лишь наиболее очевидные вещи и в некоторых случаях проиллюстрировав
    эти трюки простыми примерами.\hfil\break
    И начнём мы с действительно простых вещей – а именно некоторых
    особенностей Renderman-совместимых рендереров.
  \chapter*{ShadingRate}
  

 В RenderMan существуют несколько настроек качества
    изображения, которые являются критическими и без знания о которых
    вы не сможете добиться нормального баланса между временем расчёта и
    качеством картинки. Основная такая настройка – это
    ShadingRate.
  

 Prman, равно как и многие другие рендереры,
    накладывает шейдеры после тесселяции объекта на микрополигоны -
    будем называть их сэмплами. Если посмотреть на этот вопрос немного
    упрощённо, то ShadingRate – это то количество пикселей, которое
    рендерер будет просчитывать за один вызов шейдера. Таким образом,
    вызов
  

ShadingRate 0.25
  

 означает, что на каждый пиксель картинки шейдер
    будет вызван приблизительно 4 раза (1/0.25) в различных точках
    внутри этого пикселя, а полученные значения будут усреднены и,
    соответственно, качество картинки будет гораздо выше. Установка же
    такого значения:
  

ShadingRate 20
  

 означает, что вызов шейдера будет производиться
    один раз на примерно 20 пикселей, что даст гораздо худшее качество
    – но рендеринг завершится очень быстро. Так вот, поскольку именно
    шейдинг в процессе обсчёта картинки занимает бОльшую часть времени
    (в особенности для сложных процедурных шейдеров, например, с
    нойзом), то эта настройка является самой часто используемой в
    Renderman.
  

 Давайте рассмотрим простой пример. Откроем простой
    RIB с шариком, с которым мы экспериментировали на протяжение всей
    нашей главы, и настроим в нём величину ShadingRate (файл на диске
    под именем shadingrate025.rib):
  

Display "RenderMan"
    "framebuffer" "rgb"
  

Format 256 192 1
  

ShadingRate 0.25
  

Translate 0 0
    2.7650300856
  

WorldBegin
  

  TransformBegin
  

    LightSource
    "ambientlight" 500
  

     "lightcolor" [0.051
    0.051 0.051]
  

    LightSource
    "distantlight" 501 "from" [1 1.5 -1] "to" [0 0 0]
  

    LightSource
    "distantlight" 502 "lightcolor" [0.2 0.2 0.2]
  

                "from" [-1.3 -1.2 -1.0] "to" [0 0 0]
  

  TransformEnd
  

  Surface "plastic"
  

  Sphere 1 -1 1 360
  

WorldEnd
  

 Напускаем prman на этот файл и получаем в результате:
  

 \gr{image117}
  

 Для
    продвинутых: в этом примере вы
    использовали ShadingRate, равный 0.25, то есть 4 вызова шейдера для
    каждого пикселя. По умолчанию, если величину ShadingRate в сцене не
    указывать, то она будет выставлена в 0.5. Большинство экспортеров
    трехмерной сцены  в
    RIB,  и SLIM в их
    числе, специально выставляют ShadingRate – об этом следует помнить,
    особенно в процессе работы в SLIMе (выставляющем по умолчанию 5.0),
    когда вам начинает казаться, что качество картинки в окне
    предварительного просмотра не очень высокое.
  

 Изменяем значение ShadingRate (файл на диске
    shadingrate64.rib):
  

Display "RenderMan"
    "framebuffer" "rgb"
  

Format 256 192 1
  

ShadingRate 64
  

Translate 0 0
    2.7650300856
  

WorldBegin
  

  TransformBegin
  

    LightSource
    "ambientlight" 500
  

                "lightcolor" [0.051 0.051 0.051]
  

    LightSource
    "distantlight" 501 "from" [1 1.5 -1] "to" [0 0 0]
  

    LightSource
    "distantlight" 502 "lightcolor" [0.2 0.2 0.2]
  

                "from" [-1.3 -1.2 -1.0] "to" [0 0 0]
  

  TransformEnd
  

  Surface "plastic"
  

  Sphere 1 -1 1 360
  

WorldEnd
  

 После рендеринга видим:
  

 \gr{image119}
  

 Рендеринг закончился гораздо быстрее, но качество
    результата оставляет желать лучшего – шейдер вызывался один раз на
    64 пикселя и теперь вы видим границы между сэмплами.
  

 Существует простой способ сгладить эти неприятные
    грани, добавив в RIB-файл где-нибудь рядом с ShadingRate
    строку
  

ShadingInterpolation  "smooth".
  

 Prman после этого будет интерполировать значения
    соседних шейдинг сэмплов, и наш шарик вновь станет гладким. Но на
    примере с текстурой из файла будет очевидно, что изображение
    текстуры от этого чётче не стало, а всего лишь «размазалось» по
    поверхности.
  

 Для
    продвинутых: как понять, насколько
    быстрее закончился рендеринг? Можно посидеть над компьютером с
    секундомером или песочными часами в руках или написать скрипт,
    который будет отсчитывать время за вас. А можно вставить в RIB
    опцию вывода статистики:
  

Option "statistics"
    "endofframe"
    [1]
  

 {\it или}
  

Option "statistics"
    "endofframe"
    [2]
  

 в зависимости
    от того, насколько детальная информация вас интересует. Для наших
    двух примеров оказывается, что количество сэмплов, которое prman
    создал в процессе рендеринга для каждой из сцен, равняется 1030 для
    первой сцены и всего лишь 56 – для второй. Экономия памяти
    составила 1 мегабайт; экономия времени – несколько раз (поскольку
    картинка маленькая и простая, то сосчитать точно количество раз
    сложно даже с такой статистикой – для второй сцены она показала 0
    секунд).
  

 Особенностью Renderman, которая позволяет очень
    гибко использовать ShadingRate, является тот факт, что почти все
    переменные в сцене можно указывать как глобально – один раз на
    сцену – так и локально – для каждой модели и чуть ли не для каждого
    полигона. Делается это очень просто, например, вот так (приведен
    небольшой фрагмент RIB-файла):
  
\begin{lstlisting}[frame=single, framerule=0pt, framesep=10pt, xleftmargin=10pt, xrightmargin=10pt]
ShadingRate 0.25 # Это значение будет работать для всей сцены
AttributeBegin
ShadingRate 4 # Кроме этого шарика, для которого
              # определено своё собственное значение
Surface "plastic"
Sphere 1 -1 1 360
AttributeEnd
\end{lstlisting}
  

 Таким образом, вы можете установить высокое
    значение ShadingRate для той части сцены, которая находится близко
    к камере, и низкое – для дальних или несущественных объектов. Более
    того, используя Mel, это значение можно сделать динамически
    зависящим от расстояния от объекта до камеры.
  

 Для
    продвинутых: Опытные TD советуют
    использовать для просчёта карт теней высокие значения ShadingRate,
    вплоть до 32 (но всё же не больше 8, если используется
    displacement). Как оказалось, на видимое качество теней это почти
    не влияет, а ускорение просчёта составляет до 20
    раз.
  

 К сожалению,
    этот трюк хорошо работает не для всех типов теней; также его не
    рекомендуется применять для key light и областей с сильным
    дисплейсментом из-за возможности появления “грязи”.Сами Pixar также
    советуют при использовании motion blur увеличивать ShadingRate –
    поскольку объекты и так будут “разблюрены”, то небольшое падение
    качества к видимому ухудшению картинки не приведёт.
  

 По опыту
    крупных зарубежных студий, скорее всего, к моменту окончательного
    рендеринга у вас не будет большого выбора в значениях ShadingRate,
    и тому есть несколько причин. Во-первых, поддержание единого
    значения ShadimgRate в сцене позволяет избежать разнобоя в картинке
    – грубо говоря, артефакты на всех моделях выглядят одинаково.
    Далее, модели с высокочастотными текстурами (особенно с большим их
    количеством) очень требовательны к ShadingRate. И наконец, даже
    если объект движется (и мы скрываем его при помощи motion blur и,
    по совету Pixar, увеличиваем SR), всё равно будут несколько кадров,
    к которых он будет неподвижен – и именно эти кадры должны показать
    работу шейдерописателей  и текстурных художников во всей красе
    – потому что именно на этих кадрах (согласно закону бутерброда)
    будут делать стоп-кадры на своих DVD-проигрывателях
    будущие трёхмерщики – и изучать, изучать, изучать...
 
 \section*{PixelSamples}
  

 Архитектура REYES, на которой построен prman, в
    момент своего создания сильно отличалась от общепринятых подходов к
    трехмерному рендерингу, в частности, из-за того, что процесс
    шейдинга в ней происходит до определения видимости и вычисления
    краевого антиалиасинга поверхностей.
  

 Как это происходит? Получается, что на какой-то
    стадии шейдеры вызваны, объекты уже закрашены, а картинка всё ещё
    девственно чиста? Что же тогда «красят» шейдеры?
  

 Всё очень просто: закраска происходит не в
    двумерном пространстве картинной плоскости, а в трехмерном мире
    сцены. Дело в том, что ещё до шейдинга prman производит тесселляцию
    (разбиение) видимых поверхностей на микрополигоны, размеры которых
    приблизительно определяются числом занимаемых ими на картинной
    плоскости пикселей – да-да, ShadingRate как раз тут и говорит свое
    веское слово. После такой тесселляции в распоряжении рендерера
    оказывается сетка из микрополигонов, которые, с одной стороны,
    являются геометрическими объектами – малюсенькими прямоугольными
    патчами; с другой стороны, их можно рассматривать как шейдинг
    сэмплы - минимальные кусочки поверхностей, закрашиваемые сплошным
    цветом.
  

 Вот здесь и начинается собственно шейдинг: для
    каждого такого патчика будет сформирован  однократный вызов соответствующих
    шейдеров и каждый патчик получит свой цвет и прозрачность. А если
    на поверхность назначен и displacement шейдер – патчики-сэмплы
    будуть сдвинуты в  указанном шейдером направлении. Но, повторяю, это все пока
    происходит в трехмерном пространстве, до картинной плоскости дело
    еще не дошло!
  

 Следующая задача – спроецировать эту трехмерную
    информацию на пиксели картинки. Тут вроде бы всё просто – нужно
    только грамотно определить видимость шейдинг сэмплов из пикселей
    картинки, чтобы определить, какой шейдинг сэмпл закрасит какой
    пиксель.
  

 В реальности же всё намного сложнее, в основном,
    из-за того, что тут мы попадаем в царство Его Величества
    Антиалиасинга. Не вдаваясь в подробности, скажу лишь, что
    для  этой цели алгоритм
    REYES использует oversampling, то есть многократный точечный
    сэмплинг. Через каждый пиксель картинной плоскости (грубо говоря –
    нашего конечного изображения) провешивается несколько лучиков
    (точнее будет сказать, point samples, или точечных сэмплов, – не
    путать с шейдинг сэмплами, с одной стороны, и со световыми лучами
    при рейтрейсинге – с другой стороны!) к сцене, и выясняется, в
    шейдинг сэмплы с каким цветом/прозрачностью они упёрлись. Эти
    значения взвешиваются по каждому пикселю, с использованием одной из
    нескольких доступных для пользователя фильтрующих функций и в
    зависимости от удаленности лучиков от центра пикселя. Результаты
    взвешивания записываются в пиксель готовой картинки.
  

 Так вот, параметры PixelSamples определяют
    количество точечных сэмплов на каждый пиксель – по вертикали и
    горизонтали. PixelSamples 5 5 заставит prman запросить
    цвет/прозрачность шейдинг сэмплов, спроецировааных в каждый пиксель
    ровно 5 х 5 = 25 раз.
  

 Если копнуть ещё глубже, то выяснится, что в
    реальности Prman провешивает лучики, не разбивая каждый пиксель на
    регулярную сетку, размером, скажем, 5х5, как в нашем примере, а
    случайным образом разбрасывая их в пределах ячеек этой сетки. По
    науке это называется Stochastic Sampling – именно на эту технологию
    Pixar получила свой патент, и именно при помощи этого патента
    засудила компанию Exluna.
  

 Интересно, что такие эффекты как Motion Blur и
    Depth of Field prman-ом производятся как раз при помощи
    PixelSamples. Всё очень просто: направления PixalSample-лучей
    опять-таки случайным образом изменяются c учетом вектора «движения»
    шейдинг сэмплов и/или их удаленности от «точки фокуса» камеры.
    Собственно, вот вам и объяснение того, что prman держит пальму
    первенства в скорости рендеринга Motion Blur – там, где
    другие рендереры при включении MB замедляются в
    разы, он теряет всего десяток-другой процентов, потому что
    продолжает использовать те же самые алгоритмы, что и до этого. И
    именно поэтому сцены с motion blur являются
    более  требовательными к PixelSamples, чем
    обычные – ведь чем сильнее блюр, тем больше может потребоваться
    сэмплов, чтобы размазанность выглядела равномерно, а не
    “разбросанной по экрану”.\hfil\break
    Ну и последнее уточнение, без которого описанная выше схема
    рендеринга не имеет никакого практического значения: алгоритм REYES
    потребует колоссального количества памяти, если попытаться
    реализовать его, начиная с тесселляции поверхностей, для всей
    картинки сразу. К счастью, он прекрасно работает и с ее частями
    (помните, я рассказывал про бакеты?), по умолчанию, размером 16х16
    пикселей, из которых потом, как из мозаики, собирается все
    изображение.
  

 Ффух. Отдышались. Поехали дальше.
  

 По умолчанию величина PixelSamples в prman
    установлена как:
  

PixelSamples 2
    2
  

 Это означает, что на каждый пиксель приходится 4
    точечных сэмпла . Если мы изменим эту величину (файл на диске
    pixelsamples11.rib):
  

Display "RenderMan"
    "framebuffer" "rgb"
  

Format 256 192 1
  

ShadingRate 64
  

PixelSamples 1 1
  

Translate 0 0
    2.7650300856
  

WorldBegin
  

  TransformBegin
  

    LightSource
    "ambientlight" 500
  

                "lightcolor" [0.051 0.051 0.051]
  

    LightSource
    "distantlight" 501 "from" [1 1.5 -1] "to" [0 0 0]
  

    LightSource
    "distantlight" 502 "lightcolor" [0.2 0.2 0.2]
  

                "from" [-1.3 -1.2 -1.0] "to" [0 0 0]
  

  TransformEnd
  

   Surface "plastic"
  

   Sphere 1 -1 1 360
  

WorldEnd
  

 то получим картинку без антиалиасинга и с видимыми
    проблемами с шейдингом – но получим её очень быстро:
  

 \gr{image121}
  

 Управляя двумя переменными – ShadingRate (для любых
    участков вашей сцены) и PixelSamples (для всей сцены) – можно очень
    гибко настраивать рендеринг, добиваясь максимального качества за
    минимальное время.
  

 Для
    продвинутых: покажем, как управлять
    этими параметрами при помощи утилиты cpp – как мы отмечали раньше,
    такая утилита есть в поставке почти всех рендереров, совместимых с
    Renderman. Для этого внесём изменения в RIB (вы можете просто
    открыть cpptest.rib с диска):
  

Display
    "RenderMan" "framebuffer" "rgb"
  

Format
    256 192 -1
  

ShadingRate
    SR
  

PixelSamples
    SX SY
  

Translate
    0 0 2.7650300856
  

{\it WorldBegin}
  

{\it    TransformBegin}
  

     LightSource "ambientlight"
    500
  

{\it                  "lightcolor" [0.051 0.051 0.051]}
  

     LightSource "distantlight" 501 "from" [1
    1.5 -1] "to" [0 0 0]
  

     LightSource "distantlight" 502
    "lightcolor" [0.2 0.2 0.2]
  

{\it                  "from" [-1.3 -1.2 -1.0] "to" [0 0 0]}
  

{\it    TransformEnd}
  

{\it    Surface "plastic"}
  

{\it   }  Sphere 1 -1 1 360
  

{\it  WorldEnd}
  

 Мы определили
    внутри файла три символьные константы – SR, SX и SY. Теперь вызовем
    в командной строке препроцессор:
  

cpp  -DSX=2
    -DSY=2
    -DSR=0.25 cpptest.rib  | prman
  

 и получим в
    результате картинку с необходимыми установками – константы внутри
    файла будут заменены на значения, объявленные при вызове
    cpp.
  

 Несколько
    неожиданный ход – использовать препроцессор языка C++, которым
    обычно обрабатываются файлы SL, для работы с файлов RIB. В этом вся
    соль расширяемости Renderman – поскольку и те, и другие файлы
    являются текстовыми, то для их обработки годятся одни и те же
    утилиты.

  \section*{LOD}
  

 Ещё одним трюком из арсенала читинга в Renderman
    является Level of Details, или сокращенно LOD.
  

 Смысл данной настройки состоит в том, что в RIB
    файле можно указывать не одну геометрическую модель, а несколько, и
    при этом указать, начиная с какого расстояния до камеры
    использовать ту или иную модель в данном месте сцены. Представьте
    себе битву с участием сотен сражающихся. Без LOD всё, что вы могли
    бы сделать – это использовать одну и ту же высококачественную
    модель как для бойцов первого плана, так и для статистов на заднем
    плане и что совсем печально – для огромной массы бойцов на
    горизонте, каждый из которых отрендерится максимум в десяток
    пикселей. Применяя LOD, вы можете автоматически подставлять для
    переднего плана – максимально детальную и проработанную модель; для
    среднего плана – что-то немного упрощённое; для массовки на заднем
    плане – палку-палку-огуречик или вообще один полигон с наложенной
    текстурой – и тем самым сильно уменьшить запросы по памяти и
    времени расчёта – а разницу никто не заметит.
  

 Особенно привлекательным представляется
    использование LOD в случае большого количества одинаковых или почти
    одинаковых объектов (толпа, деревья в лесу), но этот финт также
    активно используется при архитектурной визуализации (с её огромными
    и очень детальными моделями зданий на переднем плане), особенно при
    анимации внутри городских джунглей.
  

 Для
    продвинутых: как отголосок этой
    техники, иногда полезно иметь не только несколько моделей с
    различной детализацией проработки, но и различные наборы шейдеров и
    текстур – один набор, скажем, для первого плана, другой набор – для
    статистов вдалеке от камеры. Поверьте, дополнительное время,
    потраченное на настройку нескольких таких наборов, вернётся
    сторицей на этапе финального рендеринга.
  
\section*{Архивы и Замороженные Архивы.}
  

 В примере для продвинутых мы показали, что для
    обработки файлов RIB можно использовать такую утилиту, как
    препроцессор языка программирования C++, через который и так
    проходят все шейдеры перед компиляцией. Одной из возможностей
    препроцессора являются так называемые вложенные файлы (includes) –
    мы уже пробовали их при рассмотрении шейдеров. Аналогично вставке
    кода из файлов *.H в файлы *.SL при помощи препроцессора, мы можем
    вставлять RIB-файлы друг в друга.
  

 Вполне логично предположить, что в стандарте RIB
    существует аналогичная функциональность. Так оно и есть – вы можете
    вставить один RIB в другой (как кусок текста) при помощи команды
    ReadArchive, например, вот так:
  

 ReadArchive "templates/teapot.inc"
  

 А в файле teapot.inc, находящемся в директории
    templates, будет находиться необходимая геометрия (в нашем случае,
    чайник). Вынеся таким образом геометрию в отдельный файл, вы можете
    экспериментировать с глобальными настройками сцены, не опасаясь
    что-либо испортить в модели.
  

 В Maya нечто подобное под названием instancing
    появилось лишь в версии 6.5; в Renderman эта опция существует уже
    более 20ти лет.
  

 Но стандарт Renderman не остановился на достигнутом
    и ввёл дополнительную возможность под названием
    DelayedReadArchive.
  

Procedural  "DelayedReadArchive"
    [ "small.rib"
    ] [-1 1 -1 1 -1 1]
  

 Работает этот вызов в точности как и предыдущий, с
    одним важным отличием – в нём указана область нахождения данного
    объекта в пространстве (в просторечии, bounding box). Так вот,
    загрузка объекта в память рендерера произойдёт только в том случае,
    если область объекта пересечётся с видимой областью камеры, и,
    следовательно,  только
    тогда, когда расчёты сцены дойдут до этой области.
  

 В точности такую же философию проповедуют и
    процедурные генераторы RIB-кода, вызов которых можно осуществлять
    таким образом:
  
\begin{lstlisting}[frame=single, framerule=0pt, framesep=10pt, xleftmargin=10pt, xrightmargin=10pt]
Procedural "RunProgram" [ "teapot.pl" "some_teapot1_data" ] [-1 1 -1 1 -1 1 ]
\end{lstlisting}
  

 Код нашего скрипта на Перле вызовется только в том
    случае, если он попадёт в поле зрения камеры, и только тогда, когда
    это необходимо.
  

 Для
    любознательных: именно эту технологию
    применила студия Dreamworks при съёмках мультфильма Shark Tale. В
    силу производственной необходимости, вся геометрия для этого
    мультфильма подготавливалась в внутреннем формате студии, не
    совместимом с prman (для совсем продвинутых – это были специальные
    NURBSы). Поэтому был написан процедурный конвертер из этого
    формата; вызовы этого конвертера были вставлены в обычные RIBы.
    Таким образом, на диске геометрия хранилась в удобном для
    Dreamworks формате и преобразовывалась в RIB только во время
    рендеринга – и только в том случае, если в этом была
    необходимость.
  

 Возможности стандарта Renderman в области
    процедурной генерации геометрии дадут серьёзную фору многим
    существующим рендерерам. А мы тем временем обратимся к экзотическим
    возможностям prman.
 
 \section*{Разговаривающие и адаптивные шейдеры}
  

 Одной из новых возможностей, реализованных в
    Renderman Pro Server раньше, чем это появилось в стандарте,
    является возможность передавать сообщения между шейдерами (message
    passing). В рамках этой функциональности один шейдер может
    запросить значение переменной, определённой в другом шейдере,
    наложенном на эту же модель.
  

 Классический пример использования message passing –
    получение surface-шейдером информации из displacement-шейдера о
    величине отклонения и соответствующая закраска поверхности –
    например, более выпуклые области модели закрашиваются более тёмным
    цветом. Связывать таким образом можно любые шейдера, относящиеся к
    одному объекту – например, поставив displacement в зависимость от
    освещённости. Сложная на первый взгляд, передача сообщений между
    шейдерами предоставляет ещё один метод тонкой настройки отображения
    поверхностей, в том числе возможность реализовать адаптивные
    шейдеры – оптимизирующие свою функциональность непосредственно во
    время просчёта.
  

 Для
    продвинутых: простой пример
    адаптивного шейдера. В процессе расчёта ambient occlusion мы в
    каждом сэмпле выстреливаем некоторое количество лучиков в
    полусферу, причём для того, чтобы в картинке не было грязи и чтобы
    при анимации она не дрожала, это некоторое количество обычно
    является достаточно большим – как минимум, 256. Поскольку говорим
    мы о ray tracing, то любая оптимизация этого процесса будет к
    месту. Способ адаптации шейдера к свойствам поверхности, на которой
    он вызван, состоит в том, чтобы смотреть на локальную кривизну
    поверхности в данной точке (например, спрашивая об этом
    displacement shader, или просто вычисляя ее по изменению текстурных
    координат и их производных) и если кривизна большая – то в этой
    точке стрелять лучиков меньше (всё равно за мелкими деталями грязи
    видно не будет), а если небольшая – то побольше.
  

 Ещё один
    простой пример, чтобы немного взбударажить фонтазию нашего
    читателя. Попробуйте поуправлять величиной спекуляра или размером
    дисплейсмента в зависимости от расстояния до камеры. Или попробуйте
    наложить дисплейсмент только на края объекта, ограничившись в
    остальной части более примитивным и скоростным
    бампом.
  

 Для
    продвинутых-2: MTOR поддерживает
    мощный механизм для связи майских атрибутов с шейдерами. Чтобы
    воспользоваться этим механизмом, достаточно добавить к любому
    майскому объекту новый атрибут с именем, например, rmanF{\it myS} {\it  }{\it Приставка} {\it rmanF} (таких приставок
    несколько и они подробно описаны в документации) даёт MTORу сигнал, что
    данный атрибут должен быть обработан специальным образом и его
    значение должно быть подготовлено к экспорту. Всё, что остаётся
    сделать – это использовать полученное значение в шейдере, например,
    вот так:
  
surface test( varying float myS = 0; )

  
{
  
  
   Ci = color(myS);

  
}

  

 Ну вот, по внутренностям прошлись – а теперь
    настало время от оптимизации и ухищрений на микроуровне перейти к
    чистой воды читингу на макроуровне, а также просто полезным
    советам. Если предыдущие хинты имели смысл по большей части только
    для Renderman-совместимых рендереров, то теперь мы постараемся быть
    полезными и для других движков, в том числе для Mental Ray и самой
    Maya. Настало время для крупнокалиберной артиллерии и грязных
    трюков.

  \section*{Только то, что видно (Стыдно – когда видно)}
  

 Этот приём очевиден, но удивительно, насколько
    редко он применяется на практике. Не нужно включать в просчёт то,
    чего всё равно никто не увидит!
  

 Если какая-то часть сцены не попадает в камеру и не
    участвует в отражениях или в ambient occlusion – удаляйте её,
    вручную или при помощи скриптов. Если вместо огромного города,
    смоделированного до последнего гвоздика на подоконнике, вполне
    пройдёт текстура или matte-рисунок на заднике – удаляйте город.
    Считаете тени? Удаляйте всё, что не видно из этого источника света.
    Считаете отражения? Удаляйте всё, что по другую сторону зеркала. В
    конечном счёте, рендерер это сделает и сам – но зачем заставлять
    его прокачивать гигабайты мусора, из которого не получится ни
    одного полезного пикселя?
  

 Особую пользу этот совет принесёт вам в случае,
    если вы используете RAT. MTOR известен своей {\it втыкучестью} (слово нехорошее,
    нелитературное, но зато очень хорошо иллюстрирующее суть процесса)
    в момент экспорта сцены; небольшой скрипт на Mel, который
    пробежится по сцене перед экспортом и выключит ту геометрию,
    которая в данный момент из камеры не видна, ускорит генерацию RIBов
    для действительно сложных сцен  на порядки.
  
\section*{По слоям}
  

 Рендеринг “по слоям” уже давно стал обязательным
    требованием работы в продакшн хаусах. Обычно в отдельные файлы
    (“слои”) выводятся все персонажи; отдельно от них считаются
    спецэффекты (огонь, вода и прочие медные трубы); отдельно
    записываются environment. Затем все эти отдельные слои сводятся
    воедино уже на этапе композинга; это снимает ограничения на
    всевозможные настройки освещения и удаление-добавление персонажей в
    сцену, которые так любят проделывать заказчики в самый последний
    момент.
  

 Более того, в большинстве студий в отдельные слои
    выводятся не только персонажи, но тени от них, карты нормалей для
    персонажей, Z-buffer – толковые двухмерщики всегда найдут, где
    применить эту информацию, благо на время рендеринга сцены это
    влияет не сильно.
  

 Более того, Renderman даёт возможность выводить в
    отдельные изображения переменные, определённые внутри шейдеров –
    как глобальные (например, нормаль поверхности, цвет поверхности),
    так и локальные (например, диффузную составляющую освещённости
    конкретного персонажа). Остановимся на этом финте ушами
    поподробнее.
  

 Откройте в Maya сцену, которую мы использовали при
    тестировании RATа – ту самую, с овечкой (файл step2.ma вполне
    подойдёт). Убедитесь, что всё в порядке и сцена нормально
    рендерится – RenderMan=>Render.
  

 Воспользуемся встроенными возможностями MTOR для
    того, чтобы показать, как вывести в отдельный слой информацию о
    цвете модели, без учёта её освещённости. Для этого вызываем окно
    настроек RenderMan=>RenderMan Globals…, в нём в табе Display
    переключаемся в локальный таб второго уровня Secondary:
  

 \gr{image123}
  

 Выбираем канал, который мы хотели бы вывести
    -  нажимаем на кнопку
    New, после этого настраиваем параметры, как указано на
    скриншоте:
  

 \gr{image125}
  

 Что мы с вами только что сделали? Мы указали, что
    хотим отрендерить в отдельный файл значение Cs (выбранное из
    длинного списка доступных переменных в поле Mode), которое в языке
    шейдеров соответствует цвету объекта до того, как мы в шейдере
    применили к нему какие-то операции. Вызываем рендерер и видим на
    экране:
  

 \gr{image127}
  

 Обычный результат нашего рендеринга. Где же
    подевался запрашиваемый нами дополнительный слой? Он находится в
    поддиректории rmanpix вашего проекта. Посмотрим внутрь файла
    untitled.new\_channel.0001.tif, появившегося в этой директории, при
    помощи утилиты sho из поставки Renderman Pro Server:
  

sho untitled.new\_channel.0001.tif
  

    Тот, кто до сих пор шарахается от командной строки, может
    использовать подростковую программу типа ACDSee.
  

 \gr{image129}
  

 Как видно, перед нами почти готовая заливка для
    мультипликационного персонажа.
  

 Для
    продвинутых: чтобы получить ещё и
    картунную обводку (cartoon outline), выведите в виде отдельного
    слоя карту нормалей и преобразуйте ее в любимом композере при
    помощи фильтра Sobel. Вы получите достаточно примитивный и не
    всегда удачный, но вполне качественный контур, особенно если
    отрендерите картинку в двойном разрешении, а потом уменьшите её
    обратно – это скроет возможные артефакты. По методам
    нефотореалистического рендеринга с помошью prman можно написать
    пару толстых и тяжёлых книг, но мы ограничимся тремя фразами в этой
    вставке для любознательных – в надежде подтолкнуть вас в нужном
    направлении.
  

 Как-нибудь на досуге поиграйтесь с другими
    переменными из списка Mode; особый интерес представляет переменная
    uniform float \_\_CPUtime, которая в виде картинки покажет вам, на
    рендеринг каких части вашей модели тратится больше всего времени. А
    мы тем временем собираемся узнать, как вывести в отдельный слой
    произвольную переменную из нашего шейдера. Для простоты выберем не
    самый продуктивный, но зато самый интересный способ.
  

 Итак, находим наш рукописный шейдер
    textured\_noise.sl и переименовываем в aov.sl (AOV означает
    Arbitrary Output Variable – общепринятое название используемой нами
    в данном случае техники):
  
\begin{lstlisting}[frame=single, framerule=0pt, framesep=10pt, xleftmargin=10pt, xrightmargin=10pt]

surface
    textured_noise ( float freq=100; )
  

{
  

  Ci = texture("texture.tx", s,
    t)*noise(freq*s,freq*t);
  

 }
\end{lstlisting}
  

 В код этого шейдера нужно внести некоторые
    дополнения, с тем, чтобы рендерер мог увидеть внутренние
    переменные:
  
\begin{lstlisting}[frame=single, framerule=0pt, framesep=10pt, xleftmargin=10pt, xrightmargin=10pt]

surface
    aov ( float freq=100;
  

               output varying color tex = 0;
  

               output varying float noi = 0;)
  

{
  

  tex = texture("texture.tx", s, t);
  

  noi = noise(freq*s,freq*t);
  

  Ci
    = tex*noi;
  

 }
\end{lstlisting}
  

 Исправления, которые мы только что внесли, очевидны
    – мы создали две новые переменные, которые специальным образом
    объявили в заголовке шейдера – и этим переменным присвоили значения
    внутри шейдера.
  

 Дальше начинается самое интересное. Компилируем
    шейдер, копируем получившийся файл aov.slo и используемую текстуру
    texture.tx в папку с майским проектом, прямо в корень (файл с
    исходным кодом шейдера aov.sl можно не копировать).
    Возвращаемся в Maya, и в окне Slim делаем File=>Import
    Appearance=> Import Appearance… (или
    нажимаем Ctrl-I). Выбираем наш aov.slo и вуаля – мы только что
    загрузили написаный руками шейдер в Slim. Импортированый таким
    образом шейдер имеет массу ограничений – например, к его параметрам
    нельзя присоединять другие темплейты, как мы это делали раньше – но
    тем не менее, удобство налицо.
  

 Для
    продвинутых: аналогичную операцию
    выполняет утилита командной строки toslim.
  

 Дальнейшая процедура тривиальна – присоединяем
    шейдер к овечке и рендерим:
  

 \gr{image131}
  

 Во-о-он у неё на ушке наша надпись видна – видите?
    Но мы сейчас интересуемся совсем не этим – мы хотим вывести с
    отдельные слои переменные tex и noi. В очередной раз открываем
    RenderMan Globals, удаляем старый слой и добавляем два новых, как
    показано на скриншоте:
  

 \gr{image133}
  

 Поскольку мы собираемся использовать не глобальные,
    а локальные переменные нашего шейдера, то в списке переменных их
    нет; соответствующие значения в поле Mode придётся скопировать из
    заголовка исходного кода шейдера – а именно,
  

varying
    color tex
  

 и
  

varying
    float noi
  

 Отправляем на рендеринг и получаем в папке rmanpix
    2 новых файла - untitled.noise.0001.tif:
  

 \gr{image135}
  

 в который вывелось значение переменной noi нашего
    шейдера (а поскольку шейдер мы применили только к голове овечки, то
    и видим мы его только на голове) – в этой переменной мы сохранили
    величину нойза до того, как перемножили её на текстуру – и файл
    untitled.texture.0001.tif (в него мы выводили текстуру без
    нойза):
  

 \gr{image137}
  

 В этом месте с людьми обычно случается кондратий и
    начинается вакханалия неконтролируемого рендеринга – в отдельные
    слои выводится всё, что только можно, а затем сцена фактически
    заново собирается в композере, что даёт невиданные ранее
    возможности настройки и изменения финального изображения после
    рендеринга – без самого рендеринга.
  

 Полученные таким образом слои можно также
    использовать в качестве текстуры для шейдера. Например, вы можете
    просчитать ambient occlusion для всей сцены один раз, а потом
    подмешивать его в качестве текстуры в свой шейдер. Или всё-таки
    делать это в композере – на ваше усмотрение.
  

 Для
    любопытных: Среди крупных студий
    до сих пор нет консенсуса в том, какие именно слои и значения каких
    именно переменных нужны и важны для композа. Философия Weta
    Digital, например, в этом случае состоит в том, что рендерить нужно
    в максимально готовом к композитингу виде, с тем, чтобы композеры
    не воссоздавали сцену заново из множества слоёв, повторяя
    фактически работу рендерера. Например, в ходе работы над фильмом
    “I, Robot” Digital Domain и Weta Digital разделили между собой
    планы с участием роботов NS5; при этом Weta считала отдельно только
    металические и пластмассовые части; DD считал diffuse, specular и
    отражение в отдельные файлы, и в результате получая десятки слоёв
    на каждого робота.
  

 Для
    продвинутых: рендеринг по слоям –
    всего лишь один из множества методов борьбы за уменьшение времени
    получения конечного результата при рендеринге. Ещё один похожий на
    него приём из той же серии – это всевозможное кэширование значений
    и последующее использование этих кэшей рендерером без долгого
    пересчёта. Копайтесь в документации к вашему рендереру – например,
    для Prman ключевыми
    словами будут {\it cache и} {\it brick.}

  \section*{Мантра про raytracing.}
  

 В жизни любого TD есть несколько
    извечных дилем. Например, такая – человеческое время стоит гораздо
    дороже, чем время машинное, и ни того, ни другого никогда не
    хватает. Или вот ещё одна – нам нужен мегакачественный продукт, и
    причём ещё вчера.
  

 При чём здесь рейтрейсинг, спросит наш читатель?  
  

 Да при нём, при самом – при результате
    рендеринга.
  

 Существуют множество способов имитировать raytracing и GI. Например, в
    случае зеркального отражения – поставить камеру в необходимую
    точку, а затем результат рендеринга наложить как текстурную карту
    на само зеркало. Или вместо Global Illumination –
    по-умному расположить в сцене уйму источников света.
  

 К сожалению, при всей возможной автоматизации этих
    триков – они всё равно требуют больших человеческих затрат опытных
    сотрудников – а их время стоит что? Правильно – дороже машинного.
    Что ещё хуже – очень часто эти трики не срабатывают или срабатывают
    не так, как хотелось бы. И тогда приходится прибегать к честному
    рейтрейсингу и его родственникам – ambient occlusion и GI.
  

 Без рейтрейсинга в современном кино- и
    видео-бизнесе, как говорится – никуда. Трейсить ambient occlusion
    придётся всё равно – это один из основных приёмов быстрого
    достижения столь желанной фотореалистичности. Но у этой медали есть
    обратная сторона - простое включение рейтрейсинга в сцене
    увеличивает время просчета в разы, каким бы ультраоптимизированным
    рендерером вы не пользовались, и prman не
    исключение, а скорее яркий пример. Поэтому, если вы делаете
    компьютерную графику в действительно больших количествах – для
    рекламы, кино или телевидения – постарайтесь поменьше raytrace’ить
    (а также ambient occlude’ить, global illuminate’ить и final
    gather’ить).

    А если уж совсем припекло, стекло не получается стеклянным, а всего
    лишь тоскливо стекловидным – воспользуйтесь предыдущим советом и
    сразу записывайте результаты в отдельные слои, а потом используйте
    записанное, вместо того, чтобы каждый раз производить длительные
    расчёты заново.
  

 Для
    продвинутых: строго говоря, prman
    до сих пор не является рейтрейсером, даже несмотря на то, что
    подобную функциональность он реализует в полной мере. Основным
    алгоритмом, который он использует при рендеринге, является REYES;
    этот алгоритм полировался и оптимизировался в течение более чем
    десятка лет. Поддержка raytracing была добавлена в Renderman Pro
    Server не так уж и давно, и, судя по описаниям и опубликованным
    статьям, в ядро рендерера она не встроена. Таким образом, если вы
    не используете эту поддержку – всё ещё запускается старый добрый
    ультабыстрый REYES-движок; если вам хочется потрейсить – к нему
    присоединяется новый; движки интенсивно обмениваются данными и даже
    совместно используют некоторые структуры в памяти – но всё равно,
    ядро prman до сих пор остаётся старым, а raytracing – это
    всего-лишь добавление к нему, существенно замедляющее
    процесс.
\chapter*{Собственное всё}
  

 Рассказывая о работе студии Digital Domain над
    фильмом “I, Robot”, мы вскользь упомянули тот факт, что в различные
    слои рендерятся не только различные части робота, но и различные
    части шейдера, в частности diffuse-проход. Как это делается?
    Рассмотрим на примере стандартного шейдера plastic:
  

surface plastic( float Ks=.5, Kd=.5, Ka=1,
  

                  roughness=.1;
    color specularcolor=1 )
  

{
  

     normal Nf;
  

     vector V;
  

     Nf = faceforward( normalize(N), I
    );
  

     V = -normalize(I);
  

     Oi = Os;
  

     Ci = Os * ( Cs * (Ka*ambient() +
    Kd*diffuse(Nf)) +
  

             specularcolor *
    Ks * specular(Nf,V,roughness) );
  

 }
  

 Как видно из заголовка шейдера, никакие переменные
    из него вывести в виде отдельного слоя не получится; для этого
    придётся заниматься модификацией кода. Чтобы получить diffuse в
    виде отдельного слоя, шейдер нужно изменить, например, вот таким
    образом:
  

surface
    plastic( float Ks=.5, Kd=.5, Ka=1,
  

                  roughness=.1; color specularcolor=1;
  

                  output color diffuseC = 0; )
  

{
  

     normal Nf;
  

     vector V;
  

     Nf = faceforward( normalize(N), I
    );
  

     V = -normalize(I);
  

     diffuseC = Kd*diffuse(Nf);
  

     Oi = Os;
  

     Ci = Os * ( Cs * (Ka*ambient() +
    diffuseC) +
  

             specularcolor *
    Ks * specular(Nf,V,roughness) );
  

 }
  

 Как видите, для решения даже простой задачи,
    которую поставили перед нами требования реального продакшна,
    потребовалось пусть небольшое, но изменение стандартного шейдера.
    Отсюда и последует наш совет.
  

 Ничто не даёт бОльшего чуства контроля над
    ситуацией, как собственноручно сделанный, отобранный и настроенный
    инструментарий. В особенности это касается шейдеров, и в ещё
    большей особенности – тех самых шейдеров, которые идут в
    стандартной поставке любого Renderman-совместимого
    рендерера.
  

 Эти шейдеры хороши, подробно описаны, по ним хорошо
    учиться – но потом наступает момент, когда вы хотите чего-то
    нового, существующие возможности вас не устраивают – это означает,
    что пора начинать писать своё.
  

 Так вот совет наш будет таким – не используйте
    стандартные шейдеры. С ними сложно рендерить по слоям, с ними
    сложно настраивать освещение, экпериментировать с многослойными
    текстурами. Напишите свои шейдеры, хорошенько их проверьте в бою,
    прооптимизируйте и настройте под себя – и всегда и везде
    пользуйтесь ними.
  \chapter*{Для
    любознательных – пример из жизни}
  

 Вы думаете, что всё, выше написанное – трудно,
    тяжело и малоприменимо? Вот вам реальный пример из реального
    проекта со слов человека, его делавшего, и пересказаный моими
    собственными словами.
  

 Итак, в наличии имеется сцена с внутренностями
    огромного здания. Изначальная диспозиция такова: здание
    смоделировано полигонами в количестве сколько-то там миллионов
    штук, при этом постоянно обновляется и достраивается. Генерация
    RIBа занимает порядка десяти минут. Здание буквально купается в
    ярком солнечном свете; решётчатая облицовка, колонны и
    перекрытия-галерей отбрасывают чёткие тени на пол и стены. Всё
    остальное освещение в сцене – отражённый и рассеянный свет от стен
    и предметов. Посреди холла установлена огромная отполированная
    статуя робота.
  

 Налицо патовая ситуация – явное проявление болезни
    под названием “нужна radiocity” с осложнениями в виде невозможно
    детализированной модели.
  

 Первым делом в руки берётся топор и из сцены к
    такой-то бабушке удаляется всё, что не будет видно камере в
    конкретном кадре. Имена всех удалённых объектов аккуратно
    записываются в отдельный файлик; таким образом нам будет легче
    убирать их в следующий раз, когда гигантскую модель здания в
    очередной раз обновят.
  

 Далее сцена анализируется и разбивается на части. В
    качестве критерия разбивки используется материал, из которого
    изготовлен предмет – пластик, бетон, стекло, металл, мрамор и так
    далее. Каждая часть экспортируется в RIB, удаляется из сцены и
    заново подключается в сцену с использованием RIBbox’ов – которые
    вешаются на простые кубики и шарики. На этом этапе к сцене
    вернулась возможность поворачиваться во вьюпорте без натужного
    скрипа жёстким диском.
  

 На этом этапа начинается работа над освещением.
    Изначально планировалось использовать новые возможности Renderman
    Pro Server – “запечь” отражённый свет в специальный кэш (irradiance
    cache file), который бы затем использовался в шейдерах – но
    оказалось, что для моделей такой сложности работать с кэшем просто
    невозможно – его размер быстро превышает все разумные рамки, и
    пропорционально падает скорость доступа к нему. Кроме того любое
    изменение модели потребовало бы по крайней мере частичного
    пересчета кэша; изменения, которые потенциально могли внести
    режиссер или супервайзор проекта также потребовали бы длительной и
    трудоемкой перенастройки параметров global illumination и опять
    таки пересчета кэша.
  

 Пришлось хитрить. Один раз для сцены было
    просчитано решение ambient occlusion и записано в виде отдельного
    слоя. Этот слой затем аккуратно подмешивался в источники света – а
    было их в этой сцене не так уж и много, не больше десяти
    spotlight’ов с разными яркостями, направлениями и
    цветом.
  

 Кроме этих спотов, в сцену добавили один источник
    света с тенью, генерируемой при помощи raytrace – к сожалению, без
    него не обошлось; использовались с десяток локальных источников,
    которыми подмазывали, подкрашивали и подсвечивали.
  

 Cначала рендерился основной проход – в нём всё
    стекло выключили. В расчёте ambient occlusion его тоже не было
    видно для камеры – но occlusion-лучи, выстреливаемые из каждой
    точки сцены, стекло видели и таким образом добавили необходимые
    детали в освещение этажей.
  

 И наконец, отдельно считалось стекло.Кроме
    основного слоя, для каждой стекляшки также считались нормали
    поверхности – потом, на этапе сборки сцены из кусков в композере,
    при помощи этих нормалей на гранях между стёклами (то есть в точке
    разрыва и резкого изменения нормалей) усиливались блики.
  

 Абсолютно естественно, что все роботы и люди,
    которых вы видели в кадре – тоже считались отдельно и приклеивались
    уже на композе.
  

 Как видите, все те советы и трики, о которых мы с
    вами поговорили, более чем реальны и применяются в реальном кино- и
    видео-производстве – данный пример с говорит сам за себя. Нам
    остаётся дать последний в нашей главе – и самый главный
    совет.

\chapter*{Думайте о неожиданном}
  

 Экспериментируйте. Пробуйте. Применяйте новые
    интересные техники в необычных ситуациях. 
  

 Интернет битком набит информацией, новостями,
    туториалами. Почти каждый год на сайте renderman.org выкладываются
    новые PDFы с материалами конференции Siggraph и учебного курса по
    Renderman, проходящего на этой конференции. Эти документы – кладезь, неисчерпаемый
    поток знаний, умений и опыта, концентрированное знание множества
    очень умных людей, которые оттачивают все эти приёмы на
    каждодневной практике реального кино- и видео-производства. Их
    можно читать наискосок, проходить по шагам, постоянно сверяясь с
    текстом, пытаться шаблонно использовать в своей работе – а можно
    подойти творчески и начать пробовать и задавать вопросы.
  

 Что будет, если в шейдере в diffuse подмешать
    specular?
  

 А если использовать diffuse в качестве
    specular?
  

 Что будет, если на вход шейдера Blinn повесить
    выход шейдера Noise?
  

 А что получится, если в одном шейдере использовать
    несколько различных спекуляров?
  

 Можно ли использовать карту теней для ускорения
    просчёта сцены (ответ: можно)?
  

 Что получится, если выставить источнику света
    отрицательную величину яркости?
  

 А если связать источник с текстурой, которая бы
    моделировала освещение из окна – понадобится ли само
    окно?
  

 А можно ли сделать такой свет, который был бы синим
    в тени, и белым в остальных областях? А чтобы он содержал только
    specular-составляющую? А чтобы он светил в одну сторону, а тени
    отбрасывал – в другую? А чтобы он мог отбрасывать тени без объекта,
    сам?
  

 Что вы скажете по поводу анимированных текстур,
    которые содержат в себе текстурные координаты другого
    объекта?
  

 А как вам идея насчёт двух карт глубины – одной из
    камеры, а другой из источника света – и определения разницы между
    их значениями?
  

 Можно ли написать шейдер, который покажет, в каком
    месте кривизна поверхности максимальна?
  

 Можно ли хранить в текстуре массив – и потом
    обращаться к нему по индексу?
  

 Можно ли написать шейдер, который в случае ошибки
    отправит вам SMS?
  

 Как известно, наука – это удовлетворение
    собственного любопытства за счёт заказчика. Так вот Chi-Ting – это
    наука. Удовлетворите своё любопытство. Получите удовольствие от
    своих открытий. Поделитесь ними с окружающими.
  

 Кинопроектор показывает 24 кадра в секунду;
    телеэкран – 25. Объем информации, сваливающийся на зрителя,
    настолько велик, что если вы его немного обманете, но он не заметит
    подвоха и поверит в реальность происходящего – то вы
    победили.
\chapter*{Мы строили, строили...}

Вот и подошло к концу наше путешествие в мир рендеринга. Для того, чтобы дать вам необходимые для дальнейшего самостоятельного
продвижения опорные точки, приведу небольшой список литературы. Список будет совсем небольшим, ведь мы не библиография, да и с появлением
Google с Амазоном любой может изготовить такой или подобный список за считанные минуты. Более того, мы, в отличие от остальной книги,
приведём наш список здесь, в конце главы. Раз уж так получилось, что мы так себе потихоньку прошмыгнули внутрь книжки про Maya, то не
будем сильно наглеть и требовать себе места ещё и в общем списке литературы.
  
 Итак:
  
     Первым пунктом у нас будет – эта {\bf книжка}.

      Странное пожелание, не правда ли? И тем не менее – дочитайте её до
      конца, не бросайте на середине. Поверьте мне – я видел {\it неотредактированные} исходники
      некоторых глав – и уже они были великолепны. Вы, скорее всего,
      больше ничего не узнаете про внешние рендереры из оставшихся глав
      книги – и тем не менее.

	\begin{description}
	\item [Advanced RenderMan: Creating CGI for Motion Pictures]   Основа основ
      Renderman. Начиная с азов, достигает нереальных глубин познания.
      Заставляет мозги работать в ранее неизведанных направлениях – лично
      я по прочтении этой книжки заинтересовался и занялся
      нефотореалистичным рендерингом. Интересно, в какую сторону
      повернутся мозги у вас?

	\item [Renderman Companion] Пока не появился Advanced
      RenderMan, эта книжка была основой основ. Капитально устарела,
      к сожалению. Я долго думал, стоит ли включать эту книгу, настолько
      она устарела и настолько она пугающе нацелена в первую очередь на
      программистов – но тем не менее, вот она, в нашем
      списке.

	\item [Texturing and Modeling: A Procedural Approach]    И опять основа основ, на
      этот раз для тех, кто всерьёз заинтересовался процедурными
      текстурами, историей и современностью этого вопроса. Берите
      исключительно третье издание, в нём куча нового материала и все
      картинки – цветные.

	\end{description}

 Ничего, что все эти книжки – на английском? Ну тогда продолжим. Несколько книжек для начинающих, но к сожалению, тоже не на русском:

\begin{description}
	\item [Essential RenderMan fast]   
	\item [Rendering for Beginners: Image synthesis using RenderMan]   Я не читал ни одну из них, поскольку к моменту их
    выхода начинающим уже не был, но хорошо знаком с автором первой и
    слышал много хорошего об авторе второй и посему – мои
    рекомендации.

\end{description}
  
 Далее – небольшой прыжок в сторону, для тех, кто
    хочет узнать, как вообще делаются рендереры:

	\begin{description}
		\item [Production Rendering, Design and Implementation.] Сборник эссе
	      о том, как делаются рендереры вообще и Renderman-совместимые – в
	      частности. В создании сборника участвовали авторы таких рендереров,
	      как Mantra, RenderDotC, Air, Aqsis, LightFlow и ART VPS. Отличная
	      книжка; жаль только, очень дорогая.
	  
		\item [Physically Based Rendering: From Theory to Implementation]  Учебник для
	      тех, кто хочет написать свой собственный рендерер. В процессе
	      изучения этого фолианта читатель имеет возможность проследить
	      процедуру создания своего рендерера, оснащённого под самую завязку
	      самыми современными технологиями, с подробным описанием
	      используемых алгоритмов – и с полным исходным кодом рендерера,
	      напечатанным прямо в книге (с использованием Literate Programming).
	      1000 страниц и почти 2.5 килограмма чистого счастья для тех, кто
	      понимает.  И под конец нашего списка:
	
		\item [Криптономикон.] А это просто книжка, которую я читаю сейчас, в момент написания
	      этого списка литературы. Она не имеет никакого отношения к предмету
	      разговора; просто она мне понравилась, может, кому-то из вас тоже
	      понравится. Не перепутайте случайно с Некрономиконом – это совсем разные
	      вещи.
	
	\end{description}

  
 И это ещё не конец, на самом-то деле. В наш век
    информационных технологий обойтись одними книжками в своих
    исследованиях вам будет – скажем так – трудно. С другой стороны, в
    наш век поглощений, закрытий и беспрестанных обновлений страниц,
    сайтов и компаний, приводить огромные списки URLей тоже вроде как
    неразумно. Так что, обойдёмся – сколько там у нас было книжек,
    девять? – девятью ссылками:

	\begin{description}
  
		\item [     renderman.ru



]
      Отбросив ложную скромность, приведём сайт, который я создал и
      поддерживаю (при неоценимой помощи и поддержке множества людей).
      Всё о Renderman на русском языке, но в первую очередь – сообщество,
      и лишь затем – учебник и помощник.

		\item [     renderman.org



]
      Корень всего. Полуофициальный  сайт, на котором есть всё о
      Renderman. Бонус-трек:
      www.renderman.org/RMR/OtherLinks/blackSIGGRAPH.html - история о
      закрытии Exluna.

		\item [     renderman.pixar.com
]



      Официальный сайт RAT и Prman. Там же – официальный форум
      поддержки.

		\item [     C.g.r.r. 



]
      Старая добрая конференция comp.graphics.rendering.renderman в
      Usenet. В последнее время не самое активное в плане обсуждения
      место, но в подавляющем большинстве случаев – ультраполезный
      источник информации.

		\item [     www.dotcsw.com/links.html ] 




      А вот если вы хотите найти список ВСЕХ Renderman-совместимых
      рендереров, когда-либо существовавших в природе – то вам сюда.
      Воистину, поучительное чтиво.

		\item [     rendermanacademy.com



]
      Академия Renderman. Сайт взял на себя трудную, но почётную
      обязанность учить народ Renderman-у, с чем в меру сил и умения
      справляется.

		\item [     film.nvidia.com



]
      Родина Gelato. Официоз и PR, в основном.

		\item [     jot.cs.princeton.edu
]



       Домашняя
      страница рендерера Jot. Что ещё сказать? Страница как страница,
      бывают и лучше, конечно.

 		\item [    Google.com 



]
      Этот адрес я мог и не писать, конечно. Найдётся всё, понимаешь.

      Бонус-ссылка: 		

\item [scholar.google.com ]
      специализированный поиск по научным работам, диссертациям и
      исследованиям.

\end{description}


 Я хотел бы поблагодарить нескольких человек, чья
    помошь в работе над этой главой не поддаётся трезвой оценке. Это
    Александр Сегаль, Сергей Невшупов, Константин Харитонов – без их
    комментариев, поправок и советов эта глава не состоялась
    бы.

 Отдельная благодарность Юрию Мешалкину, Егору
    Чащину, Александру Халявину и Вячеславу Богданову за
    предоставленные скрипты и плагины и разрешение их
    опубликовать.

 Вы можете найти всех этих людей на сайте
    нашего community – Renderman.ru.

 До встречи – и успехов!



\end{document}  